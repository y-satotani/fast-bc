\chapter{準備}
\label{chap:preliminary}
本章では,後の章に向けての準備をする.はじめに,最短経路や距離を含む,
グラフ理論の基本事項を説明する.
次に,距離や最短経路の数に関する重要な性質を示す.
最後に,媒介中心性とその計算に用いられるペア依存度について説明する.

\section{グラフの数学的表現}
\label{sect:graph-theory}

\textbf{グラフ}とは,二つの集合$V$と$E$の組$G=(V,E)$で,
$E\subseteq V\times V$を満たす,つまり,$V$から元を2個取り出した集合の部分集合である.
$V$を\textbf{頂点集合}とよび,その元を\textbf{頂点}とよぶ.
また,頂点集合の要素数$|V|$を\textbf{頂点数}とよぶ.
$E$を\textbf{辺集合}とよび,その元を\textbf{辺}とよぶ.
また,辺集合の要素数$|E|$を\textbf{辺数}とよぶ.
辺が非順序対ならば$G$は\textbf{無向グラフ}と呼ばれ,その辺を$\{v,w\}$のように表す..
また,辺が順序対ならば,$G$は\textbf{有向グラフ}と呼ばれ,その辺を$(v,w)$のように表す.
無効グラフの辺$e=\{v,w\}$について,頂点$v$と$w$を\textbf{端点}と呼ぶ.
また,有向グラフの辺$e=(v,w)$について,頂点$v$を\textbf{始点},頂点$w$を\textbf{終点}と呼ぶ.
グラフ$G=(V,E)$と$G'=(V',E')$に対して,$V'\subseteq V$かつ$E'\subseteq E$
を満たすとき,$G'$は$G$の\textbf{部分グラフ}であるといい,$G'\subseteq G$と記す.
有向グラフ$G$について,各辺$(v,w)$を逆順にしたグラフを$G$の\textbf{逆}と呼び,$\bar{G}$で表す.
すなわち,$G=(V,E)$のとき,$\bar{G}=(V,\{(w,v)|(v,w)\in E\})$である.

\textbf{重み付きグラフ}とは,すべての辺$(v,w)$もしくは$\{v,w\}$に対して,
辺の重み$l_{vw}\in\mathbb{R}$が定められているグラフである.
以下,特に断らない限り辺の重みは正であるとする.
また,重み付きグラフでないグラフを\textbf{重みなしグラフ}と呼ぶ.

無向グラフにおいて,頂点$v$と辺$e$が\textbf{接続する}とは,$v\in e$つまり,
$e$の端点のどちらかが$v$であることをいう.
頂点$v$と頂点$w$が\textbf{隣接する}とは,$\{v,w\}$が辺集合$E$に属することをいう.
無向グラフ$G$の頂点$v$について,$v$と隣接する頂点の集合を頂点$v$の\textbf{近傍}と
呼び,$\mathcal{N}_G(v)$で表す.また,無向グラフ$G$について$v$の近傍の数
$|\mathcal{N}_G(v)|$を\textbf{次数}と呼び,$k(v)$あるいは$k_G(v)$と記す.
すべての頂点の次数が等しいグラフを\textbf{正則グラフ}と呼ぶ.

無向グラフ$G=(\{1,\ldots,n\},E)$に対して,次の行列$A\in\{0,1\}^{n\times n}$を
\textbf{隣接行列}と呼ぶ.
\begin{equation*}
  a_{ij}=
  \begin{cases}
    1 & (i\text{と}j\text{が隣接する}), \\
    0 & (i\text{と}j\text{が隣接しない}),
  \end{cases}
\end{equation*}

有向グラフ$G$の頂点$v$について,$v$を終点とする辺の始点の集合を\textbf{先行}
(predecessor)と呼び,$\mathcal{P}_G(v)$と記す.すなわち,
$\mathcal{P}_G(v)=\{u|(u,v)\in E\}$である.
同様に,$v$を始点とする辺の終点の集合を\textbf{後継}(successor)と呼び,
$\mathcal{S}_G(v)$と記す.すなわち,$\mathcal{S}_G(v)=\{w|(v,w)\in E\}$である.

有向な\textbf{道グラフ},あるいは単に道とは,$P=(V,E)$のグラフで,
\begin{equation*}
  \begin{aligned}
  V&=\{v_0,v_1,\ldots,v_k\}, \\
  E&=\{(v_0,v_1),(v_1,v_2),\ldots,(v_{k-1},v_k)\}
  \end{aligned}
\end{equation*}
を満たすものである.道グラフ$P=(\{v_i\},\{e_i\})$の頂点および辺の並びを
示すために,次のように頂点と辺を交互に記す.
\[ v_1,e_1,v_2,\ldots,v_{n-1},e_n,v_n \]
重み付き経路グラフに対して,その重みの総和を経路の長さと呼ぶ.

重み付きグラフ$G$上の頂点$s$と頂点$t$について,それらを結ぶ道$P\subseteq G$のうち,
長さが最小のものを\textbf{最短経路}と呼ぶ.$G$の頂点$s$と頂点$t$の最短経路のうち,
ひとつの長さを$s$と$t$の\textbf{最短距離}あるいは\textbf{距離}と呼び,$d_{st}$と表す.
重みなしグラフの距離は,重みをすべて$1$と仮定して計算することが多く,
本稿でも,特に断りがない限りそのように計算する.
$s$と$t$を結ぶ最短経路の数を\textbf{最短経路数}と呼び,$\sigma_{st}$と記す.
$G$が無向グラフであるならば,すべての$s,t\in V$に対して$d_{st}=d_{ts}$かつ
$\sigma_{st}=\sigma_{ts}$である.
以下の議論では,便宜上,すべての$s\in V$に対し$d_{ss}=0$, $\sigma_{ss}=1$とする.

また,$s$から$t$への最短経路の中で頂点$v$を通るものの個数を$\sigma_{st}(v)$で表す.
頂点$s$から他のすべての頂点への最短経路で構成される有向グラフを$G_s=(V_s,E_s)$で表す.
同様に,頂点$s$から頂点$t$へのすべての最短経路で構成される有向グラフを$G_{st}=(V_{st},E_{st})$で表す.
さらに,頂点$s$と頂点$v$について,$G_s$のトポロジカルソートにおいて$v$より後に位置する頂点集合と$v$で
構成される部分グラフを$G_s(v)=(V_s(v),E_s(v))$で表す.すなわち,
$V_s(v)=\{w|w\in V_s,d_{sw}=d_{sv}+d_{vw}\}, E_s(v)=\{(w,x)|(w,x)\in E_s,d_{sx}=d_{sv}+d_{vw}+l_{wx}\}$
である.

最短経路及び最短経路数について,以下の性質が知られている.
この補題は,最短距離および最短経路数の陰な表現を示している.

\begin{lemma}
  \label{lemma:distance-and-geodesics}
  $G=(V,E)$の異なる二頂点$s,t\in V$について,次が成り立つ.
  \begin{equation*}
    \begin{aligned}
      d_{st}&=\min\{l_{si}+d_{it}|i\in\mathcal{N}_G(s)\} \\
      \sigma_{st}&=\sum_{i\in\mathcal{N}_G(s),\,d_{st}=l_{si}+d_{it}}\sigma_{it}
    \end{aligned}
  \end{equation*}
\end{lemma}

\section{媒介中心性とペア依存度}
\label{sect:pairwise-dependency}

本節では,媒介中心性の定義を示した後,効率的な媒介中心性計算アルゴリズムであるBrandesのアルゴリズムを説明する.
媒介中心性の定義は次の通り.

\begin{definition}[Freeman\cite{Freeman1977}]
  \label{def:betweenness-centrality}
  頂点$i$の媒介中心性$B_v$は式\eqref{eq:betweenness-centrality}で定義される.
  \begin{equation}
    B_v=\sum_{s\neq v}\sum_{t\neq {v,s}}\frac{\sigma_{st}(v)}{\sigma_{st}}
    \label{eq:betweenness-centrality}
  \end{equation}
\end{definition}

すなわち,頂点$v$の媒介中心性は$s$から$t$への最短経路の個数とその中で
頂点$v$を通るものの個数の比を$s,t$のすべての組について足し合わせたものである.
したがって,媒介中心性の大きな頂点は多くの最短経路上にあり,この意味で重要度が高いと言える.

すべての頂点の媒介中心性を求める単純な方法は次の通りである.
まず,$s\in V$に対して,Dijkstra法を用いて$s$から他のすべての頂点への最短経路を求める.
その過程で,$s$から他のすべての頂点への最短距離と最短経路数も同時に求める.
次に,互いに異なる3頂点の組$\{s,t,v\}$のすべてに対して,$\sigma_{st}(v)$の値を計算する.
最後に媒介中心性$B_v$ $(v\in V)$を定義の式により計算する.
この方法の計算量は$\mathcal{O}(|V|^3)$であり,大規模なネットワークでは膨大な計算時間が必要になる.

媒介中心性の効率的計算法として広く用いられているのは
Brandes\cite{Brandes2001}によって提案されたアルゴリズムである.
以下,このアルゴリズムについて説明する.
まず,ペア依存度の定義を示す.

\begin{definition}[Brandes\cite{Brandes2001}]
  \label{def:pairwise-dependency}
  頂点$s,t,v$のペア依存度および頂点$s,v$の依存度はそれぞれ
  式\eqref{eq:def-pairwise-dependency}および式\eqref{eq:def-dependency}で定義される.
  \begin{align}
      \delta_{st}(v)&=\frac{\sigma_{st}(v)}{\sigma_{st}}
      \label{eq:def-pairwise-dependency} \\
      \delta_{s\bullet}(v)&=\sum_{t\neq v,s}\delta_{st}(v)
      \label{eq:def-dependency}
  \end{align}
\end{definition}

依存度を使うと媒介中心性は$B_v=\sum_{s\neq v}\delta_{s\bullet}(v)$と変形できる.

$\delta_{s\bullet}(v)$について,次のことが知られている.
\begin{theorem}[Brandes\cite{Brandes2001}]
  \label{th:implicit-pd}
  \begin{equation}
    \label{eq:implicit-pd}
    \delta_{s\bullet}(v)=\sum_{(v,w)\in E_s}\frac{\sigma_{sv}}{\sigma_{sw}}(1+\delta_{s\bullet}(w)).
  \end{equation}
\end{theorem}

定理\ref{th:implicit-pd}は,$s$と$v$の依存度$\delta_{s\bullet}(v)$は
$s$と$w\in\mathcal{S}_{G_s}(v)$の依存度$\delta_{s\bullet}(w)$を用いて陰に計算できることを示している.
Brandesのアルゴリズムは定理\ref{th:implicit-pd}の性質を応用して媒介中心性を高速に計算している.
Brandesのアルゴリズムをアルゴリズム\ref{algo:brandes}に示す.

このアルゴリズムの計算量を考える.まず,ある$s\in V$とすべての$v\in V$に対して
$\sigma_{sv}$を計算するときの計算量は,Dijkstra法より$\mathcal{O}(|V|\log |V|+|E|)$である.
依存度の計算についても,$G$のすべての頂点$v$を$s$からの最短経路の逆向きに一度だけ辿ることにより
$\delta_{s\bullet}(v)$の値を更新できるため,$\mathcal{O}(|E|)$の時間で実行できる\cite{Brandes2001}.
したがって,Brandesのアルゴリズムの計算量は$\mathcal{O}(|V|^2\log |V|+|V||E|)$であり,
それは単純な方法の計算量$\mathcal{O}(|V|^3)$よりも小さい.

\begin{algorithm}[H]
  \caption{Brandesのアルゴリズム}
  \label{algo:brandes}
  \begin{multicols}{2}
    \begin{algorithmic}[1]\small
      \Procedure{Brandes}{$G$}
      \State $B_x\gets 0,\:\forall x\in V$
      \ForAll{$s\in V$}
      \State $S\gets()$ \Comment スタック
      \State $P_w\gets (),\:\forall w\in V$ \Comment リスト
      \State $\sigma_t\gets 0,\:\forall t\in V;\:\sigma_s\gets 1$
      \State $d_t\gets -1,\:\forall t\in V;\:d_s\gets 0$
      \State $Q\gets ()$ \Comment 順位キュー
      \State $\mathrm{push}(Q, s)$
      \While{$\lvert Q\rvert>0$}
      \State $v\gets\mathrm{pop}(Q)$
      \State $\mathrm{push}(S, v)$
      \ForAll{$w\in\mathcal{N}_G(v)$}
      \If{$d_w<0\lor d_v+l_{vw}<d_w$}
      \State $d_w\gets d_v+l_{vw}$
      \State $\sigma_w\gets\sigma_v$
      \State $P_w\gets(v)$
      \State $\mathrm{updatekey}(Q, w, d_w)$
      \EndIf
      \If{$d_w=d_v+l_{vw}$}
      \State $\sigma_w\gets\sigma_w+\sigma_v$
      \State $\mathrm{append}(P_w, v)$
      \EndIf
      \EndFor
      \EndWhile
      \State $\delta_v\gets 0,\:v\in V$
      \While{$\lvert S\rvert>0$}
      \State $w\gets\mathrm{pop}(S)$
      \ForAll{$v\in P_w$}
      \State $\delta_v\gets\delta_v+\frac{\sigma_v}{\sigma_w}(1+\delta_w)$
      \EndFor
      \If{$w\neq s$}
      \State $B_w\gets B_w+\delta_w$
      \EndIf
      \EndWhile
      \EndFor
      \EndProcedure
    \end{algorithmic}
  \end{multicols}
\end{algorithm}

