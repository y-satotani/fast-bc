\chapter{準備}
\label{chap:preliminary}
本章では,後の章で提案するアルゴリズムを説明するための準備をする.はじめに,最短経路などの
グラフ理論の基本事項を説明する.
次に,距離や最短経路の数に関する重要な性質を示す.
最後に,媒介中心性とその計算に用いられるペア依存度について説明する.

\section{グラフの数学的表現}
\label{sect:graph-theory}

\textbf{グラフ}とは,ふたつの集合$V$と$E$の組$G=(V,E)$で,
$E$は$E\subseteq V^2$を満たす.つまり,$E$は$V$の元の二つ組から成る集合の部分集合である.
$V$を\textbf{頂点集合}とよび,その元を\textbf{頂点}とよぶ.
また,頂点集合の要素数$|V|$を\textbf{頂点数}とよぶ.
$E$を\textbf{辺集合}とよび,その元を\textbf{辺}とよぶ.
また,辺集合の要素数$|E|$を\textbf{辺数}とよぶ.
グラフの頂点および辺は,ネットワークにおけるノードとリンクに対応する.
$E$の要素が丸括弧で表される順序対ならば,$G$は\textbf{有向グラフ}と呼ばれ,
波括弧で表される非順序対ならば$G$は\textbf{無向グラフ}と呼ばれる.
有向グラフの辺$e=(v,w)$において,頂点$v$を\textbf{始点},頂点$w$を\textbf{終点}と呼ぶ.
グラフ$G=(V,E)$と$G'=(V',E')$に対して,$V'\subseteq V$かつ$E'\subseteq E$
を満たすとき,$G'$は$G$の\textbf{部分グラフ}であるといい,$G'\subseteq G$と記す.
有向グラフ$G$について,各辺$(v,w)$を逆向きにしたグラフを$G$の\textbf{逆}と呼び,$\bar{G}$で表す.
すなわち,$G=(V,E)$のとき,$\bar{G}=(V,\{(w,v)|(v,w)\in E\})$である.

\textbf{重み付きグラフ}とは,すべての辺$(v,w)$もしくは$\{v,w\}$に対して,
辺の\textbf{重み}$l_{vw}\in\mathbb{R}$が定められているグラフである.
以下,特に断らない限り辺の重みは正とする.
また,辺に重みが定められていないグラフを\textbf{重みなしグラフ}と呼ぶ.
後述する距離などの辺の重みが必要な量を重みなしグラフに対して計算する際,
全ての辺に対して同一の重みが付与されていると仮定する.
特に断らない限り,その重みを$1$とする.

無向グラフにおいて,頂点$v$と辺$e$が\textbf{接続する}とは,$v\in e$つまり,
$e$の端点に$v$が属することをいう.
頂点$v$と頂点$w$が\textbf{隣接する}とは,$(v,w)$または$\{v,w\}$が辺集合$E$に属することをいう.
無向グラフ$G$の頂点$v$について,$v$と隣接する頂点の集合を頂点$v$の\textbf{近傍}と
呼び,$\mathcal{N}_G(v)$あるいは$\mathcal{N}(v)$で表す.また,無向グラフ$G$について$v$の近傍の数
$|\mathcal{N}_G(v)|$を\textbf{次数}と呼び,$k_G(v)$あるいは$k(v)$と記す.
すべての頂点の次数が等しいグラフを\textbf{正則グラフ}と呼ぶ.

有向グラフ$G$の頂点$v$について,$v$を終点とする辺の始点の集合を\textbf{先行}
(predecessor)と呼び,$\mathcal{P}_G(v)$あるいは$\mathcal{P}(v)$と記す.すなわち,
$\mathcal{P}_G(v)=\{u|(u,v)\in E\}$である.
同様に,$v$を始点とする辺の終点の集合を\textbf{後継}(successor)と呼び,
$\mathcal{S}_G(v)$あるいは$\mathcal{S}(v)$と記す.すなわち,$\mathcal{S}_G(v)=\{w|(v,w)\in E\}$である.

有向な\textbf{道グラフ},あるいは単に道とは,$P=(V,E)$のグラフで,
\begin{equation*}
  \begin{aligned}
  V&=\{v_0,v_1,\ldots,v_k\}, \\
  E&=\{(v_0,v_1),(v_1,v_2),\ldots,(v_{k-1},v_k)\}
  \end{aligned}
\end{equation*}
を満たすものである.道グラフ$P=(\{v_i\},\{e_i\})$の頂点および辺の並びを
示すために,次の表記を用いる.
\[ v_1,e_1,v_2,\ldots,v_{n-1},e_n,v_n \]
道グラフ$P=(V,E)$に対して,$E$の各要素の重みの総和を\textbf{長さ}と呼ぶ.

グラフ$G$上の頂点$s$と頂点$t$について,それらを結ぶ道$P\subseteq G$のうち,
長さが最小のものを\textbf{最短経路}と呼ぶ.$G$の頂点$s$と頂点$t$の最短経路のうち,
ひとつの長さを$s$と$t$の\textbf{最短距離}あるいは\textbf{距離}と呼び,$d_{st}$と表す.
距離と同様に,$s$と$t$を結ぶ最短経路の数を\textbf{最短経路数}と呼び,$\sigma_{st}$と記す.
$G$が無向グラフであるならば,すべての$s,t\in S$に対して$d_{st}=d_{ts}$かつ
$\sigma_{st}=\sigma_{ts}$である.
以下の議論では,便宜上,すべての$s\in V$に対し$d_{ss}=0$, $\sigma_{ss}=1$とする.
また,$s$から$t$への最短経路の中で頂点$v$を通るものの個数を$\sigma_{st}(v)$で表す.
頂点$s$から他の頂点へのすべての最短経路で構成される有向グラフを$G_s=(V_s,E_s)$で表す.
同様に,頂点$s$から頂点$t$へのすべての最短経路で構成される有向グラフを$G_{st}=(V_{st},E_{st})$で表す.

\section{最短経路長および最短経路数の性質}
\label{sect:shortest-paths}

最短経路の長さと個数について,いくつかの補題を示す.
まず,ある頂点が最短経路に含まれる条件を示す.

\begin{lemma}[Brandes~\cite{Brandes2001}]
  \label{lemma:inclusion-v}
  $G=(V,E)$の異なる2頂点$s,t \in V$に対して,$s$から$t$への最短経路
  $G_{st}=(V_{st},E_{st})$が$v\in V$を含む,すなわち$v\in V_{st}$であるための
  必要十分条件は,式\eqref{eq:inclusion-v}が成り立つことである.
  \begin{equation}
    d_{st}=d_{sv}+d_{vt}
    \label{eq:inclusion-v}
  \end{equation}
\end{lemma}
\begin{proof}
  はじめに$v=s$の場合を考える.このとき明らかに$v \in V_{st}$であり,
  かつ$d_{sv}+d_{vt}=d_{ss}+d_{st}=d_{st}$より\eqref{eq:inclusion-v}もつねに成り立つ.
  $v=t$の場合も同様である.そこで以下では$v \not\in \{s,t\}$と仮定する.

  $G_{st}$が頂点$v$を含むならば,$G_{st}$の中に$s,\ldots,v,\ldots,t$の順に
  頂点を通る道が存在する.この道の長さは$d_{sv}+d_{vt}$で与えられるので
  \eqref{eq:inclusion-v}が成り立つ.

  次に\eqref{eq:inclusion-v}が成り立つと仮定し,道$s,\ldots,v,\ldots,t$を考える.
  この道は$s$から$t$への最短経路の一つである.
  なぜなら,その長さは$d_{sv}+d_{vt}$で与えられ,\eqref{eq:inclusion-v}より
  $s$から$t$への最短経路長$d_{st}$に等しいからである.
  したがって$v\in V_{st}$が成り立つ.
\end{proof}

補題\ref{lemma:inclusion-v}の証明の考え方を用いれば,ある頂点を通過する最短経路数を導出できる.

\begin{lemma}[Brandes\cite{Brandes2001}]
  $G=(V,E)$の異なる2頂点$s,t \in V$に対して,
  $s$から$t$への最短経路の中で$v$を通るものの個数$\sigma_{st}(v)$は
  次式で与えられる.
  \begin{equation}
    \sigma_{st}(v)=
    \begin{cases}
      \sigma_{sv} \sigma_{vt}, & d_{st}=d_{sv}+d_{vt}\,\text{のとき} \\
      0, & \text{それ以外のとき}
    \end{cases}
  \end{equation}
\end{lemma}

次の補題は,ある辺が最短経路上にあることの必要十分条件を示すものである.

\begin{lemma}
  \label{lemma:inclusion-e}
  $G=(V,E)$の異なる2頂点$s,t\in V$と辺$(u,v)\in E$を考える.$s$から$t$への最短経路
  $G_{st}=(V_{st},E_{st})$が有向辺$(u,v)$を含む,すなわち$(u,v)\in E_{st}$が成り立つための
  必要十分条件は,式\eqref{eq:inclusion-e}が成り立つことである.
  \begin{equation}
    d_{st}=d_{su}+l_{uv}+d_{vt}
    \label{eq:inclusion-e}
  \end{equation}
\end{lemma}
\begin{proof}
  はじめに$u=s$, $v=t$の場合を考える.このとき明らかに$G_{st}$は
  有向辺$(u,v)$を含み,かつ$d_{su}+l_{uv}+d_{vt}=d_{ss}+l_{st}+d_{tt}=l_{st}=d_{st}$
  より\eqref{eq:inclusion-e}を満たす.

  次に$u=s$, $v\neq t$の場合を考える.このとき$(u,v)\in E_{st}$であるための
  必要十分条件は$G_{st}$が$v$を含むことである.これは補題~\ref{lemma:inclusion-v}より
  $d_{st}=d_{sv}+d_{vt}$と等価であり,さらに右辺$d_{sv}+d_{vt}$は\eqref{eq:inclusion-e}の右辺と等しい.
  $u\neq s$, $v=t$の場合も同様である.

  そこで以下では$s,t,u,v$がすべて異なると仮定する.
  $G_{st}$が有向辺$(u,v)$を含むならば,$G_{st}$の中に
  $s,\ldots, u,v,\ldots,t$
  の順に頂点を通る有向道が存在する.
  この道の長さは$d_{su}+l_{uv}+d_{vt}$で与えられるので\eqref{eq:inclusion-e}が成り立つ.

  次に\eqref{eq:inclusion-e}が成り立つと仮定する.
  すると,$s,\ldots, u,v,\ldots,t$である道が存在する.
  この道は$s$から$t$への最短経路の一つである.なぜなら,その長さは$d_{su}+l_{uv}+d_{vt}$で与えられ,
  \eqref{eq:inclusion-e}より$s$から$t$への最短経路長$d_{st}$に等しいからである.
  したがって,$(u,v) \in E_{st}$が成り立つ.
\end{proof}

補題\ref{lemma:inclusion-e}の証明の考え方を用いれば,ある辺を通る最短経路数を導出できる.

\begin{lemma}
  $G=(V,E)$の異なる2頂点$s,t \in V$に対して,
  $s$から$t$への最短経路の中で有向辺$(u,v)$を通るものの個数を
  $\sigma_{st}(u,v)$とおくと,それは次式で与えられる.
  \begin{equation*}
    \sigma_{st}(u,v)=
    \begin{cases}
      \sigma_{su}\sigma_{vt}, & d_{st}=d_{su}+l_{uv}+d_{vt}\,\text{のとき} \\
      0, & \text{それ以外のとき}
    \end{cases}
  \end{equation*}
\end{lemma}

さらに,次の補題は距離および最短経路数をほかの頂点とのもので陰に表す.

\begin{lemma}
  \label{lemma:distance-and-geodesics}
  $G=(V,E)$の異なるに頂点$s,t\in V$について,次式が成り立つ.
  \begin{equation*}
    \begin{aligned}
      d_{st}&=\min\{l_{sv}+d_{vt}|v\in\mathcal{S}_G(s)\} \\
      \sigma_{st}&=\sum_{v\in\mathcal{S}_G(s),\,d_{st}=l_{sv}+d_{vt}}\sigma_{vt}
    \end{aligned}
  \end{equation*}
\end{lemma}
%\begin{proof}
%  すべての辺の重みは正であるため,頂点$s$から頂点$v$までの最短経路の最後の辺は,
%  $d_{su}<d_{su}$が成立する辺集合に含まれる辺$(u,v)$である.これは,補題1
%  \par $G=(V,E$の互いに異なる3頂点$s,t,v$に対して,$s$から$t$への最短経路の集合
%  $G_{st}=(V_{st},E_{st})$が$v$を含む,すなわち,$v\in E_{st}$が成り立つための
%  必要十分条件は$d_{st}=d_{sv}+d_{vt}$である.
%
%  より頂点$s$から頂点$v$までの最短経路の数は,頂点$s$から頂点$u$までの
%  最短経路の数と等しいと言える.
%\end{proof}

\section{媒介中心性とペア依存度}
\label{sect:pairwise-dependency}

本節では,媒介中心性の定義からその効率的な計算方法を説明する.

\begin{definition}[Freeman\cite{Freeman1977}]
  \label{def:betweenness-centrality}
  頂点$i$の媒介中心性$B_v$は式\eqref{eq:betweenness-centrality}で定義される.
  \begin{equation}
    B_v=\sum_{s\neq v}\sum_{t\neq {v,s}}\frac{\sigma_{st}(v)}{\sigma_{st}}
    \label{eq:betweenness-centrality}
  \end{equation}
\end{definition}

定義より,頂点$v$の媒介中心性は$s$から$t$への最短経路の個数とその中で
頂点$v$を通るものの個数の比を$s,t$のすべての組について足し合わせたものと言える.
したがって,媒介中心性の大きな頂点は多くの最短経路上にあり,この意味で重要度が高いと言える.

すべての頂点の媒介中心性を求める単純な方法は以下の通りである.
まず,$s\in V$に対して,Dijkstra法を用いて$s$から他のすべての頂点への最短経路を求める.
その過程で,$s$から他のすべての頂点への最短距離と最短経路数も同時に求める.
次に,互いに異なる3頂点の組$\{s,t,v\}$のすべてに対して,$\sigma_{st}(v)$の値を計算する.
最後に媒介中心性$B_v$ $(v\in V)$を定義の式により計算する.
この方法の時間計算量は$\mathcal{O}(|V|^3)$であり,大規模なネットワークでは膨大な計算時間が必要になる.

媒介中心性の効率的計算法として最も広く用いられているのは
Brandes~\cite{Brandes2001}によって提案されたアルゴリズムである.
以下,Brandesのアルゴリズムについて説明する.
まず,ペア依存度および依存度の定義を示す.

\begin{definition}[Brandes\cite{Brandes2001}]
  \label{def:pairwise-dependency}
  頂点$s,t,v$のペア依存度および頂点$s,v$と他の頂点の依存度はそれぞれ
  式\eqref{eq:def-pairwise-dependency}と
  式\eqref{eq:def-sum-pairwise-dependency}で定義される.
    \begin{align}
      \delta_{st}(v)&=\frac{\sigma_{st}(v)}{\sigma_{st}}
      \label{eq:def-pairwise-dependency} \\
      \delta_{s\bullet}(v)&=\sum_{t\neq v,s}\delta_{st}(v)
      \label{eq:def-sum-pairwise-dependency}
    \end{align}
\end{definition}

ペア依存度および依存度を使うと,媒介中心性は次式で表される.
\begin{equation*}
  \begin{aligned}
    B_v&=\sum_{s\neq v}\sum_{t\neq v,s}\frac{\sigma_{st}(v)}{\sigma_{st}}
    &=\sum_{s\neq v}\sum_{t\neq v,s}\delta_{st}(v)
  \end{aligned}
\end{equation*}

依存度$\delta_{s\bullet}(v)$を効率的に計算する方法について,次の性質が知られている.

\begin{lemma}
  \label{lemma:implicit-dependency-part}
  グラフ$G=(V,E)$の異なる頂点$s,t$について,$\sigma_{st}=1$ならば,
  頂点$v$の$s$に対するペア依存度は式\eqref{eq:implicit-dependency-part}を満たす.
  \begin{equation}
    \label{eq:implicit-dependency-part}
    \delta_{s\bullet}(v)=\sum_{(v,w)\in E_s}(1+\delta_{s\bullet}(w))
  \end{equation}
\end{lemma}
\begin{proof}
  この補題は,図\ref{fig:implicit-dependency-1}のようにすべての最短経路の辺と頂点を形成している
  頂点$s$の木構造であることを表している.したがって,頂点$s$から頂点$t$までの
  すべての経路もしくは,それ以外の経路上に頂点$v$は存在する.
  例えば,頂点$v$は$w_1,w_2,w_3$の最短経路上に存在する.
\end{proof}

\begin{figure}[tb]
  \centering
  \def\svgwidth{.35\linewidth}
  \input{implicit-dependency-1.pdf_tex}
  \caption{頂点$s$の木構造}
  \label{fig:implicit-dependency-1}
\end{figure}

さらに$\sigma_{st}>1$のような一般的な場合も,次の性質が知られている.

\begin{theorem}
  \label{th:implicit-dependency}
  グラフ$G=(V,E)$の異なる頂点$s,t$について,
  頂点$v$の$s$に対するペア依存度は式\eqref{eq:implicit-dependency}を満たす.
  \begin{equation}
    \label{eq:implicit-dependency}
    \delta_{s\bullet}(v)=\sum_{(v,w)\in E_s}\frac{\sigma_{sv}}{\sigma_{sw}}(1+\delta_{s\bullet}(w)).
  \end{equation}
\end{theorem}
\begin{proof}
  $E_{st}$の上に$v$があるような$t$に対してのみ$\delta_{st}(v)>0$が成り立ち,このような
  最短経路は$v\in\mathcal{P}_{E_{st}}(w)$を満たす辺$(v,w)$をひとつだけもつ.
  このような状況は,図\ref{fig:implicit-dependency-2}のように,
  図\ref{fig:implicit-dependency-1}の場合より複雑である.

  辺$e\in E$を含むように拡張したペア依存度を$\delta_{st}(v,e)=\sigma_{st}(v,e)/\sigma_{st}$で定義する.
  ただし,$\sigma_{st}(v,e)$は$s$から$t$への最短経路の中で$v$と$e$を通るものの個数とする.
  つまり,
  \[ \delta_{s\bullet}(v)=\sum_{t\in V}\delta_{st}(v)=\sum_{t\in V}\sum_{(v,w)\in E_s}\delta_{st}(v,\{v,w\})=\sum_{(v,w)\in E_s}\sum_{t\in V}\delta_{st}(v,(v,w)). \]

  頂点$w$は$\mathcal{P}_{E_s}(w)$を満たすとする.すると,$s$から$w$への
  $\sigma_{sw}$個の最短経路のうち,$\sigma_{sv}$個の最短経路は$s$から$v$を通り,
  $(v,w)$を通る.つまり,$s$と$t$と$v$と$(v,w)$のペア依存度は,次で求められる.
  \begin{equation*}
    \delta_{st}(v,\{v,w\})=
    \begin{cases}
      \cfrac{\sigma_{sv}}{\sigma_{sw}}, & t=w\text{のとき} \\
      \cfrac{\sigma_{sv}}{\sigma_{sw}}\cdot\cfrac{\sigma_{st}(w)}{\sigma_{st}} & t\neq w\text{のとき}
    \end{cases}
  \end{equation*}

  これを上式に代入すると,
  \begin{align*}
    \delta_{s\bullet}(v)&=\sum_{t\neq s,v}\cfrac{\sigma_{st(v)}}{\sigma_{st}} \\
    &=\sum_{t\neq s,v}\sum_{(v,w)\in E_s}\cfrac{\sigma_{st}(v,(v,w))}{\sigma_{st}} \\
    &=\sum_{(v,w)\in E_s}\sum_{t\neq s,v}\cfrac{\sigma_{st}(v,(v,w))}{\sigma_{st}} \\
    &=\sum_{(v,w)\in E_s}\left(\sum_{t\neq s,v}\cfrac{1}{\sigma_{st}}\cdot\frac{\sigma_{sv}}{\sigma_{sw}}\cdot\sigma_{st}(w)+\cfrac{\sigma_{sv}}{\sigma_{sj}}\right) \\
    &=\sum_{(v,w)\in E_s}\cfrac{\sigma_{sv}}{\sigma_{sw}}\cdot\left(1+\sum_{t\neq s,v,w}\cfrac{\sigma_{st}(v)}{\sigma_{st}}\right) \\
    &=\sum_{(v,w)\in E_s}\cfrac{\sigma_{sv}}{\sigma_{sw}}\cdot(1+\delta_{s\bullet}(w))
  \end{align*}
  を得る.
\end{proof}

\begin{figure}[tb]
  \centering
  \def\svgwidth{.35\linewidth}
  \input{implicit-dependency-2.pdf_tex}
  \caption{頂点$s$の木構造}
  \label{fig:implicit-dependency-2}
\end{figure}

Brandesのアルゴリズムは定理\ref{th:implicit-dependency}の性質を応用して媒介中心性を高速に計算している.
Brandesのアルゴリズムの具体的な手順を説明する.まず,Dijkstra法に基づいて,始点$s$から各頂点$t$への距離$d_{st}$および最短経路数$\sigma_{st}$を求める.
この際,距離$d_{st}$の昇順に頂点$t$をプッシュしたスタック$\texttt{S}$および,各頂点$t$に対する$E_{s}$上の先行$\mathcal{P}_{E_s}(t)$である$\texttt{P}_t$を計算する.
最短経路の計算後,$\texttt{S}$から要素を取り出し定理\ref{th:implicit-dependency}による$\delta_{s\bullet}(v)$の計算を,$\texttt{S}$が空になるまで繰り返す.
Brandesのアルゴリズムをアルゴリズム\ref{alg:brandes}に示す.

このアルゴリズムの計算量を考える.まず,ある$s\in V$とすべての$v\in V$に対して
$\sigma_{sv}$を計算するときの計算量は,Dijkstra法より$\mathcal{O}(|V|\log |V|+|E|)$である.
ペア依存度の積算についても,$G$のすべての頂点$v$を$s$からの最短経路の逆向きに一度だけ辿ることにより
$\delta_{s\bullet}(v)$の値を更新できるため,$\mathcal{O}(|E|)$の時間で実行できる\cite{Brandes2001}.
したがって,すべての$s\in V$に対するアルゴリズムの計算量は$\mathcal{O}(|V|^2\log |V|+|V||E|)$であり,
それは$\mathcal{O}(|V|^3)$よりも小さい.

\begin{algorithm}[H]
  \caption{Brandesのアルゴリズム}
  \label{alg:brandes}
  \begin{multicols*}{2}
    \begin{algorithmic}[1]\small
      \Procedure{Betweenness}{$G$}
      \State $B_x\gets 0,\:\forall x\in V$
      \ForAll{$s\in V$}
      \State $\texttt{S}\gets\varnothing$
      \State $\texttt{P}_w\gets\varnothing,\:\forall w\in V$
      \State $\sigma_{st}\gets 0,\:\forall t\in V;\:\sigma_{ss}\gets 1$
      \State $d_{st}\gets\infty,\:\forall t\in V;\:d_{st}\gets 0$
      \State $\texttt{PQ}\gets\varnothing$
      \State $s$を優先度を$0$として$\texttt{PQ}$にプッシュ
      \While{$\lvert\texttt{PQ}\rvert>0$}
      \State 優先度最大の要素$v$を$\texttt{PQ}$からポップ
      \State $v$を$\texttt{S}$にプッシュ
      \ForAll{$w\in\mathcal{S}_G(v)$}
      \If{$d_{sv}+l_{vw}<d_{sw}$}
      \State $d_{sw}\gets d_{sv}+l_{vw}$
      \State $\sigma_{sw}\gets\sigma_{sv}$
      \State $\texttt{P}_w\gets\{v\}$
      \State $\texttt{PQ}$の$w$の優先度を$d_{sw}$に変更
      \EndIf
      \If{$d_{sw}=d_{sv}+l_{vw}$}
      \State $\sigma_{sw}\gets\sigma_{sw}+\sigma_{sv}$
      \State $\texttt{P}_w\gets\texttt{P}_w\cup\{v\}$
      \EndIf
      \EndFor
      \EndWhile
      \State $\delta_{s\bullet}(v)\gets 0,\:v\in V$
      \While{$\lvert\texttt{S}\rvert>0$}
      \State $\texttt{S}$から要素$w$をポップする
      \ForAll{$v\in \texttt{P}_w$}
      \State $\delta_{s\bullet}(v)\gets\delta_{s\bullet}(v)+\sigma_{sv}/\sigma_{sw}(1+\delta_{s\bullet}(w))$
      \EndFor
      \If{$w\neq s$}
      \State $B_w\gets B_w+\delta_{s\bullet}(w)$
      \EndIf
      \EndWhile
      \EndFor
      \EndProcedure
    \end{algorithmic}
  \end{multicols*}
\end{algorithm}

