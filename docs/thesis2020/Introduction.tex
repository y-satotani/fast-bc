\chapter{序論}
\label{chap:introduction}

本章では研究背景としてネットワーク科学および,中心性とその一種である媒介中心性について
簡単に説明した後,関連研究として媒介中心性を計算するアルゴリズムを説明する.
最後に,本研究の目的を示して,本稿の構成を説明する.

\section{ネットワーク科学}

グラフ理論から派生したネットワーク科学が注目を集めている.
ネットワーク科学とは,つながりから成る集団全体の特性や,つながりの中の個々の特性の理解を目的とした科学である.
ネットワーク科学の強みは,物とそれらの間の関係が定義できれば,あらゆる分野に応用できることである.
例えば,人のつながりや,原子の結合を対象にできる.

初期のネットワーク科学は学校や職場,あるいは低分子といった小規模な集まりを対象としていたが,
計算機の発展と普及により,対象とできる規模は大きくなった.例えば,感染症の世界的流行のシミュレーション
や国レベルの道路ネットワークの分析,ソーシャルネットワーキングサイトの分析が挙げられる.

分析が大規模になった一方,現実のネットワークの特徴を上手く表現するようなモデルが開発されてきた.
その中でも著名なものは,WattsとStrogatzのスモールワールドネットワーク\cite{Watts1998}と,
Barab{\'{a}}siとAlbertのスケールフリーネットワーク\cite{Barabasi1999}である.

\section{中心性}

巨大で複雑なネットワークを分析するうえで,ネットワークのノードの重要さを測ることは古くからある課題である.
例えば,ソーシャルネットワーキングサービスにおける,いわゆるインフルエンサーや,スケールフリーネットワークにおけるハブといった重要なノードは
ネットワーク全体に影響を与えうる.ゆえにそのようなノードを発見することは,応用上きわめて有益である.

ノードの重要さについて定量的に議論するため,中心性の概念が導入された.
現在まで多くの中心性が提案されてきた.例えば,隣接しているノードの数である次数をそのまま使った
次数中心性や,他の頂点との距離の総和の逆数を使った近接中心性\cite{Beauchamp1965},
ネットワークの隣接の様子を行列にした隣接行列の固有ベクトルを使う固有ベクトル中心性
\cite{Bonacich1991}などが提案された.

その中でも媒介中心性\cite{Freeman1977}は,
すべての最短経路のうち,そのノードを通るものの割合によってノードの重要さを決定する中心性である.
第\ref{chap:preliminary}で詳細を説明するが,ノード$v$の媒介中心性は
どの程度あるかを表した中心性である.具体的には,ノードペア$(s,t)$の最短経路数$\sigma_{st}$
と対象ノード$v$を通る最短経路数$\sigma_{st}(v)$の比を全ノードペアについて足し合わせたものである.
あるノードの媒介中心性が高いならば,そのノードには比較的多くの最短経路が通っていることとなり,
道路ネットワークや通信ネットワークにおいて重要なノードであると言える.

\section{関連研究}

実用上の観点から,媒介中心性を高速に計算することは,それが提案されてから重要な課題である.
媒介中心性をその定義に従って計算した場合,
その時間計算量はノード数を$|V|$とすると$\mathcal{O}(|V|^3)$である.
計算量を小さくするため,Brandesはノード数$|V|$とリンク数$|E|$として,
計算量が$\mathcal{O}(|V|^2\log|V|+|V||E|)$であるアルゴリズムを開発した
\cite{Brandes2001}.Brandesのアルゴリズムは第\ref{chap:preliminary}で説明する
依存度と呼ばれる量を陰に求め,それを積算することで媒介中心性を高速に計算している.

Brandesのアルゴリズムが開発されて以来,そのアイデアを発展させたアルゴリズムが多数開発された.
例えば,前処理によりネットワークを簡単にする方法\cite{Puzis2012,Bentert2018}や,
分割統治法の考え方を用いた方法\cite{Erdos2015}がある.
また,処理を並列に行い高速化を図った方法\cite{Bader2006,Tan2009,Edmonds2010,Bernaschi2016}や,
媒介中心性の近似値を効率的に求める方法\cite{Brandes2007,Bader2007,Pfeffer2012,Yoshida2014}も
開発されている.
これらのアルゴリズムは,ノードやリンクの追加や削除が行われない,時不変ネットワークに対して高速に
媒介中心性を計算する.

一方,現実のネットワークはノードやリンクの追加や削除が起こる時変ネットワーク\cite{Holme2012}
である.時変ネットワークに対しては,Brandesのアルゴリズムで始めから計算するより,
変更に対する差分のみを計算する方がより効率的であると期待できる.
そのような考えから,時変ネットワークの媒介中心性を効率的に計算する方法も提案されている.
Minimum union cycleと呼ばれる閉路の集合と媒介中心性を保持し,変更に対してそれらを更新する方法
\cite{Lee2012,Singh2015}や,
Hypergraph sketch\cite{Yoshida2014}を保持,更新することで媒介中心性の近似値を求める方法
\cite{Hayashi2015},
時不変ネットワークに対する近似法を応用した方法\cite{Bergamini2015a,Bergamini2015b}が提案された.

さらに,媒介中心性と共に最短経路を保持し,それらを更新する方法も提案されている.
Kasらは,RamalingamとRepsの最短経路更新法\cite{Ramalingam1996}を応用して,
辺挿入時の媒介中心性を更新する方法を開発した\cite{Kas2013}.
また,Nasreらは,Kargerらの\cite{Karger1993}の方法を時変ネットワークに適用して
辺挿入時の媒介中心性を更新する方法を開発し\cite{Nasre2014a},
DemetrescuとItalianoの最短経路更新法\cite{Demetrescu2003}を
基にして辺削除時の媒介中心性を更新する方法を開発した\cite{Nasre2014b}.
また,PontecorviとRamachandranは,DemetrescuとItalianoの方法を応用して
頂点追加時の媒介中心性を更新する方法を開発した\cite{Pontecorvi2015}.
さらに,Bergaminiらは,RamalingamとRepsのアルゴリズムを基に,
辺挿入時の媒介中心性更新アルゴリズムを開発した\cite{Bergamini2017}.

\section{研究目的}

現在まで多くの媒介中心性更新アルゴリズムが開発されてきたが,
RamalingamとRepsの方法に基づく辺削除時の媒介中心性更新アルゴリズムは知られていない.
そこで,本研究ではRamalingamとRepsの最短経路更新法に基づく変削除時の媒介中心性更新法を提案する.
また,その方法の有用性を理論解析および数値実験によって検証する.

本稿の以降の構成は次のとおりである.
続く第\ref{chap:preliminary}章でグラフの数学的表現などの説明をして,
第\ref{chap:algorithm}章でRamalingamとRepsの方法に基づく,
辺削除時の媒介中心性更新アルゴリズムを提案する.
さらに第\ref{chap:complexity-analysis}章で提案手法の性能を解析的に,
第\ref{chap:experiment}章で提案手法の性能を実験的に評価する.
最後に第\ref{chap:conclusion}章で結論を述べる.
