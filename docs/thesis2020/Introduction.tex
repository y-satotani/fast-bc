\chapter{序論}
\label{chap:introduction}

本章では研究背景としてネットワーク科学と,ネットワークのノードの重要さを示す中心性について
簡単に説明した後,ネットワークに操作がされたときに中心性一種である媒介中心性を更新する
アルゴリズムを説明する.最後に,本研究の目的を示して,本稿の構成を説明する.

\section{ネットワーク科学}

グラフ理論から派生したネットワーク科学が注目を集めて久しい.
ネットワーク科学の強みは,物とそれらの間の関係が定義できれば,あらゆる分野に応用できることである.
例えば,社会科学における人のつながりや,化学における原子や分子の結合を対象にできる.

初期のネットワーク科学は学校や職場,あるいは低分子といった小規模な集まりを対象としていたが,
計算機の発展と普及により,対象とできる規模は大きくなった.例えば,感染症の世界的流行のシミュレーション
や国レベルの道路ネットワークの分析,ソーシャルネットワーキングサイトの分析が挙げられる.

分析が大規模になった一方,現実のネットワークの特徴を上手く表現するようなモデルが開発されてきた.
その中でも著名なものは,WattsとStrogatzのスモールワールドネットワーク\cite{Watts1998}と,
Barab{\'{a}}siとAlbertのスケールフリーネットワーク\cite{Barabasi1999}である.

\section{中心性}

複雑ネットワークを分析するうえで,ネットワークのノードの重要さに関して議論されることは自然である.
例えば,いわゆるインフルエンサーや,スケールフリーネットワークにおけるハブといった重要なノードは
ネットワーク全体に影響を与えうる.

ノードの重要性に関して定量的に議論するため,中心性の概念が導入された.
現在まで多くの中心性が考案されてきた.例えば,隣接しているノードの数である次数をそのまま使った
次数中心性や,他の頂点との距離の総和の逆数を使った近接中心性\cite{Beauchamp1965},
ネットワークの隣接の様子を行列にした隣接行列の固有ベクトルを使う固有ベクトル中心性
\cite{Bonacich1991}などが定義された.

その中でも媒介中心性\cite{Freeman1977}は,そのノードの上に最短経路が
どの程度あるかを表した中心性である.具体的には,ノードペア$(s,t)$の最短経路数$\sigma_{st}$
と対象ノード$v$を通る最短経路数$\sigma{st}(v)$の比を全ノードペアについて足し合わせたものである.
あるノードの媒介中心性が高いならば,そのノードには比較的多くの最短経路が通っていることとなり,
そのような意味で重要なノードであることを表す.

実用上の観点から,媒介中心性を高速に計算することは重要である.その定義に従って媒介中心性を計算した場合,
その計算量はノード数を$|V|$とすると$\mathcal{O}(|V|^3)$であるが,Brandesはリンク数$|E|$に
ついて,計算量が$\mathcal{O}(|V|^2\log|V|+|V||E|)$であるアルゴリズムを開発した
\cite{Brandes2001}.Brandesのアルゴリズムは依存度と呼ばれる量を陰に求め,それを積算することで
媒介中心性を高速に計算している.

\section{関連研究}

Brandesのアルゴリズムが開発されて以来,そのアイデアを発展させたアルゴリズムが多数開発された.
例えば,前処理によりネットワークを簡単にする方法\cite{Puzis2012,Bentert2018}や,
分割統治法の考え方を用いた方法\cite{Erdos2015}がある.
また,処理を並列に行い高速化を図った方法\cite{Bader2006,Tan2009,Edmonds2010,Bernaschi2016}や,
媒介中心性の近似値を効率的に求める方法\cite{Brandes2007,Bader2007,Pfeffer2012,Yoshida2014}も
開発されている.

一方,現実のネットワークは時間とともにリンクが出現したり消滅したりする時変ネットワーク\cite{Holme2012}
であることから,そのようなネットワークの媒介中心性を効率的に計算する方法も開発されている.
Minimum union cycleと呼ばれる閉路の集合と媒介中心性を保持し,変更に対してそれらを更新する方法
\cite{Lee2012,Singh2015}や,
Hypergraph sketch\cite{Yoshida2014}を保持,更新することで媒介中心性の近似値を求める方法
\cite{Hayashi2015},
時不変ネットワークに対する近似法を用いた方法\cite{Bergamini2015a,Bergamini2015b}が開発されている.

さらに,最短経路やそれに関する値を保持し,更新することで媒介中心性を更新する方法も提案されている.
Kasらは,RamalingamとRepsの最短経路更新法\cite{Ramalingam1996}を応用して,
辺挿入時の媒介中心性を更新する方法を開発した\cite{Kas2013}.
また,Nasreらは,Kargerらの\cite{Karger1993}の方法を時変ネットワークに適用して
辺挿入時の媒介中心性を更新する方法を開発し\cite{Nasre2014a},
DemetrescuとItalianoの最短経路更新法\cite{Demetrescu2003}を
基にして辺削除時の媒介中心性を更新する方法を開発した\cite{Nasre2014b}.
また,PontecorviとRamachandranは,DemetrescuとItalianoの方法を応用して
頂点追加時の媒介中心性を更新する方法を開発した\cite{Pontecorvi2015}.
さらに,Bergaminiらは,RamalingamとRepsのアルゴリズムを基に,
辺挿入時の媒介中心性更新アルゴリズムを開発した\cite{Bergamini2017}.

\section{研究目的}

現在まで多くの媒介中心性更新アルゴリズムが開発されてきたが,
RamalingamとRepsの方法に基づく辺削除時の媒介中心性更新アルゴリズムは知られていない.
そこで,本研究ではRamalingamとRepsの最短経路更新法に基づく変削除時の媒介中心性更新法を提案する.
また,その方法の有用性を理論解析および数値実験によって検証する.

本稿の以降の構成は次のとおりである.
続く第\ref{chap:preliminary}章でグラフの数学的表現などの説明をして,
第\ref{chap:algorithm}章でRamalingamとRepsの方法に基づく,
辺削除時の媒介中心性更新アルゴリズムを提案する.
さらに提案したアルゴリズムの性能を第\ref{chap:complexity-analysis}章で解析的に,
第\ref{chap:experiment}章で実験的に評価する.
最後に第\ref{chap:conclusion}章で結論を述べる.
