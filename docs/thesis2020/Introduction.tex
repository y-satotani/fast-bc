\chapter{序論}
\label{chap:introduction}

本章では研究背景としてネットワーク科学および,中心性とその一種である媒介中心性について
説明する.その後,関連研究として媒介中心性を計算するアルゴリズムを説明する.
最後に,本研究の目的を示して,本稿の構成を説明する.

\section{ネットワーク科学}

グラフ理論から派生したネットワーク科学が注目を集めている.
ネットワーク科学とは,つながりから成る集団全体の特性や,つながりの中の個々の特性の理解を目的とした科学である.
ネットワーク科学の強みは,個とそれらの間の関係が定義できれば,あらゆる分野に応用できることである.
例えば,人のつながりを対象にしたり,原子の結合を対象にしたりできる.

初期のネットワーク科学は学校や職場,あるいは低分子といった小規模な集団を対象としていたが,
計算機の発展と普及により,対象にできる集団の規模は大きくなった.例えば,感染症の世界的流行のシミュレーション
や国レベルの道路ネットワークの分析,ソーシャルネットワーキングサイトの分析への応用が可能になった.

分析が大規模になった一方,現実のネットワークの特徴を上手く再現するようなモデルが提案されてきた.
その中でも著名なものは,スモールワールド性を再現したWattsとStrogatzのスモールワールドネットワーク\cite{Watts1998}と,
次数のスケールフリー性を再現したBarab{\'{a}}siとAlbertのスケールフリーネットワーク\cite{Barabasi1999}である.

\section{中心性}

巨大で複雑なネットワークを分析するうえで,ネットワークのノードの重要さを測ることは古くからある課題である.
例えば,ソーシャルネットワーキングサービスにおける,いわゆるインフルエンサーや,スケールフリーネットワークにおけるハブといった重要なノードは
ネットワーク全体に影響を与えうる.ゆえにそのようなノードを発見することは,応用上きわめて有益である.

ノードの重要さを定量的に議論するため,\textbf{中心性}の概念が導入された.
重要さの意味が様々に考えられるため,現在まで多くの中心性が提案されてきた.
例えば,隣接しているノードの数である次数をそのまま使った次数中心性や,
他の頂点との距離の総和の逆数を使った近接中心性\cite{Beauchamp1965},
ネットワークのノードの隣接の状況を行列として表した隣接行列の固有ベクトルを使う固有ベクトル中心性
\cite{Bonacich1991}などが提案された.

その中でも媒介中心性\cite{Freeman1977}は,最短経路に着目した中心性である.
より具体的には第\ref{chap:preliminary}章で説明するが,ノード$v$の媒介中心性の値は
ノードペア$(s,t)$の最短経路数$\sigma_{st}$に占める対象ノード$v$を通る最短経路数
$\sigma_{st}(v)$の割合$\sigma_{st}(v)/\sigma_{st}$を全ノードペアについて
足し合わせたものである.
あるノードの媒介中心性が高いならば,そのノードには比較的多くの最短経路が通っていることとなり,
道路ネットワークや通信ネットワークにおいて重要なノードであると言える.

\section{関連研究}

媒介中心性の値を高速に計算することは実用上の観点から重要な課題である.
全ノードの媒介中心性の値をその定義に従って計算した場合,
その時間計算量はノード数を$|V|$とすると$\mathcal{O}(|V|^3)$であり,
ノードの数の増加に伴い必要な計算時間が膨大になる.
計算量を小さくするため,Brandesはノード数$|V|$とリンク数$|E|$として,
計算量が$\mathcal{O}(|V|^2\log|V|+|V||E|)$であるアルゴリズムを提案した
\cite{Brandes2001}.Brandesのアルゴリズムは第\ref{chap:preliminary}章で説明する
依存度と呼ばれる量を陰に求め,それを積算することで媒介中心性の値を高速に計算している.

Brandesのアルゴリズムが提案されて以来,そのアイデアを発展させたアルゴリズムが多数提案された.
例えば,前処理により等価で簡単なネットワークの媒介中心性を求める方法\cite{Puzis2012,Bentert2018}や,
分割統治法の考え方を用いた方法\cite{Erdos2015}がある.
また,処理を並列に行い高速化を図った方法\cite{Bader2006,Tan2009,Edmonds2010,Bernaschi2016}や,
媒介中心性の近似値を効率的に求める方法\cite{Brandes2007,Bader2007,Pfeffer2012,Yoshida2014}も
提案されている.
これらのアルゴリズムは,ノードやリンクの追加や削除が行われない,時不変ネットワークに対して高速に
媒介中心性を計算する.

一方,現実にある多くのネットワークはノードやリンクの追加や削除が起こる時変ネットワーク\cite{Holme2012}
である.時変ネットワークに対しては,Brandesのアルゴリズムで始めから計算するより,
変更に対する差分のみを計算する方がより効率的であると期待できる.
そのような考えから,時変ネットワークのノードの媒介中心性の値を効率的に更新する方法も提案されている.
Minimum union cycleと呼ばれる閉路の集合と媒介中心性を保持し,変更に対してそれらを更新する方法
\cite{Lee2012,Singh2015}や,
Hypergraph sketch\cite{Yoshida2014}を保持,更新することで媒介中心性の近似値を求める方法
\cite{Hayashi2015},
時不変ネットワークに対する近似法を応用した方法\cite{Bergamini2015a,Bergamini2015b}が提案された.

さらに,媒介中心性と共に最短経路を保持し,それらを更新する方法も提案されている.
Kasらは,RamalingamとRepsの最短経路更新法\cite{Ramalingam1996}を応用して,
辺挿入時の媒介中心性を更新する方法を提案した\cite{Kas2013}.
また,Nasreらは,Kargerらの\cite{Karger1993}の方法を時変ネットワークに適用して
辺挿入時の媒介中心性を更新する方法を提案し\cite{Nasre2014a}た.
また,Nasreらは,DemetrescuとItalianoの最短経路更新法\cite{Demetrescu2003}を
基にして辺削除時の媒介中心性を更新する方法を提案した\cite{Nasre2014b}.
また,PontecorviとRamachandranは,DemetrescuとItalianoの方法を応用して
頂点追加時の媒介中心性を更新する方法を提案した\cite{Pontecorvi2015}.
さらに,Bergaminiらは,RamalingamとRepsのアルゴリズムを基に,
辺挿入時の媒介中心性更新アルゴリズムを提案した\cite{Bergamini2017}.
最後に,現在までに提案された,媒介中心性を最短経路と共に更新するアルゴリズムの一覧を
表\ref{tab:comparison-of-algorithms}に示す.

\begin{table}[tb]
  \centering
  \caption{最短経路と共に媒介中心性を更新するアルゴリズム}
  \label{tab:comparison-of-algorithms}
  \begin{tabular}{ccc}
    \hline
    アルゴリズム & 最短経路更新アルゴリズム & 辺の操作 \\ \hline
    Kasら\cite{Kas2013} & Ramalingamら\cite{Ramalingam1996} & 挿入 \\ \hline
    Nasreら\cite{Nasre2014a} & Kargerら\cite{Karger1993} & 挿入 \\ \hline
    Nasreら\cite{Nasre2014b} & Demetrescuら\cite{Demetrescu2003} & 削除 \\ \hline
    Pontecorviら\cite{Pontecorvi2015} & Demetrescuら\cite{Demetrescu2003} & 挿入/削除 \\ \hline
    Bergaminiら\cite{Bergamini2017} & Ramalingamら\cite{Ramalingam1996} & 挿入 \\ \hline
    本研究 & Ramalingamら\cite{Ramalingam1996} & 削除 \\ \hline
  \end{tabular}
\end{table}

\section{研究目的}

現在まで多くの媒介中心性更新アルゴリズムが提案されてきたが,
RamalingamとRepsの方法に基づく辺削除時の媒介中心性更新アルゴリズムは知られていない.
そこで,本研究ではRamalingamとRepsの最短経路更新法に基づく辺削除時の媒介中心性更新法を提案する.
それと同時に本稿では,RamalingamとRepsの最短経路更新法に基づく辺挿入時の媒介中心性更新法を説明する.
また,それらの方法の有用性を理論解析および数値実験によって検証する.

本稿の以降の構成は次のとおりである.
続く第\ref{chap:preliminary}章で後の議論で必要なグラフの数学的表現や,最短経路,媒介中心性について説明をして,
第\ref{chap:algorithm}章でRamalingamとRepsの方法に基づく,
辺削除時の媒介中心性更新アルゴリズムを提案する.
さらに第\ref{chap:complexity-analysis}章で提案手法の時間計算量を解析し,
第\ref{chap:experiment}章で提案手法の性能を実験的に評価する.
最後に第\ref{chap:conclusion}章で結論を述べる.
