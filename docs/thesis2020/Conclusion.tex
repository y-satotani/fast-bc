\chapter{結論}
\label{chap:conclusion}
本研究は,RamalingamとRepsの最短経路更新アルゴリズムに基づく辺削除時の媒介中心性更新法の提案と評価を目的とした.
本研究で提案したアルゴリズムは,まず操作の影響を受ける分だけ媒介中心性の値を減算し,次にRamalingamとRepsに基づく方法で最短経路を更新した後,操作の影響を受けた分だけ媒介中心性の値を加算するものである.

提案法の評価は理論解析と実験によって行われた.
理論解析では,提案したアルゴリズムの時間計算量を導出し評価した.
具体的には,提案アルゴリズムを媒介中心性増減アルゴリズムと最短経路更新アルゴリズムの二つに分け,
それぞれのアルゴリズムの時間計算量を導出し,
既存法であるBrandesのアルゴリズムとDijkstraのアルゴリズムとそれぞれ比較した.
その結果,両アルゴリズムの時間計算量は各アルゴリズムによって走査される頂点の数が少ないとき,
既存アルゴリズムの時間計算量よりも小さいことを示した.

さらに実験では,提案アルゴリズムとBrandesのアルゴリズムの性能を比較した.
比較の結果,人工ネットワークと実ネットワーク両方で提案アルゴリズムの方がBrandesのアルゴリズムよりも媒介中心性の値を高速に更新することを示した.
特に,実ネットワークにおける性能比較の結果から,提案アルゴリズムはBrandesのアルゴリズムと比べて$6.87$倍の性能をもつことを示した.
さらに,提案アルゴリズムを応用して,媒介中心性の最大値を最小および最大にする辺操作の探索を行った.

今後の目標は,提案アルゴリズムを複数の辺操作に対応できるように拡張することと,
媒介中心性の最大値が最小であるようなグラフを遺伝アルゴリズムを基に探索するアルゴリズムの開発である.
%しろよ,絶対しろよ(熱い湯船の縁に掴まりながら)
