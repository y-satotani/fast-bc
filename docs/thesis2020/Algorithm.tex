\chapter{提案アルゴリズム}
\label{chap:algorithm}

\section{提案アルゴリズムの概要}

本章では,有向グラフ$G=(V,E)$に辺$(u,v)$が挿入,または,辺$(u,v)$が削除された後の各頂点からの最短経路および
各頂点の媒介中心性の値を更新するアルゴリズムを提案する.
挿入操作について$(u,v)\notin E$が,
削除操作については$(u,v)\in E$が成り立っていると仮定する.
記号について,挿入または削除の操作後であることを明示するため,記号$'$をつける.
例えば,操作後のグラフは$G'$と表す.
なお,辺$(u,v)$を挿入する操作は,無限大の$l_{uv}$を有限にする操作であると言えるから,
挿入操作に対する提案アルゴリズムを拡張して,ある辺の重みを減少させる操作に対応できる.
同様に,辺$(u,v)$を削除する操作は,有限の$l_{uv}$を無限大にする操作であると言えるから,
削除操作に対する提案アルゴリズムを拡張して,ある辺の重みを増加させる操作に対応できる.
さらに,無向辺$\{a,b\}$を有効辺$(a,b)$と$(b,a)$の組と見なすことにより,
提案アルゴリズムは無向グラフに対しても適用可能である.

頂点$t$に対して,辺操作によって$t$への最短経路が変化する頂点の集合を$S(t)$と表す.
同様に,頂点$s$に対して,辺操作によって$s$からの最短経路が変化する頂点の集合を$T(s)$と表す.
$S(t)$および$T(s)$に属する頂点は,それぞれ$t$および$s$について,辺更新の影響を受ける頂点集合と言える.
後述するが,$S(t)$および$T(s)$具体的な内容は,操作によって異なる.
さらに,頂点$s,x\in V$に対して,$T(s)$に含まれる頂点$t$についてペア依存度$\delta_{st}(x)$の総和を
$\Delta_{s\bullet}(x)$で表す\cite{Bergamini2017}.
つまり,$\Delta_{s\bullet}(x)=\sum_{t\in T(s)}\delta_{st}(x)$である.
同様に,$\Delta'_{s\bullet}(x)=\sum_{t\in T(s)}\delta'_{st}(x)$とする.

挿入と削除に共通する処理の流れは以下のとおりである.
まず,$T(u)$および$S(v)$を計算する.
各$s\in S(v)$に対して以下を実行する.
初めに$x\in V$に対する$\Delta_{s\bullet}(x)$を求めて,媒介中心性の値$B_x$から減算する.
つまり,$B''(x):=B(x)-\sum_{s\in S(v)}\Delta_{s\bullet}(x)$を計算する.
次に,各$t\in V$の$d_{st}$および$\sigma_{st}$を更新する.
さらに,$x\in V$に対する$\Delta'_{s\bullet}(x)$を求めて,
減算された媒介中心性の値$B''_s$に加算する.
つまり,$B'(x):=B''(x)+\sum_{s\in S(v)}\Delta'_{s\bullet}(x)$を計算する.
最後に,$t\in T(u)$に対して各$s\in V$の$d_{st}$および$\sigma_{st}$を更新する.

全体的なアルゴリズムをアルゴリズム\ref{alg:proposed-algorithm-full}に示す.
以降,媒介中心性の値を減少および増加させるアルゴリズムを説明した後,
挿入操作および削除操作それぞれに対する最短経路更新アルゴリズムを説明する.

\begin{algorithm}[tb]
  \caption{辺$(u,v)$の操作時に全頂点の媒介中心性の値$B_x$を更新するアルゴリズム}
  \label{alg:proposed-algorithm-full}
  \begin{algorithmic}[1]\small
    \Procedure{UpdateBetweenness}{}
    \ForAll{$s\in S(v)$}
    \State $\textproc{DecreaseBetweenness}(s)$
    \State $\textproc{UpdateSSSP}(s)$
    \State $\textproc{IncreaseBetweenness}(s)$
    \EndFor
    \ForAll{$t\in T(u)$}
    \State $\textproc{UpdateSTSP}(t)$
    \EndFor
    \EndProcedure
  \end{algorithmic}
\end{algorithm}

\section{媒介中心性の値を増加/減少させるアルゴリズム}
\label{sect:update-betweenness}

本節では,$\Delta_{s\bullet}(x)$および$\Delta_{s\bullet}'(x)$を計算する方法を説明する.
次の定理は,更新による依存度の変化量と$\Delta_{s\bullet}(x)$が一致することを示す.
\begin{theorem}
  任意の$s,x\in V$に対して,式\eqref{eq:change-amount-dependency}が成り立つ.
  \begin{equation}
    \delta'_{s\bullet}(x)-\delta_{s\bullet}(x)=\Delta'_{s\bullet}(x)-\Delta_{s\bullet}(x)
    \label{eq:change-amount-dependency}
  \end{equation}
\end{theorem}
\begin{proof}
  定理\ref{thm:condition-of-invariability-of-shortest-path}により,任意の$t\notin T(s)$に対して,
  $\sigma_{st}=\sigma'_{st}$である.さらに,$t\notin T(s)$の条件下で,任意の$x\in V$について
  $\sigma_{st}(x)=\sigma'_{st}(x)$が成り立つ.したがって,次が成り立つ.
  \begin{equation*}
    \begin{aligned}
      \delta'_{s\bullet}(x)-\delta_{s\bullet}(x)
      &=\sum_{t\neq x}\left(\frac{\sigma'_{st}(x)}{\sigma'_{st}}-\frac{\sigma_{st}(x)}{\sigma_{st}}\right) \\
      &=\sum_{t\in T(s)}\left(\frac{\sigma'_{st}(x)}{\sigma'_{st}}-\frac{\sigma_{st}(x)}{\sigma_{st}}\right)
      +\sum_{t\notin T(s)}\left(\frac{\sigma'_{st}(x)}{\sigma'_{st}}-\frac{\sigma_{st}(x)}{\sigma_{st}}\right) \\
      &=\sum_{t\in T(s)}\left(\frac{\sigma'_{st}(x)}{\sigma'_{st}}-\frac{\sigma_{st}(x)}{\sigma_{st}}\right) \\
      &=\Delta'_{s\bullet}(x)-\Delta_{s\bullet}(x)
    \end{aligned}
  \end{equation*}
\end{proof}

次の定理は,$\Delta_{s\bullet}$および$\Delta'_{s\bullet}$を効率的に計算する指針を与える.
\begin{theorem}[Bergamini et.al.\cite{Bergamini2017}]
  任意の$s\in S(v)$と$x\in V$について,次式が成り立つ.
  \begin{align}
    \Delta_{s\bullet}(x)
    &=\sum_{y\in\mathcal{S}_{G_s}(x)\cap T(s)}\frac{\sigma_{sx}}{\sigma_{sy}}(1+\Delta_{s\bullet}(y))
    +\sum_{y\in\mathcal{S}_{G_s}(x)\setminus T(s)}\frac{\sigma_{sx}}{\sigma_{sy}}\Delta_{s\bullet}(y)
    \label{eq:delta-dependency-1} \\
    \Delta'_{s\bullet}(x)
    &=\sum_{y\in\mathcal{S}_{G'_s}(x)\cap T(s)}\frac{\sigma'_{sx}}{\sigma'_{sy}}(1+\Delta'_{s\bullet}(y))
    +\sum_{y\in\mathcal{S}_{G'_s}(x)\setminus T(s)}\frac{\sigma'_{sx}}{\sigma'_{sy}}\Delta'_{s\bullet}(y)
    \label{eq:delta-dependency-2}
  \end{align}
\end{theorem}
\begin{proof}
  $t\in T(s)$とする.ただし,$t\neq x$とする.すると,$\sigma_{st}(x)/\sigma_{st}=\sum_{y\in\mathcal{S}_{G_s}(x)}\sigma_{st}(x,y)/\sigma_{st}$が成り立つ.したがって,式\ref{eq:delta-dependency-1}の左辺は,次のように書き換えられる.
  \begin{equation*}
    \begin{aligned}
      \Delta_{s\bullet}(x)=\sum_{t\in T(s)}\frac{\sigma_{st}(x)}{\sigma_{st}}
      &=\sum_{t\in T(s)}\sum_{y\in\mathcal{S}_{G_s}(x)}\frac{\sigma_{st}(x,y)}{\sigma_{st}}\\
      &=\sum_{y\in\mathcal{S}_{G_s}(x)}\sum_{t\in T(s)}\frac{\sigma_{st}(x,y)}{\sigma_{st}}
    \end{aligned}
  \end{equation*}
  ここで,$\sigma_{sy}$個の$s-y$間の最短経路の中に,$\sigma_{sx}$個は$x$を通る.
  したがって,$t\neq x$に対して,$s-t$間の最短経路の内,$\sigma_{sx}/\sigma_{sy}\cdot\sigma_{st}(y)$個は
  $x$と$y$を通る.つまり,$\sigma_{st}(x,y)=\sigma_{sx}/\sigma_{sy}\cdot\sigma_{st}(y)$である.

  一方,$t=x$に対して,$\sigma_{st}(x,y)$は単に$\sigma_{st}$である.
  したがって,次が成り立つ.
  \begin{equation*}
    \begin{aligned}
      &\sum_{y\in\mathcal{S}_{G_s}(x)\cap T(s)}\left(\frac{\sigma_{sx}}{\sigma_{sy}}+\sum_{t\in T(s)\setminus\{y\}}\frac{\sigma_{st}(y,x)}{\sigma_{st}}\right)
      +\sum_{y\in\mathcal{S}_{G_s}(x)\setminus T(s)}\sum_{t\in T(s)\setminus\{y\}}\frac{\sigma_{st}(y,x)}{\sigma_{st}} \\
      &\quad=\sum_{y\in\mathcal{S}_{G_s}(x)\cap T(s)}\frac{\sigma_{sx}}{\sigma_{sy}}\left(1+\sum_{t\in T(s)\setminus\{y\}}\frac{\sigma_{st}(y)}{\sigma_{st}}\right)
      +\sum_{y\in\mathcal{S}_{G_s}(x)\setminus T(s)}\frac{\sigma_{sx}}{\sigma_{sy}}\sum_{t\in T(s)\setminus\{y\}}\frac{\sigma_{st}(y)}{\sigma_{st}} \\
      &\quad=\sum_{y\in\mathcal{S}_{G_s}(x)\cap T(s)}\frac{\sigma_{sx}}{\sigma_{sy}}\left(1+\Delta_{s\bullet}(y)\right)
      +\sum_{y\in\mathcal{S}_{G_s}(x)\setminus T(s)}\frac{\sigma_{sx}}{\sigma_{sy}}\Delta_{s\bullet}(y) \\
    \end{aligned}
  \end{equation*}
  $\Delta'_{s\bullet}(x)$に対しても同様に証明できる.
\end{proof}

$s\in S(v),x\in V$について$\Delta_{s\bullet}(x)$を計算するには,$s\in S(v)$に対して
Brandesのアルゴリズムに基づく次の手順を行う.まず,$T(s)$の各要素$t$を,その優先度を
$d_{st}$として順位付きキュー\texttt{PQ}に挿入する.キューに要素が有る限り次を繰り返す.
\texttt{PQ}の優先度最大の要素$x$を取り出し,$\mathcal{P}(x)$の各要素$y$について,
$(y,x)$が$s$からの最短経路に含まれているならば,式\eqref{eq:delta-dependency-1}を用いて
$\Delta_{s\bullet}(y)$に加算する.さらに,$y$が\texttt{PQ}に含まれていなければ,
優先度を$d_{sy}$として$y$を\texttt{PQ}に追加する.
これらの操作を優先度付きキュー\texttt{PQ}が空になるまで繰り返す.
$\Delta'_{s\bullet}(x)$についても同様の手順で計算できる.

最後に,辺$(u,v)$の操作時に媒介中心性の値$B_x$を$\Delta_{s\bullet}(x)$だけ
減少させるアルゴリズムをアルゴリズム\ref{alg:decrease-betweenness}に示す.
アルゴリズム\ref{alg:proposed-algorithm-full}中の$\textproc{IncreaseBetweenness}$も
同様に構成される.

\begin{algorithm}[tb]
  \caption{辺$(u,v)$の操作時に媒介中心性の値$B_x$を$\Delta_{s\bullet}(x)$だけ減少させるアルゴリズム}
  \label{alg:decrease-betweenness}
  \begin{algorithmic}[1]\small
    \Procedure{DecreaseBetweenness}{$s$}
    \State $\Delta_{s\bullet}(x)\gets0,\:\forall x\in V$
    \State $\texttt{PQ}\gets\varnothing$
    \ForAll{$t\in T(s)$}
    \State $t$を優先度$d_{st}$として\texttt{PQ}に挿入する
    \EndFor
    \While{$|\texttt{PQ}|>0$}
    \State \texttt{PQ}から優先度最大の要素$x$を取り出す
    \State $B(x)\gets B(x)-\Delta_{s\bullet}(x)$
    \ForAll{$y\in\mathcal{P}(x)$}
    \If{$y=s\lor d_{sx}<d_{sy}+l_{yx}$}
    \State \textbf{continue}
    \EndIf
    \If{$x\in T(s)$}
    \State $\Delta_{s\bullet}(y)\gets\Delta_{s\bullet}(y)+\sigma_{sy}/\sigma_{sx}\cdot(1+\Delta_{s\bullet}(x))$
    \Else
    \State $\Delta_{s\bullet}(y)\gets\Delta_{s\bullet}(y)+\sigma_{sy}/\sigma_{sx}\cdot\Delta_{s\bullet}(x)$
    \EndIf
    \If{$y\notin\texttt{PQ}$}
    \State $y$を優先度$d_{sy}$として\texttt{PQ}に挿入する
    \EndIf
    \EndFor
    \EndWhile
    \EndProcedure
  \end{algorithmic}
\end{algorithm}

\section{最短経路の変化に関する条件}
本節では,グラフ$G=(V,E)$に辺$(u,v)$が挿入または削除されたあとの最短経路が変化する条件について述べる.
次の定理は,最短経路の変化が最短距離および最短経路数の変化につながることを示す.
\begin{theorem}
  \label{thm:condition-of-invariability-of-shortest-path}
  $E_{st}=E'_{st}$であることの必要十分条件は,$d_{st}=d'_{st}$かつ$\sigma_{st}=\sigma'_{st}$である.
\end{theorem}
\begin{proof}
  $E_{st}=E'_{st}$ならば$d_{st}=d'_{st}$かつ$\sigma_{st}=\sigma'_{st}$であることは自明である.

  逆は背理法を用いて証明する.
  つまり,図\ref{fig:general-shortest-paths-weighted}に示した$E_{st}$に対する操作のうち,
  $d_{st}=d'_{st}$かつ$\sigma_{st}=\sigma'_{st}$を満たすものは存在しないことを証明する.
  図において,同じ$L_i$に含まれる頂点$v$について$d_{sv}$および$d_{vt}$は同一とする.

  挿入時,長さが$L_i$間の距離より長い辺を挿入すると,$E_{st}=E'_{st}$を満たす.
  また,長さが$L_i$間の距離と等しい辺を挿入すると,$\sigma_{st}\neq\sigma'_{st}$となる.
  また,長さが$L_i$間の距離より短い辺を挿入すると$d_{st}\neq d'_{st}$となる.

  削除時,$e\notin E_{st}$である辺を削除すると,$E_{st}=E'_{st}$である.
  また,$e\in E_{st}$である辺を削除すると,$d_{st}\neq d'_{st}$または$\sigma_{st}\neq\sigma'_{st}$となる.

  したがって$d_{st}=d'_{st}$かつ$\sigma_{st}=\sigma'_{st}$を満たすように,
  $E_{st}\neq E'_{st}$とすることは不可能である.
  ゆえに$d_{st}=d'_{st}$かつ$\sigma_{st}=\sigma'_{st}$ならば$E_{st}=E'_{st}$である.
\end{proof}
\begin{figure}
  \centering
  \def\svgwidth{.5\columnwidth}
  \input{general-shortest-paths-weighted.pdf_tex}
  \caption{$s$と$t$の一般的な最短経路.同じ$L_i$に含まれる頂点$v$について$d_{sv}$および$d_{vt}$は同一とする.}
  \label{fig:general-shortest-paths-weighted}
\end{figure}

以下では,一辺挿入と一辺削除それぞれの操作に対する最短経路更新法を説明する.

\section{一辺挿入時の最短経路更新法}
\label{sect:update-apsp-on-insert}
本節では,グラフに一つの辺が挿入されたときの最短経路を更新する方法を説明する.
$G=(V,E)$に辺$e=(v,w) \not\in E$を挿入して得られるグラフを$G'=(V,E')$とする.
まず,ある頂点ペア$s,t$の間の最短経路が削除によって変化するための条件を示す.

\begin{lemma}
  \label{lmm:update-augdist-on-insert}
  $s,t\in V$が$E'_{st}\neq E_{st}$を満たすための必要十分条件は,次の式が成り立つことである.
  \begin{equation}
    d_{st}\geq d_{su}+l'_{uv}+d_{vt}
    \label{eq:update-augdist-on-insert}
  \end{equation}
\end{lemma}
\begin{proof}
  $E_{st}'\neq E_{st}$が成り立つならば,定理
  \ref{thm:condition-of-invariability-of-shortest-path}より,
  $d'_{st}\neq d_{st}$または$\sigma'_{st}\neq\sigma_{st}$が成り立つ,
  このとき,新たに挿入された辺は$(u,v)$の一つで,これが新たな最短経路に含まれるので,
  $d_{st}\geq d'_{st}=d_{su}+l'_{uv}+d_{vt}$が成り立つ.

  逆に,式\eqref{eq:update-augdist-on-insert}が成り立つならば,
  辺$(u,v)\notin E_{st}$が新たな最短経路に含まれることになるので,
  $(u,v)\in E'_{st}$が成り立つ.したがって,$E'_{st}\neq E_{st}$である.
\end{proof}

次の定理は,挿入時の$T(x)$および$S(x)$の具体的な内容を示している.
\begin{theorem}
  \label{thm:affected-vertices-on-insert}
  挿入時の$T(x)$および$S(x)$はそれぞれ式\eqref{eq:affected-targets-on-insert}と
  \eqref{eq:affected-sources-on-insert}で表される.
  \begin{align}
    T(x)&=\{t|t\in V\land d_{xt}\geq d_{xu}+l'_{uv}+d_{vt}\}
    \label{eq:affected-targets-on-insert} \\
    S(x)&=\{s|s\in V\land d_{sx}\geq d_{su}+l'_{uv}+d_{vx}\}
    \label{eq:affected-sources-on-insert}
  \end{align}
\end{theorem}

以下,始点$s$が与えられたときにすべての$t$に対して$d'_{st}$および
$\sigma'_{st}$を計算する方法を説明する.
この方法は,RamalingamとRepsの方法に倣って優先度付きキューを用いることによって探索を行う.
まず,優先度付きキューに$v$を挿入する.このとき,優先度は$d_{su}+l'_{uv}$とする.
キューに要素が存在する限り,以下を繰り返す.
キューから優先度最小の要素を取り出し,これを$x$とする.このとき,
優先度を$\hat{d}_{sx}$とする.
その後,次式にしたがって$d'_{sx}$および$\sigma'_{sx}$を計算する.
\begin{equation*}
  \begin{aligned}
    d'_{sx}&=\hat{d}_{sx} \\
    \sigma'_{sx}&=\sum_{y\in\mathcal{P}(x)\land d'_{sx}=d'_{sy}+l'_{yx}}\sigma'_{sy}
  \end{aligned}
\end{equation*}
さらに,$y\in\mathcal{S}(x)$のうち,式\eqref{eq:update-augdist-on-insert}を満たす,
つまり,$d_{sy}\geq d'_{sx}+l_{xy}$を満たす$y$を優先度付きキューに挿入する.
このとき,その優先度を$d'_{sx}+l_{xy}$とする.
以上の手順をキューに要素が存在する限り繰り返す.

一辺挿入時の最短経路更新の具体的な手続きをアルゴリズム\ref{alg:incremental-algorithm}に示す.
なお,アルゴリズム\ref{alg:proposed-algorithm-full}における$\textproc{UpdateSTSP}$は
$s$と$t$,$S$と$T$,$\mathcal{S}$と$\mathcal{P}$をそれぞれ入れ替えることによって得られる.

\begin{algorithm}[tb]
  \caption{辺$(u,v)$の挿入時の$d_{st}$と$\sigma_{st}$を更新するアルゴリズム}
  \label{alg:incremental-algorithm}
  \begin{algorithmic}[1]\small
    \Procedure{UpdateSSSP}{$s$}
    \If{$d_{sv}<d_{su}+l_{uv}$}
    \State \textbf{return}
    \EndIf
    \State $\texttt{PQ}\gets\varnothing$
    \State 優先度を$d_{su}+l_{uv}$として$v$を$\texttt{PQ}$に挿入
    \While{$|\texttt{PQ}|>0$}
    \State $\texttt{PQ}$から優先度最小の要素を取り出し$x$とする
    \State このときの優先度を$\hat{d}_{sx}$とする
    \State $d_{sx}\gets\hat{d}_{sx}$
    \State $\sigma_{sx}\gets\sum_{y|y\in\mathcal{P}(x)\land d_{sx}=d_{sy}+l_{yx}}\sigma_{sy}$
    \ForAll{$y\in\mathcal{S}(x)$}
    \If{$d_{sy}\geq d_{sx}+l_{xy}$}
    \State \texttt{PQ}の$y$の優先度を$d_{sx}+l_{xy}$に設定
    \EndIf
    \EndFor
    \EndWhile
    \EndProcedure
  \end{algorithmic}
\end{algorithm}

\section{一辺削除時の最短経路更新法}
\label{sect:update-apsp-on-delete}
本節では,グラフから一つの辺が削除されたときの最短経路を更新する方法を説明する.
$G=(V,E)$に辺$(u,v)\in E$を削除して得られるグラフを$G'=(V,E')$とする.
まず,ある頂点ペア$s,t$の間の最短経路が削除によって変化するための条件を示す.

\begin{lemma}
  \label{lmm:update-augdist-on-delete}
  $s,t\in V$に対して,$E_{st}'\neq E_{st}$であるための必要十分条件は
  $s,t$が式\eqref{eq:update-augdist-on-delete}を満たすことである.
  \begin{equation}
    d_{su}+l_{uv}+d_{vt}=d_{st}
    \label{eq:update-augdist-on-delete}
  \end{equation}
\end{lemma}
\begin{proof}
  $E_{st}'\neq E_{st}$が成り立つならば,定理
  \ref{thm:condition-of-invariability-of-shortest-path}より,
  $d'_{st}\neq d_{st}$または$\sigma'_{st}\neq\sigma_{st}$が成り立つ,
  このとき,削除された辺は$(u,v)$の一つで,これが削除前の最短経路に含まれていたので,
  $d_{st}=d_{su}+l_{uv}+d_{vt}$が成り立つ.

  逆に,式\eqref{eq:update-augdist-on-delete}が成り立つならば,
  辺$(u,v)\in E_{st}$が成り立つので,削除後の最短経路は
  $(u,v)\notin E'_{st}$を満たす.したがって,$E'_{st}\neq E_{st}$である.
\end{proof}

次の定理は,削除時の$T(x)$および$S(x)$の具体的な内容を示している.
\begin{theorem}
  \label{thm:affected-vertices-on-delete}
  削除時の$T(x)$および$S(x)$はそれぞれ式\eqref{eq:affected-targets-on-delete}と
  \eqref{eq:affected-sources-on-delete}で表される.
  \begin{align}
    T(x)&=\{t|t\in V\land d_{xt}=d_{xu}+l_{uv}+d_{vt}\}
    \label{eq:affected-targets-on-delete} \\
    S(x)&=\{s|s\in V\land d_{sx}=d_{su}+l_{uv}+d_{vx}\}
    \label{eq:affected-sources-on-delete}
  \end{align}
\end{theorem}

以後,始点$s$が与えられたときにすべての$t$に対して$d'_{st}$および
$\sigma'_{st}$を計算する方法を説明する.
挿入時と同様に,$\sigma'_{st}$は$d'_{st}$の昇順に計算する必要がある.
しかし,挿入時と違い,$d'_{st}$が最小となる$t$を直ちに求めることはできない.
正しい順序で$d'_{st}$および$\sigma'_{st}$を計算するため,
次に示すような,RamalingamとRepsの方法に基づく手順で計算を行う.

まず,各$x\in T(s)$について,削除の影響を受けない頂点を先行にもつ頂点,つまり,
$\exists y\in\mathcal{P}(x),\,y\notin T(s)$を満たす頂点を優先度付きキューに
挿入する.このとき,頂点$x$の優先度を
$\min\{d_{sx}+l_{xy}|y\in\mathcal{P}(x)\land y\notin T(s)\}$とする.
キューに要素が存在する限り,以下を繰り返す.
キューから優先度最小の要素を取り出し,これを$x$とする.このとき,
優先度を$\hat{d}_{sx}$とする.
その後,次式にしたがって$d'_{sx}$および$\sigma'_{sx}$を計算する.
\begin{equation*}
  \begin{aligned}
    d'_{sx}&=\hat{d}_{sx} \\
    \sigma'_{sx}&=\sum_{y\in\mathcal{P}(x)\land d'_{sx}=d'_{sy}+l'_{yx}}\sigma'_{sy}
  \end{aligned}
\end{equation*}
さらに,$y\in\mathcal{S}(x)$のうち,式\eqref{eq:update-augdist-on-delete}を満たす,
つまり,$d_{sy}=d_{sx}+l_{xy}$を満たす$y$を優先度付きキューに挿入する.
このとき,その優先度を$d'_{sx}+l_{xy}$とする.
以上の手順をキューに要素が存在する限り繰り返す.

提案手法の具体的な手続きをアルゴリズム\ref{alg:decremental-algorithm}に示す.
なお,アルゴリズム\ref{alg:proposed-algorithm-full}における$\textproc{UpdateSTSP}$は
$s$と$t$,$S$と$T$,$\mathcal{S}$と$\mathcal{P}$をそれぞれ入れ替えることによって得られる.

\begin{algorithm}[tb]
  \caption{辺$(u,v)$の削除時の$d_{st}$と$\sigma_{st}$を更新するアルゴリズム}
  \label{alg:decremental-algorithm}
  \begin{algorithmic}[1]\small
    \Procedure{UpdateSSSP}{$s$}
    \If{$d_{sv}<d_{su}+l_{uv}$}
    \State \textbf{return}
    \EndIf
    \State $\texttt{PQ}\gets\varnothing$
    \ForAll{$x\in T(s)$}
    \If{$\exists y\in\mathcal{P}(x),\,y\notin T(s)$}
    \State $\hat{d}\gets\min\{d_{sx}+l_{xy}|y\in\mathcal{P}(x)\land y\notin T(s)\}$
    \State 優先度を$\hat{d}$として$x$を\texttt{PQ}に挿入
    \EndIf
    \EndFor
    \While{$|\texttt{PQ}|>0$}
    \State $\texttt{PQ}$から優先度最小の要素を取り出し$x$とする
    \State このときの優先度を$\hat{d}_{sx}$とする
    \State $d_{sx}\gets\hat{d}_{sx}$
    \State $\sigma_{sx}\gets\sum_{y|y\in\mathcal{P}(x)\land d_{sx}=d_{sy}+l_{xy}}\sigma_{sy}$
    \ForAll{$y\in\mathcal{S}(x)$}
    \If{$y\in T(s)$}
    \State \texttt{PQ}の$y$の優先度を$d_{sx}+l_{xy}$に設定
    \EndIf
    \EndFor
    \EndWhile
    \EndProcedure
  \end{algorithmic}
\end{algorithm}
