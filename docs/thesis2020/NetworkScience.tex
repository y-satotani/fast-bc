\chapter{ネットワーク科学}
\label{chap:network-science}

本章では,本研究に関連した分野であるネットワーク科学の概要と応用例について説明する.
その中でも,ネットワーク科学の応用例とネットワークノードの重要度を測る中心性について詳しく説明する.

ネットワーク科学とは,二つの主体間の関係を表すデータの分析およびモデル化に関する研究分野である.
ここで言う主体とは,対象とする関係によって異なる.例えば,友人関係ならば,主体は個人を表し,食物連鎖の関係ならば,主体は動物の種を表す.
二つの主体間の関係を表すデータであれば主体の種類は問わない.
そのため,ネットワーク科学はあらゆる主体を対象にとることが可能という意味で普遍性を有する.
その考え方は,主にデータ構造の面で近いグラフ理論から引き継がれている.
その他,化学においては,分子同士の結合を記述するためにネットワーク構造が用いられたり\cite{Sylvester1878},
社会科学においては,人間関係をネットワーク構造として表したものをソシオグラムと呼んだり\cite{Moreno1978}
と複数の研究分野と相互に影響を与えている.

近年,高性能な計算機が普及したことによって,現実にある膨大なネットワークを分析することが可能になった.
例えば,Twitterのフォローネットワーク\cite{Kwak2010}や,
StackExchangeでの回答ネットワーク\cite{Movshovitz-Attias2013},
GitHubのコラボレーションネットワーク\cite{Lima2014}が挙げられる.
このような計算機の進化により,人のミクロな交流をマクロな視点から考察できるようになった.

その一方,現実のネットワークの特徴を上手く表現するようなモデルが開発されてきた.
古くからあるモデルは1959年にErd{\"{o}}sとR{\'{e}}nyiが開発したランダムグラフ\cite{Erdos1959}で,
完全グラフからランダムに辺を取り出す操作を繰り返すことで得られる.
近年では,ネットワークに見られる六次の隔たり\cite{Travers1969}の特徴を表した
WattsとStrogatzのスモールワールドネットワーク\cite{Watts1998}や,
次数分布がベキ分布であるスケールフリー性を再現した,Barab{\'{a}}siとAlbertによる
スケールフリーネットワーク\cite{Barabasi1999}が開発された.
それに加えて,既存のネットワークに新たな概念を加えたネットワークモデルも考案されている.
主なモデルは,主体の関係が時々刻々と変化するテンポラルネットワーク(時変ネットワーク)\cite{Holme2012}や,
職場とサークルなど,同じ主体の組み合わせで関係が複数あるものの議論を可能にする多層ネットワーク\cite{Kivela2014}がある.

\section{ネットワーク科学の応用例}

以後,ネットワーク科学のいくつかの応用を示す.まず,ネットワーク科学の応用として注目され続ける
感染症の伝搬について説明する.その後,本研究の応用として考えられる交通網と社会ネットワークの分析
について説明する.
本章で説明していないネットワーク科学と社会との関係や,
その他の応用は,Barab{\'{a}}siの著書\cite{Barabasi2016}に詳しく記されている.

\subsection{感染症の伝搬}

インフルエンザやエボラ出血熱など,感染症の中には我々の脅威となるものがある.
古くは14世紀にヨーロッパで流行したペスト,18世紀から19世紀にかけて世界的に流行したコレラ,
20世紀初頭に流行したスペイン風邪がある.
そのような感染症の対策として,その病理の解明と同時に感染の規模を予測することも重要である.

20世紀以前の感染症拡散のモデルはひとつの空間に基づくもので,これは同じ社会物理的な空間にいるならば,
感染する可能性があるとするものだった.つまり,駅構内や病院などの一定の広さの空間のみを対象としていた.

その一方,Hufnagelらのモデル\cite{Hufnagel2004}は,従来の局所的な感染に加え,
感染者が飛行機で移動し,その感染者から感染することも考慮されている.
この感染症予測は,インフルエンザあるいはエボラ出血熱の流行の予測に用いられる.
実際,そのモデルは2009年に発生したH1N1型インフルエンザウイルスによるパンデミックの
規模を正確に予測した\cite{Balcan2009}.

また,この手法を用いると,生物を媒介とするウイルスの他に,コンピュータウイルスや社会的なウイルスの
ミームの拡散を予測することも可能である.
例えば,Wangらは2010年に携帯電話を媒介するコンピュータウイルスが拡散する条件を導出した\cite{Wang2009}.

\subsection{交通網の分析}

感染症の予測では空の交通網が活用されていたが,陸の交通網である道路もネットワーク科学で分析できる.
特に,頑健な社会インフラの構築のために,交通網の脆弱な箇所を見つけて保全することは重要である.

2007年にTaylorらは,高速道路をネットワークとみなすことで,交通網の脆弱な箇所を特定した\cite{Taylor2007}.
Taylorらによる道路ネットワークの頑健性の概念には,信頼性と脆弱性がある.
信頼性とはある時間の水準で車が到達できるかを表す.道路の通行のしやすさおよび移動できる確率に依存する.
一方,脆弱性とは道路が不全になった時に引き起こされる社会的,経済的な影響の大きさを表す.
Taylorらは,これらの概念に対する簡単な指標を定義して,実際のオーストラリアの主要道路を分析した.

2010年,Tizghadamらは,交通網のネットワークの頑健性を最適化するために,
道路の許容量を調整する進化計算のアルゴリズムを開発した\cite{Tizghadam2010}.
Tizghadamらのネットワークの頑健性は後に説明する媒介中心性の派生に基づいている.
その指標はランダムウォークの考え方に基づき,その指標が高い地点はより頻繁に通過される地点であることを表す.

\subsection{社会ネットワークの分析}

社会ネットワークとは,もともとは学校や職場といった特定の集団に属するネットワークを指す.
この考え方を使って,学校や職場など,特定の集団の中にある個人の心理学について議論された\cite{Moreno1978}.
しかし,近年では,ソーシャルネットワーキングサービスの台頭により,従来とは比較にならない程大きなネットワークが取得可能となった.
それにより,特定の集団に限定されない,より巨視的な議論が可能となった.

その一つに,巨大なネットワークを特定の集団に分けるコミュニティ検出\cite{Fortunato2010}が挙げられる.
古典的には,最小カット問題を解く要領でネットワークを分割する方法や,
類似するノードを同じコミュニティにまとめることを繰り返す方法がある.
後者は事前にコミュニティの数を与える必要がないことが利点である.
また,隣接行列をふたつの行列の積を用いて近似することを応用した方法\cite{Wang2011}もある.

また,オンラインゲームの利用者同士の協力行動の理解も試みられている.
Szellらは,多層ネットワークの考え方を使って,オンラインゲームの利用者間の交流を分析した\cite{Szell2010}.
その結果,攻撃などの負の関係で構築されたネットワークはベキ次数分布を示すことが分かった.
特に,敵対関係のネットワークは優先的選択\cite{Barabasi1999}で構築される.
さらに,友好などの正の関係のネットワークのクラスタリング係数は,負の関係のネットワークのものより高いことも分かった.
クラスタリング係数が高いネットワークは,その凝集性から協力関係が広がりやすいことが知られている.

ネットワークを分析するに当たって,ネットワーク内の重要なノードを発見することは有意義である.
次節では,重要なノードの発見や,ノードの重要性の計測に役立つ中心性について述べる.

\section{中心性}
\label{sect:centrality}

組織内の重要な人物を発見することは古くからの組織論の課題である\cite{Christie1952,Ibarra1993}.
ネットワークのノードの重要度は中心性と呼ばれ,用途に沿っていくつかの中心性が開発されてきた.
その中で最も単純なものは各ノードの次数(隣接しているノードの数)をそのまま使う次数中心性である.
しかし,次数が高くても真に重要であるとは限らない.
例えば,フォロワー数が数万だとしても,そのほとんどが機械的に
運用されているアカウントであれば,真に影響力を持っているとは言い難い.
この節では以下,現在まで提案された主な中心性について説明する.
グラフ理論の記号および意味に関しては,第\ref{chap:preliminary}章を参照されたい.

\subsection{近接中心性}
\label{subsect:closeness}

近接中心性\cite{Bavelas1948,Beauchamp1965}とは,
ネットワーク内の他のノードとの距離の総和の逆数として定義される.つまり,
\begin{equation*}
  C_v=\cfrac{1}{\sum_wd_{v,w}}.
\end{equation*}

近接中心性が高いノードは他のノードに近い位置にあり,ここから情報を伝達すると
情報を速く伝達できると考えられる.
しかし,この定義の通信モデルは最短経路を通ることを仮定しているので,
別の通信モデルでは現実的でないことがある.
これを考慮して,ランダムウォークに基づく近接中心性も提案されている\cite{White2003}.

\subsection{媒介中心性}
\label{subsect:betweenness}

媒介中心性\cite{Anthonisse1971,Freeman1977}とは,
頂点ペア$(s,t)$の最短経路数に占める,自身を通る最短経路数の割合の総和として
定義される.つまり,
\begin{equation*}
  B_v=\sum_{s\neq v}\sum_{t\neq {v,s}}\frac{\sigma_{st}(v)}{\sigma_{st}}.
\end{equation*}

媒介中心性の大きなノードは多くの最短経路上にあり,通信に欠かせないノードであると言える.
近接中心性と同じく,この定義の通信モデルは最短経路を通ることを仮定しているので,
別の通信モデルでは現実的でないことがある.
これを考慮して,ランダムウォークに基づく媒介中心性も提案されている\cite{Newman2005}.

\subsection{固有ベクトル中心性}
\label{subsect:eigenvector}

固有ベクトル中心性(Eigenvector centrality)\cite{Bonacich1991}とは,
隣接する頂点の固有ベクトル中心性の総和として表現される.
つまり,$\lambda$を定数として,
\begin{equation*}
  x_v=\frac{1}{\lambda}\sum_{w\in\mathcal{N}_G(v)}x_w.
\end{equation*}
また,$\mathbf{A}$をグラフの隣接行列とすると,固有ベクトル方程式として表すことができる.
\begin{equation*}
  \mathbf{A}\mathbf{x}=\lambda\mathbf{x}.
\end{equation*}
一般に$\mathbf{A}$の固有値は複数存在するが,$\lambda$は$\mathbf{A}$の最大の固有値とする.

固有ベクトル中心性は,隣接する頂点の固有ベクトル中心性が高いほど自身の中心性も高いので,
Webページや学術論文誌のスコアリングに応用される.
