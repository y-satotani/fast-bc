\chapter{計算量の解析}
\label{chap:complexity-analysis}

本章では,第\ref{chap:algorithm}章で提案したアルゴリズムの計算量を与える.
まず,以下の議論のためにいくつかの記号を導入しておく.
$\tau(s)$および$\tau'(s)$は,それぞれ$T(s)$と$\Delta_{s\bullet}(x)>0$
および$\Delta'_{s\bullet}(x)>0$を満たす頂点の和集合である.つまり,
$\tau(s)=T(s)\cup\{x|\Delta_{s\bullet}(x)>0\}$,
$\tau'(s)=T(s)\cup\{x|\Delta'_{s\bullet}(x)>0\}$とする.
また,頂点集合$V$について$\left\vert V\right\vert$を$V$の要素数,
$\left\Vert V\right\Vert$を$V$の各頂点と接続する辺の数とする.

\section{媒介中心性を減少/増加させるアルゴリズムの計算量}

本節では,媒介中心性を減少および増加させるアルゴリズムの計算量を与える.
次の定理は,最短経路の更新の前後に行われる媒介中心性の更新の計算量を示す.
まず,古い最短経路に関する依存度の分だけ媒介中心性を減算するアルゴリズムの計算量を示す.

\begin{theorem}[Bergamini et.al.\cite{Bergamini2017}]
  \label{thm:decrease-betweenness-weight-complexity}
  アルゴリズム\ref{alg:decrease-betweenness}の\textproc{DecreaseBetweenness}の,
  重み付きグラフと重みなしグラフに対する計算量は,それぞれ次式で与えられる.
  \begin{align}
    &\mathcal{O}\left(\left\|\tau(s)\right\|+\left|\tau(s)\right|\log\left|\tau(s)\right|\right)
    \label{eq:decrease-betweenness-weighted-complexity} \\
    &\mathcal{O}\left(\left\|\tau(s)\right\|\right)
    \label{eq:decrease-betweenness-unweighted-complexity}
  \end{align}
\end{theorem}
\begin{proof}
  重み付きグラフでは二分ヒープを,重みなしグラフではバケツヒープを優先度付きキューとして
  用いると仮定する.
  すると,8行目の要素取り出しの操作はそれぞれ対数時間と定数時間で実行できる.
  10行目から22行目で取り出された要素の近傍を走査する.
  ここで,優先度付きキューから取り出された要素の数が$|\tau(s)|$と一致することを示す.
  4行目で\texttt{PQ}に$|T(s)|$個の要素をプッシュする.
  また,20行目で\texttt{PQ}に追加される要素は,$s$と$T(s)$のひとつの頂点との間の最短経路の上にある.
  これは,少なくとも一つの$t\in T(s)$が,$\sigma_{st}(y)>0$であることを示し,
  $\Delta_{s\bullet}(y)>0$が成り立つ.したがって,\texttt{PQ}に追加される要素の個数は
  $T(s)\cup\{x|\Delta_{s\bullet}(x)>0\}=\tau(s)$である.
  したがって,アルゴリズム\ref{alg:decrease-betweenness}の\textproc{DecreaseBetweenness}の,
  計算量は,重みグラフに対して$\mathcal{O}(\|\tau(s)\|+|\tau(s)|\log|\tau(s)|)$
  で,重みなしグラフに対して$\mathcal{O}(\|\tau(s)\|)$である.
\end{proof}

定理\ref{thm:decrease-betweenness-weight-complexity}より,全ての$s$に対して
媒介中心性$B_x$を減少させるアルゴリズムの計算量は,
式\eqref{eq:decrease-betweenness-weighted-complexity}と
式\eqref{eq:decrease-betweenness-unweighted-complexity}の$s\in S(t)$に関する
総和を求めることで,それぞれ
式\eqref{eq:full-decremental-betweenness-weighted-complexity}と
式\eqref{eq:full-decremental-betweenness-unweighted-complexity}で与えられる.
\begin{align}
  &\mathcal{O}\left(\sum_{s\in S(v)}\|\tau(s)\|+|\tau(s)|\log|\tau(s)|\right),
  \label{eq:full-decremental-betweenness-weighted-complexity} \\
  &\mathcal{O}\left(\sum_{s\in S(v)}\|\tau(s)\|\right).
  \label{eq:full-decremental-betweenness-unweighted-complexity}
\end{align}

次に,最短経路更新の後に行う,新たな最短経路に関する依存度の値だけ媒介中心性を
増加させるアルゴリズムの計算量を示す.

\begin{theorem}[Bergamini et.al.\cite{Bergamini2017}]
  \label{thm:increase-betweenness-weight-complexity}
  アルゴリズム\ref{alg:decrease-betweenness}の\textproc{IncreaseBetweenness}の
  重み付きグラフと重みなしグラフに対する計算量は,それぞれ次式で与えられる.
  \begin{align}
    &\mathcal{O}\left(\|\tau'(s)\|+|\tau'(s)|\log|\tau'(s)|\right)
    \label{eq:increase-betweenness-weighted-complexity} \\
    &\mathcal{O}\left(\|\tau'(s)\|\right)
    \label{eq:increase-betweenness-unweighted-complexity}
  \end{align}
\end{theorem}
\begin{proof}
  定理\ref{thm:decrease-betweenness-weight-complexity}と同じ要領で示すことができる.
  重み付きグラフでは二分ヒープを,重みなしグラフではバケツヒープを優先度付きキューとして
  用いると仮定する.
  すると,8行目の要素取り出しの操作はそれぞれ対数時間と定数時間で実行できる.
  10行目から22行目で取り出された要素の近傍を走査する.
  ここで,優先度付きキューから取り出された要素の数が$|\tau'(s)|$と一致することを示す.
  4行目で\texttt{PQ}に$|T(s)|$個の要素をプッシュする.
  また,20行目で\texttt{PQ}に追加される要素は,$s$と$T(s)$のひとつの頂点との間の最短経路の上にある.
  これは,少なくとも一つの$t\in T(s)$が,$\sigma'_{st}(y)>0$であることを示し,
  $\Delta'_{s\bullet}(y)>0$が成り立つ.したがって,\texttt{PQ}に追加される要素は
  $T(s)\cup\{x|\Delta'_{s\bullet}(x)>0\}=\tau'(s)$である.
  したがって,アルゴリズム\ref{alg:decrease-betweenness}の
  \textproc{IncreaseBetweenness}の計算量は,
  重みグラフに対して$\mathcal{O}(\|\tau'(s)\|+|\tau'(s)|\log|\tau'(s)|)$で,
  重みなしグラフに対して$\mathcal{O}(\|\tau'(s)\|)$である.
\end{proof}

定理\ref{thm:increase-betweenness-weight-complexity}より,全ての$s$に対して
媒介中心性$B_x$を増加させるアルゴリズムの計算量は,
式\eqref{eq:increase-betweenness-weighted-complexity}と
式\eqref{eq:increase-betweenness-unweighted-complexity}の$s\in S(t)$に関する
総和を求めることで,それぞれ
式\eqref{eq:full-incremental-betweenness-weighted-complexity}と
式\eqref{eq:full-incremental-betweenness-unweighted-complexity}で与えられる.
\begin{align}
  &\mathcal{O}\left(\sum_{s\in S(v)}\|\tau'(s)\|+|\tau'(s)|\log|\tau'(s)|\right),
  \label{eq:full-incremental-betweenness-weighted-complexity} \\
  &\mathcal{O}\left(\sum_{s\in S(v)}\|\tau'(s)\|\right).
  \label{eq:full-incremental-betweenness-unweighted-complexity}
\end{align}

\section{辺挿入時の最短経路更新アルゴリズムの計算量}
次の定理は,辺挿入時の最短経路変更アルゴリズムの計算量を示す.

\begin{theorem}
  \label{thm:incremental-shortest-path-update}
  アルゴリズム\ref{alg:incremental-algorithm}の\textproc{UpdateShortestPath}の
  重み付きグラフと重みなしグラフに対する計算量はそれぞれ次式で与えられる.
  \begin{align}
    &\mathcal{O}\left(\|T(s)\|\log|T(s)|)+\sum_{t\in T(u)}(\|S(t)\|\log|S(t)|\right)
    \label{eq:incremental-shortest-path-update-weighted} \\
    &\mathcal{O}\left(\|T(s)\|+\sum_{t\in T(u)}\|S(t)\|\right)
    \label{eq:incremental-shortest-path-update-unweighted}
  \end{align}
\end{theorem}
\begin{proof}
  式\eqref{eq:incremental-shortest-path-update-weighted}のみについて示す.
  式\eqref{eq:incremental-shortest-path-update-unweighted}については,
  式\eqref{eq:incremental-shortest-path-update-weighted}と同じ要領で示すことができる.

  アルゴリズムは,11行目の$\sigma_{sy}$の計算および12行目から16行目のループで
  優先度つきキューから取り出された要素の近傍を走査している.
  さらに,14行で優先度付きキューに要素を追加している.
  そして13行目の条件から,このループは$s\in S(t)$に対して繰り返される.
  以上から,アルゴリズム\ref{alg:incremental-algorithm}の\textproc{UpdateSSSP}に,
  ある$s$が与えられたときの計算量は,
  \begin{equation*}
    \begin{aligned}
      \mathcal{O}(\sum_{s\in S(t)}(|(\mathcal{S}(s)|+|\mathcal{P}(s)|\log|S(t)|))
      &=\mathcal{O}(\|S(t)\|+\|S(t)\|\log|S(t)|) \\
      &=\mathcal{O}(\|S(t)\|\log|S(t)|)
    \end{aligned}
  \end{equation*}
\end{proof}

同様に,辺挿入時の\textproc{UpdateSTSP}の計算量は,$S$を$T$にすれば良いので,
\begin{equation*}
  \begin{aligned}
    &\mathcal{O}(\|S(v)\|+\|T(u)\|+\sum_{s\in S(v)}(\|T(s)\|\log|T(s)|)+\sum_{t\in T(u)}(\|S(t)\|\log|S(t)|))\\
    &\:=\mathcal{O}(\sum_{s\in S(v)}(\|T(s)\|\log|T(s)|)+\sum_{t\in T(u)}(\|S(t)\|\log|S(t)|)).
  \end{aligned}
\end{equation*}
である.

さらに,定理\ref{thm:incremental-shortest-path-update}より,辺挿入時の全頂点間の最短経路を更新する
処理の計算量は,$S(v)$と$T(u)$を求め,$s\in S(v)$に対して\textproc{UpdateSSSP}を,
$t\in T(u)$に対して\textproc{UpdateSTSP}を実行するため,
\begin{align}
  &\mathcal{O}(\|S(v)\|+\|T(u)\|+\sum_{s\in S(v)}(\|T(s)\|\log|T(s)|)+\sum_{t\in T(u)}(\|S(t)\|\log|S(t)|)), \label{eq:full-incremental-shortest-path-update-weighted} \\
  &\:=\mathcal{O}(\sum_{s\in S(v)}(\|T(s)\|\log|T(s)|)+\sum_{t\in T(u)}(\|S(t)\|\log|S(t)|)). \label{eq:full-incremental-shortest-path-update-unweighted}
\end{align}
である.

\section{辺削除時の最短経路更新アルゴリズムの計算量}
次の定理は,辺削除時の最短経路変更アルゴリズムの計算量を示す.

\begin{theorem}
  \label{thm:decremental-shortest-path-update}
  アルゴリズム\ref{alg:decremental-algorithm}の\textproc{UpdateSSSP}の
  重み付きグラフと重みなしグラフに対する計算量はそれぞれ次式で与えられる.
  \begin{align}
    &\mathcal{O}\left(\|T(s)\|\log|T(s)|\right)
    \label{eq:decremental-shortest-path-update-weighted} \\
    &\mathcal{O}\left(\|T(s)\|\right)
    \label{eq:decremental-shortest-path-update-unweighted}
  \end{align}
\end{theorem}
\begin{proof}
  式\eqref{eq:decremental-shortest-path-update-weighted}のみについて示す.
  式\eqref{eq:decremental-shortest-path-update-unweighted}については,
  式\eqref{eq:decremental-shortest-path-update-weighted}と同じ要領で示すことができる.

  $t\in T(s)$を7行目の条件に従って二つに分ける.つまり,
  $\exists y|y\in\mathcal{P}(t)\land y\notin T(s)$と
  $\forall y\in\mathcal{P}(t),\,y\in T(s)$に分ける.
  7行目を満たす頂点$t\in T(s)$に対して,実行される処理は以下である.
  7行目および行目で$t$の先行を走査し,9行目で優先度付きキューに追加している.
  また,16行目で$\sigma_{st}$を更新するために後継を走査して,
  17行目から21行目のループで,$t$の先行を走査し,優先度付きキューに追加している.
  これらから,7行目を満たす頂点$t\in T(s)$に対する\textproc{UpdateSSSP}
  の計算量は,
  $\mathcal{O}(|\mathcal{P}(t)|+|\mathcal{P}(t)|+|\mathcal{S}(t)|\log|T(s)|)$
  である.

  7行目を満たさない頂点$t\in T(s)$に対して実行される処理は,
  7行目を満たす頂点に対して行われる実行から,9行目の優先度付きキュー
  に追加する処理を除いたものである.
  よって,7行目を満たさない頂点$t\in T(s)$に対して実行される処理の計算量は,
  $\mathcal{O}(|\mathcal{P}(t)|+|\mathcal{P}(t)|+|\mathcal{S}(t)|\log|T(s)|)$
  である.したがって,\textproc{UpdateSSSP}にある$s$が与えられたときの計算量は,
  \begin{equation*}
    \begin{aligned}
      \mathcal{O}(\sum_{t\in T(s)}(|\mathcal{P}(t)|+|\mathcal{S}(t)|+|\mathcal{S}(t)|\log|T(s)|)
      &=\mathcal{O}(\|T(s)\|+\|T(s)\|\log|T(s)|) \\
      &=\mathcal{O}(\|T(s)\|\log|T(s)|)
    \end{aligned}
  \end{equation*}
\end{proof}

同様に,辺削除時の\textproc{UpdateSTSP}の計算量は,$S$を$T$にすれば良いので,
\begin{equation*}
  \begin{aligned}
    &\mathcal{O}(\|S(v)\|+\|T(u)\|+\sum_{s\in S(v)}(\|T(s)\|\log|T(s)|)+\sum_{t\in T(u)}(\|S(t)\|\log|S(t)|))\\
    &\:=\mathcal{O}(\sum_{s\in S(v)}(\|T(s)\|\log|T(s)|)+\sum_{t\in T(u)}(\|S(t)\|\log|S(t)|)).
  \end{aligned}
\end{equation*}
である.

さらに,定理\ref{thm:decremental-shortest-path-update}より,辺削除時の全頂点間の最短経路を更新する
処理の計算量は,$S(v)$と$T(u)$を求め,$s\in S(v)$に対して\textproc{UpdateSSSP}を,
$t\in T(u)$に対して\textproc{UpdateSTSP}を実行するため,
\begin{align}
  &\mathcal{O}(\|S(v)\|+\|T(u)\|+\sum_{s\in S(v)}(\|T(s)\|\log|T(s)|)+\sum_{t\in T(u)}(\|S(t)\|\log|S(t)|)), \label{eq:full-decremental-shortest-path-update-weighted} \\
  &\:=\mathcal{O}(\sum_{s\in S(v)}(\|T(s)\|\log|T(s)|)+\sum_{t\in T(u)}(\|S(t)\|\log|S(t)|)). \label{eq:full-decremental-shortest-path-update-unweighted}
\end{align}
である.

本章の最後に,提案アルゴリズムの最悪計算量を評価する.
定理\ref{thm:decrease-betweenness-weight-complexity}および
定理\ref{thm:increase-betweenness-weight-complexity}より,
最短経路更新の前後で媒介中心性を更新するアルゴリズムの最悪計算量は,$S(v)=\tau(s)=V$のときで,
$\mathcal{O}(|V||E|+|V|^2\log|V|)$である.これは,Brandesのアルゴリズムの計算量と同じである.
また,定理\ref{thm:incremental-shortest-path-update}および
定理\ref{thm:decremental-shortest-path-update}より,
最短経路を更新するアルゴリズムの最悪計算量は,$S(v)=T(u)=V$のときで,
$\mathcal{O}(|V||E|\log|V|)$である.これは,Dijkstraのアルゴリズムを
全頂点に対して行うときの計算量と同じである.
