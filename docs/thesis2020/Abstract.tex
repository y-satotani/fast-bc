\abst{
  グラフ理論から派生したネットワーク科学が注目を集めてきた.
  ネットワーク科学とは,つながりから成る集団全体の特性や,
  つながりの中の個々の特性の理解を目的とした科学である.
  ネットワーク科学の強みは,物とそれらの間の関係が定義できれば,あらゆる分野に応用できることである.
  例えば,人のつながりや,原子の結合を対象にできる.

  巨大で複雑なネットワークを分析するうえで,ネットワークのノードの重要さを測ることは古くからある
  課題である.
  例えば,ソーシャルネットワーキングサービスにおける,いわゆるインフルエンサーや,
  スケールフリーネットワークにおけるハブといった重要なノードは
  ネットワーク全体に影響を与えうる.
  ゆえにそのようなノードを発見することは,応用上きわめて有益である.

  ノードの重要さについて定量的に議論するため,中心性の概念が導入された.
  現在まで用途ごとにいくつかの中心性が考案されてきた.
  その中でも媒介中心性は,すべての最短経路のうち,対象ノードを通るものの割合によって
  ノードの重要さを決定する中心性である.
  その中でも媒介中心性は,そのノードの上に最短経路がどの程度あるかを表した中心性である.
  あるノードの媒介中心性が高いならば,そのノードには比較的多くの最短経路が通っていることとなり,
  道路ネットワークや通信ネットワークにおいて重要なノードであると言える.

  実用上の観点から,媒介中心性を高速に計算することは,媒介中心性が提案されてから重要な課題である.
  媒介中心性を効率的に求めるアルゴリズムとして,Brandesのアルゴリズムが知られている.
  Brandesのアルゴリズムは,ノードやリンクの追加や削除が行われない,時不変ネットワークに対して高速に
  媒介中心性を計算する.
  一方,現実のネットワークはノードやリンクの追加や削除が起こる時変ネットワーク
  であることから,時変ネットワークに対しては,Brandesのアルゴリズムで始めから計算するより,
  変更に対する差分のみを計算する方がより効率的であると期待できる.
  この考えから,時変ネットワークの媒介中心性を効率的に更新する方法も提案されている.
  その中で,媒介中心性と共に最短経路を保持し,それらを更新する方法も提案されている.

  しかし,RamalingamとRepsの最短経路更新法に基づく辺削除時の媒介中心性更新アルゴリズムは
  知られていない.そこで,本研究ではRamalingamとRepsの最短経路更新アルゴリズムに基づく
  変削除時の媒介中心性更新法を提案する.
  さらに,提案手法の有用性を理論解析と数値実験の両面から検証する.

  提案したアルゴリズムの最悪計算時間量はBrandesのアルゴリズムと同じであるが,
  辺の追加や削除の影響が小さい場合は媒介中心性をより効率的に更新できることを確認した.
  また,実験によって人工ネットワークと実ネットワークの両方に対して提案手法の方が高速に更新する
  ことを確認した.
}
