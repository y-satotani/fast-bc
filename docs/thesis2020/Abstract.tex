\abst{
  グラフ理論から派生したネットワーク科学が注目を集めて久しい.
  ネットワーク科学の強みは,物とそれらの間の関係が定義できれば,あらゆる分野に応用できることである.
  例えば,社会科学における人のつながりや,化学における原子や分子の結合を対象にできる.

  複雑ネットワークを分析するうえで,ネットワークのノードの重要さに関して議論されることは自然である.
  例えば,いわゆるインフルエンサーや,スケールフリーネットワークにおけるハブといった重要なノードは
  ネットワーク全体に影響を与えうる.

  ノードの重要性に関して定量的に議論するため,中心性の概念が導入された.
  現在まで用途ごとにいくつかの中心性が考案されてきた.
  その中でも媒介中心性は,そのノードの上に最短経路がどの程度あるかを表した中心性である.
  あるノードの媒介中心性が高いならば,そのノードには比較的多くの最短経路が通っていることとなり,
  そのような意味で重要なノードであることを表す.

  媒介中心性を効率的に求めるアルゴリズムとして,Brandesのアルゴリズムが知られている.
  一方,現実のネットワークは時間とともにリンクが出現したり消滅したりする時変ネットワーク
  であることから,そのようなネットワークの媒介中心性を効率的に計算する方法も開発されている.
  その中で,ネットワークの全頂点ペアの最短経路を保持し,それを更新することで媒介中心性を
  更新する方法も提案されている.

  しかし,RamalingamとRepsの最短経路更新法に基づく辺削除時の媒介中心性更新アルゴリズムは
  知られていない.そこで,本研究ではRamalingamとRepsの最短経路更新アルゴリズムに基づく
  変削除時の媒介中心性更新法を提案する.
  さらに,その方法の有用性を理論解析と数値実験の両面から検証する.

  提案したアルゴリズムの最悪計算時間量はBrandesのアルゴリズムと同じであるが,
  ネットワークの末端の頂点に接続する辺の削除など,ネットワーク全体にとって変化が少ない場合は
  効率的に媒介中心性を更新できることを確認した.
  また,実験によって人工ネットワークと実ネットワークの両方に対して提案手法が高速に更新する
  ことを確認した.
}
