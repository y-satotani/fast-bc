\abst{
  巨大で複雑なネットワークを分析するうえで,ネットワークの中の個を表すノードの重要さを
  測ることは古くからの課題である.
  例えば,ソーシャルネットワーキングサービスにおける,いわゆるインフルエンサーや,
  スケールフリーネットワークにおけるハブといった重要なノードは
  ネットワーク全体に影響を与えうる.
  ゆえにそのようなノードを発見することは,応用上きわめて有益である.
  ノードの重要さについて定量的に議論するために,中心性の概念が導入され,
  現在まで用途ごとにいくつかの中心性が考案されてきた.
  その中でも媒介中心性は,すべての最短経路のうち,対象ノードを通るものの割合によって
  ノードの重要さを決定する中心性である.
  あるノードの媒介中心性が高ければ,そのノードには比較的多くの最短経路が通っていることを意味し,
  道路ネットワークや通信ネットワークにおいて重要なノードであると言える.

  媒介中心性の高速な計算は,実用上の観点から重要な課題である.
  媒介中心性を効率的に求めるアルゴリズムとして,Brandesのアルゴリズムが知られている.
  これは,ネットワークのノードやリンクの追加や削除が行われない,
  時不変ネットワークに対して媒介中心性を高速に計算する.
  一方,現実の多くのネットワークはノードやリンクの追加や削除が起こる時変ネットワーク
  であることから,時変ネットワークに対しては,Brandesのアルゴリズムを用いて初めから計算するより,
  変更に対する差分のみを計算する方がより効率的な計算が期待できる.

  本論文では,時変ネットワークに対する効率的な媒介中心性の計算法として,RamalingamとRepsの
  最短経路更新アルゴリズムに基づく辺削除時の媒介中心性更新法を提案する.

  また,提案方の有用性を理論解析と実験の両面から明らかにする.まず,理論解析により,
  辺削除の影響が小さい場合はBrandesのアルゴリズムより時間計算量が小さいことを示す.
  次に,実験により人工ネットワークと実ネットワークの両方に対してBrandesのアルゴリズムより
  高速に更新できることを示す.

  媒介中心性を保持し,辺削除時にそれらを更新する方法はいくつか知られているが,
  RamalingamとRepsの最短経路更新法を利用する方法は本論文が初めてである.
}
