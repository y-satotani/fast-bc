\abst{
  \textcolor{red}{ネットワーク科学の説明}

  \textcolor{red}{中心性の説明}

  媒介中心性はネットワークのノードの重要さをはかる指標として多くの場面で用いられている.
  媒介中心性を効率的に求めるアルゴリズムとして,Brandesのアルゴリズムが知られている.

  一方,現実のネットワークは時間とともにリンクが出現したり消滅したりする時変ネットワーク
  であることから,そのようなネットワークの媒介中心性を効率的に計算する方法も開発されている.
  その中で,ネットワークの全頂点ペアの最短経路を保持し,それを更新することで媒介中心性を
  更新する方法も提案されている.

  しかし,RamalingamとRepsの最短経路更新法に基づく辺削除時の媒介中心性更新アルゴリズムは
  知られていない.そこで,本研究ではRamalingamとRepsの最短経路更新アルゴリズムに基づく
  変削除時の媒介中心性更新法を提案する.
  さらに,その方法の有用性を理論解析と数値実験の両面から検証する.

  提案したアルゴリズムの最悪計算時間量はBrandesのアルゴリズムと同じであるが,
  ネットワークの末端の頂点に接続する辺の削除など,ネットワーク全体にとって変化が少ない場合は
  効率的に媒介中心性を更新できることを確認した.
  また,実験によって人工ネットワークと実ネットワークの両方に対して提案手法が高速に更新する
  ことを確認した.
}
