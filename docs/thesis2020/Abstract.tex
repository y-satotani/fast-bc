\abst{
  グラフ理論から派生したネットワーク科学が注目を集めてきた.
  ネットワーク科学とは,ネットワークと呼ばれる,個々のつながりから成る集団全体の特性や,
  ネットワークの中の個々の特性の理解を目的とした科学である.
  ネットワーク科学の強みは,個の間の関係,すなわち,つながりが定義できれば,
  あらゆる分野に応用できることである.

  巨大で複雑なネットワークを分析するうえで,ネットワークの中の個であるノードの重要さを
  測ることは古くからの課題である.
  例えば,ソーシャルネットワーキングサービスにおける,いわゆるインフルエンサーや,
  スケールフリーネットワークにおけるハブといった重要なノードは
  ネットワーク全体に影響を与えうる.
  ゆえにそのようなノードを発見することは,応用上きわめて有益である.

  ノードの重要さについて定量的に議論するため,中心性の概念が導入された.
  現在まで用途ごとにいくつかの中心性が考案されてきた.
  その中でも媒介中心性は,すべての最短経路のうち,対象ノードを通るものの割合によって
  ノードの重要さを決定する中心性である.
  あるノードの媒介中心性が高いならば,そのノードには比較的多くの最短経路が通っていることとなり,
  道路ネットワークや通信ネットワークにおいて重要なノードであると言える.

  媒介中心性を高速に計算することは,実用上の観点から,それが提案されて以来の重要な課題である.
  媒介中心性を効率的に求めるアルゴリズムとして,Brandesのアルゴリズムが知られている.
  Brandesのアルゴリズムは,ノードやリンクの追加や削除が行われない,
  時不変ネットワークに対して高速に媒介中心性を計算する.
  一方,現実のネットワークはノードやリンクの追加や削除が起こる時変ネットワーク
  であることから,時変ネットワークに対しては,Brandesのアルゴリズムを用いて始めから計算するより,
  変更に対する差分のみを計算する方がより効率的であると期待できる.
  この考えから,時変ネットワークの媒介中心性を効率的に更新する方法も提案されている.
  その中で,媒介中心性と共に最短経路を保持し,それらを更新する方法も提案されている.
  しかし,RamalingamとRepsの最短経路更新法に基づく辺削除時の媒介中心性更新アルゴリズムは
  知られていない.そこで,本研究ではRamalingamとRepsの最短経路更新アルゴリズムに基づく
  変削除時の媒介中心性更新法を提案する.
  同時に,本稿では辺挿入時のアルゴリズムとあわせて説明する.
  さらに,提案手法の有用性を理論解析と数値実験の両面から検証する.

  提案したアルゴリズムの最悪時間計算量はBrandesのアルゴリズムと同じであるが,
  辺の追加や削除の影響が小さい場合は媒介中心性をより効率的に更新できることを確認した.
  また,実験によって人工ネットワークと実ネットワークの両方に対して提案手法の方が高速に更新する
  ことを確認した.
}
