\chapter{一辺削除時の媒介中心性の変化量}
本章では,まず一辺削除時の最短経路の長さと個数の変化量について説明し,一辺削除時の媒介中心性の変化量について説明する.

次の補題は,ある二頂点間の最短経路の個数と,その経路に含まれる
短い最短経路の個数との関係を示す.
\begin{lemma}
  \label{lemma:number-of-paths}
  $G=(V,E)$の異なる二頂点$\{s,t\}\in V$について,$v$を$d_{st}=d_{sv}+d_{vt}$
  である頂点(ただし$v\neq s,t$)とすると,次が成り立つ.
  \begin{equation}
    \label{eq:number-of-paths}
    \sigma_{st}=\frac{\sum_{v}\sigma_{sv}\sigma_{vt}}{d_{st}-1}
  \end{equation}
\end{lemma}
\begin{proof}
  $s$と$t$の間の一般的な経路を図\ref{fig:proof-number-of-paths}に示す.
  \begin{figure}
    \centering
    \def\svgwidth{.5\columnwidth}
    \input{proof-number-of-paths.pdf_tex}
    \caption{$s$と$t$の一般的な最短経路}
    \label{fig:proof-number-of-paths}
  \end{figure}
  $s$からの距離が一定の頂点を並べて,一つの層とする.$d_{sv}=k$なる頂点
  $v$の集合を,第$k$層と定義し,$L_k$と表す.
  $L_k$に属する頂点の数を$n_k$,$L_k$に属する$l$番目の頂点を$v_{kl}$と表す.
  ここで,第$k$層に属する頂点$v$は,隣接する層(第$k-1$層と第$k+1$層)
  以外の層に属する頂点$w$と隣接しないことに注意する.
  もしそのような頂点が存在すると,最短経路長が変化する.
  式\ref{eq:number-of-paths}の両辺に$d_{st}-1$を掛けて,
  次の式\ref{eq:number-of-paths1}を得る.
  \begin{equation}
    \sigma_{st}(d_{st}-1)=\sum_{v}\sigma_{sv}\sigma_{vt}
    \label{eq:number-of-paths1}
  \end{equation}
  式\ref{eq:number-of-paths1}の右辺を,
  図\ref{fig:proof-number-of-paths}にならって表すと,
  \begin{equation}
    \sum_{v}\sigma_{sv}\sigma_{vt}=
    \sum_{k=1}^m\sum_{l=1}^{n_k}\sigma_{sv_{kl}}\sigma_{v_{kl}t}
    \label{eq:number-of-paths2}
  \end{equation}
  が得られる.ここで,二つの頂点$v$と$w$について,次の隣接を表す記号$a$を導入する.
  \begin{align*}
    a_{vw}=
    \begin{cases}
      1 & vとwが隣接しているとき \\
      0 & vとwが隣接していないとき
    \end{cases}
  \end{align*}
  各々の$\sigma_{sv_{kl}}\sigma_{v_{kl}t}$について議論する.
  $a$の定義を用いて式を変形すると,
  \begin{align}
    &\sigma_{sv_{kl}}\sigma_{v_{kl}t}\nonumber\\
    =&\left(\sum_{v'\in L_{k-1}}\sigma_{sv'}a_{v'v_{kl}}\right)
    \left(\sum_{v'\in L_{k+1}}\sigma_{v_{kl}v'}a_{v't}\right)
    \nonumber\\
    =&\left(\sum_{v''\in L_{k-2}}\sum_{v'\in L_{k-1}}
    \sigma_{sv''}a_{v''v'}a_{v'v_{kl}}\right)
    \left(\sum_{v'\in L_{k+1}}\sum_{v''\in L_{k+2}}
    a_{v_{kl}v'}a_{v'v''}\sigma_{v''t}\right)
    \nonumber\\
    &\vdots\nonumber\\
    =&\left(\sum_{(v_1,\ldots,v_{k-1})\in L_1\times\cdots\times L_{k-1}}
    a_{sv_1}\cdots a_{v_{k-1}v_{kl}}\right)
    \left(\sum_{(v_{k+1},\ldots,v_m)\in L_{k+1}\times\cdots\times L_m}
    a_{v_{kl}v_{k+1}}\cdots a_{v_mv_t}\right)\nonumber\\
    =&\sum_{(v_1,\ldots,v_{k-1},v_{k+1},\ldots,v_m)\in L_1\times\cdots\times L_{k-1}\times L_{k+1}\times\cdots\times L_m}
    a_{sv_1}\cdots a_{v_{k-1}v_{kl}}a_{v_{kl}v_{k+1}}\cdots a_{v_mt}
    \label{eq:number-of-paths3}
  \end{align}
  が得られる.式\ref{eq:number-of-paths3}を式\ref{eq:number-of-paths2}に
  代入すると,
  \begin{align}
    &\sum_{k=1}^m\sum_{l=1}^{n_k}\sigma_{sv_{kl}}\sigma_{v_{kl}t}\nonumber\\
    =&\sum_{k=1}^m\sum_{l=1}^{n_k}\sum_{
      (v_1,\ldots,v_{k-1},v_{k+1},\ldots,v_m)\in
      L_1\times\cdots\times L_{k-1}\times L_{k+1}\times\cdots\times L_m
    }a_{sv_1}\cdots a_{v_{k-1}v_{kl}}a_{v_{kl}v_{k+1}}\cdots a_{v_mt}\nonumber\\
    =&\sum_{k=1}^m\sum_{(v_1,\ldots,v_m)\in L_1\times\cdots\times L_m}
    a_{sv_1}\cdots a_{v_mt}\nonumber\\
    =&m\left(\sum_{(v_1,\ldots,v_m)\in L_1\times\cdots\times L_m}
    a_{sv_1}\cdots a_{v_mt}\right)
    \label{eq:number-of-paths4}
  \end{align}
  と変形できる.式\ref{eq:number-of-paths4}の総和の対象が$1$となるのは,
  $a_{sv_1},\ldots,a_{v_mt}$のすべてが$1$のとき,
  すなわち,$s$と$v_1$,$v_1$と$v_2$,$\ldots$,$v_m$と$t$がすべて隣接している
  とき,すなわち,$s$と$t$の最短経路となっているときである.
  従って,総和の値は$s$と$t$の最短経路の数と一致し,
  式\ref{eq:number-of-paths4}は$\sigma_{st}(d_{st}-1)$と等しい.
  従って,補題が成り立つ.
\end{proof}

\subsection*{辺の削除に対する頂点間距離の更新}
$G$から辺$e=\{\alpha,\beta\}\in E(G)$を削除した後の頂点間距離$d'_{st}$と
最短経路数$\sigma'_{st}$を求める.次の定理が成り立つ.
\begin{theorem}
  \label{thm:update-distance-on-delete}
  グラフ$G=(V,E)$から辺$\{\alpha,\beta\}\in E$を削除した後の
  頂点間距離$d'_{st}$と最短経路数$\sigma'_{st}$について,次が成り立つ.
  ただし,$d_{s\alpha}\leq d_{s\beta}$とする.
  $\sigma_{st}(\alpha,\beta)$は$s$と$t$の最短経路のうち,有向辺$(\alpha,\beta)$を通るものの個数である.
  \begin{enumerate}
  \item $\sigma_{st}(\alpha,\beta)=0$ならば,次が成り立つ.
    \[ d'_{st}=d_{st},\:\sigma'_{st}=\sigma_{st} \]
  \item $0<\sigma_{st}(\alpha,\beta)<\sigma_{st}$ならば,次が成り立つ.
    \[ d'_{st}=d_{st},\:\sigma'_{st}=\sigma_{st}-\sigma_{st}(\alpha,\beta) \]
  \item $\sigma_{st}(\alpha,\beta)=\sigma_{st}>0$ならば,次が成り立つ.
    \begin{equation*}
      \begin{aligned}
        d'_{st}&=\min\{d_{sv}+d_{vt}\,|\,v\in V,\,
        \sigma_{sv}>\sigma_{sv}(\alpha,\beta),\,
        \sigma_{vt}>\sigma_{vt}(\alpha,\beta) \\
        \sigma'_{st}&=\frac{\sum_v(\sigma'_{sv}\sigma'_{vt})}{d'_{st}-1}\:
        (v\in V,\,v\neq s,t,\,d'_{st}=d'_{sv}+d'_{vt})
      \end{aligned}
    \end{equation*}
  \end{enumerate}
\end{theorem}
\begin{proof}
  辺削除による更新の操作は次の三種類である.
  \begin{enumerate}
  \item 削除による影響はなく,何も行わない
  \item 削除により最短経路数を更新する
  \item 削除により頂点間距離を更新し,さらに最短経路数を再計算する
  \end{enumerate}

  一つ目の操作が実施される場合は,有向辺$(\alpha,\beta)$が$s$と$t$との
  最短経路に含まれていないときである.これは,補題\ref{lemma:path-num-2}より,
  $\sigma_{st}(\alpha,\beta)=0$のときと言い換えられる.

  二つ目の操作が実施される場合は,
  $s$と$t$との最短経路の集合$P\neq\varnothing$と,
  $P$の中で有向辺$(\alpha,\beta)$を含む経路の集合$P'\neq\varnothing$に対して,
  $P'\subset P$のときである.
  言い換えると,補題\ref{lemma:path-num-2}より,
  $0<\sigma_{st}(\alpha,\beta)<\sigma_{st}$のときである.
  このとき,削除により$P'$に属する最短経路がなくなるため,
  その数($\sigma_{st}(\alpha,\beta)$)だけ$\sigma_{st}$から引いた値とする.

  三つ目の操作が実施される場合は,
  $s$と$t$との最短経路の集合$P\neq\varnothing$と,
  $P$の中で辺$\{\alpha,\beta\}$を含む経路の集合$P'\neq\varnothing$に対して,
  $P'=P$のときである.
  言い換えると,補題\ref{lemma:path-num-2}より,
  $\sigma_{st}(\alpha,\beta)=\sigma_{st}>0$のときである.
  このとき,削除により$P$に属する最短経路がすべて無効になるため,頂点間距離を
  再計算する.補題\ref{lemma:distance-1}より,
  新しい頂点間距離$d'_{st}$を計算できる.このとき,中間の頂点$v$は
  削除による頂点間距離の更新が起こらない頂点を選ぶ.
  また,新しい最短経路数$\sigma'_{st}$は補題\ref{lemma:number-of-paths}により
  計算できる.
\end{proof}

定理\ref{thm:update-distance-on-delete}より,一辺削除時の頂点間距離の
更新の方法が分かる.
その手順ををアルゴリズム\ref{algo:update-distance-on-delete}に示す.
\begin{algorithm}[H]
  \caption{辺$\{\alpha,\beta\}$が削除されたときの$d'_{st}$と$\sigma'_{st}$の
    更新}\label{algo:update-distance-on-delete}
  \begin{algorithmic}[1]
    \Require $G=(V,E),\ d_{st},\,\sigma_{st}$
    \Ensure $d'_{st},\,\sigma'_{st}$
    \State $P\gets()$
    \Comment{更新の対象となる頂点組$(s,t)$}
    \ForAll{$s\in\{1,2,\ldots,N\}$}
    \ForAll{$t\in\{1,2,\ldots,N\}$}
    \If{$\sigma_{st}(\alpha,\beta)=0$}
    \State $d'_{st}=d_{st}$
    \State $\sigma'_{st}=\sigma_{st}$
    \ElsIf{$0<\sigma_{st}(\alpha,\beta)<\sigma_{st}$}
    \State $d'_{st}=d_{st}$
    \State $\sigma'_{st}=\sigma_{st}-\sigma_{st}(\alpha,\beta)$
    \ElsIf{$\sigma_{st}(\alpha,\beta)=\sigma_{st}>0$}
    \Comment{頂点間距離を再計算}
    \State $d_{\min}\gets \infty$
    \ForAll{$v\in V,\,\sigma_{sv}>\sigma_{sv}(\alpha,\beta),\,\sigma_{vt}>\sigma_{vt}(\alpha,\beta)$}
    \State $d_{\min}\gets\min\left\{d_{\min},d_{sv}+d_{vt}\right\}$
    \EndFor
    \If{$d_{\min}=\infty$}
    \State $d'_{st}\gets\infty$
    \State $\sigma'_{st}\gets0$
    \Else
    \State $d'_{st}\gets d_{\min}$
    \State \parbox[t]{\linewidth}{
      $P\gets(\ldots,(s_i,t_i),(s,t),(s_{i+1},t_{i+1}),\ldots)$ \\
      ただし,$d'_{s_i,t_i}\leq d'_{st}\leq d'_{s_{i+1},t_{i+1}},\ \ldots,(s_i,t_i),\ldots\in P$
    }
    \EndIf
    \EndIf
    \EndFor
    \EndFor
    \ForAll{$(s_i,t_i)\in P$}
    \Comment{最短経路数を再計算}
    \State $\sigma'\gets0$
    \ForAll{$v\in V,\:v\neq s_i,t_i,\:d'_{s_i,t_i}=d'_{s_i,v}+d'_{v,t_i}$}
    \State $\sigma'\gets\sigma'+\sigma'_{s_i,v}\sigma'_{v,t_i}$
    \EndFor
    \State $\sigma'_{s_i,t_i}\gets\sigma'/(d'_{s_i,t_i}-1)$
    \EndFor
  \end{algorithmic}
\end{algorithm}

アルゴリズム\ref{algo:update-distance-on-delete}では,頂点間距離の
再計算の対象の組$(s,t)$を$d'_{st}$の昇順になるようにリストに追加している.
この根拠は次の系\ref{coll:update-distance-on-delete}で与える.
\begin{corollary}
  \label{cor:update-distance-on-delete}
  頂点間距離を再計算した組$(s,t)$について,最短経路数
  $\sigma'_{st}$を再計算するには,すべての$d'_{st}>d'_{uv}$なる
  $(u,v)$の$\sigma'_{uv}$を先に計算しなければならない.
\end{corollary}
\begin{proof}
  定理\ref{thm:update-distance-on-delete}の更新手順より,
  $s$と$t$の最短経路数の中に$d'_{st}>d'_{uv}$なる$u$と$v$の最短経路数
  が含まれる.よって,$\sigma'_{st}$を再計算するには,すべての
  $d'_{st}>d'_{uv}$なる$(u,v)$の$\sigma'_{uv}$先に計算しなければならない.
\end{proof}
