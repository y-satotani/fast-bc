\chapter{一辺削除時の媒介中心性更新法}
\label{chap:update-bc-on-delete}
本章では,$G=(V,E)$に辺$e=\{\alpha,\beta\} \not\in E$を削除して得られるグラフを
$G'=(V,E')$とする.$G'$における頂点$s$から頂点$t\,(\neq s)$への最短経路を有向グラフ
$G'_{st}=(V'_{st},E'_{st})$で表し,すべての頂点から頂点$s$への最短経
路を有向グラフ$G'_s=(V,E'_s)$で表す.また,$G'$における頂点$s$から
頂点$t\,(\neq s)$への最短経路の個数と長さをそれぞれ$\sigma'_{st}$, 
$d'_{st}$で表す.

\section{最短経路長または最短経路数が変化する頂点の導出}

\begin{lemma}
  \textcolor{red}{TODO}
  頂点$x\in G$が,$d_{xz}<d'_{xz}$または...を満たすための必要十分条件は次のいずれかである.
  \begin{enumerate}
  \item $v$
  \item $y\in\mathcal{N}_G'(x)$のうち,$(x,y)\in E_z$を満たすものについて,そのすべてが影響を受けるような頂点$x$
  \end{enumerate}
\end{lemma}
\begin{proof}
例えば,図\ref{fig:TODO}において,$x_1$の近傍のうち,$z$への最短経路を通っているのは$v$で,$v$は影響を受ける.そのため,$x_1$は削除の影響を受ける.一方,$y$の近傍のうち,$z$への最短経路を通っているのは2個あるが,そのうちひとつは影響を受けないので,$y$は削除の影響を受けない.
\end{proof}

\section{最短経路長および最短経路数の更新}

\begin{lemma}
  影響を受けることが分かった頂点$x$について,影響を受けない,または,最短経路長が確定した頂点$y$が近傍にあるなら,$d'_{xz}\gets l_{xy}+d'_{yz}$として,$x$の最短経路長$d'_{xz}$を確定させる.
\end{lemma}

例えば,図\ref{fig:TODO}において,$v$と$x_1$は,近傍に影響を受けない頂点$y$をもつため,直ちに最短経路長を確定することができる.$x_2$と$x_3$は$x_1$が確定した後に最短経路長を確定することができる.

\section{ペア依存度の更新}
\label{sect:update-delta-on-delete}

削除後のペア依存度$\delta'_z(x)$をBrandesのアルゴリズムによって求める.
具体的には,\ref{sect:update-augdist-on-delete}で求めた影響を受ける頂点から,
$d'_{xz}$の降順となるように次式を計算しながら走査する.
\[ \delta'_{z}(x)\gets\sum_{(y,x)\in E_z}\frac{\sigma'_{xz}}{\sigma'_{yz}}(1+\delta'_z(y)) \]

\section{計算量の解析}

\begin{lemma}
  提案した辺削除時の媒介中心性更新アルゴリズムの最悪計算量は,
  \[ \mathcal{O}(EV+V^2\log V) \]
  である.ただし,頂点数を$V$,辺数を$E$とする,
\end{lemma}
\begin{proof}
  提案したアルゴリズムは,次の4つの工程で構成される.
  \begin{enumerate}
  \item 最短経路長の更新
  \item 最短経路数の更新
  \item ペア依存度の更新
  \item 媒介中心性の更新
  \end{enumerate}

  以下,それぞれの工程の最悪計算量を解析する.頂点数を$V$,辺数を$E$とする.

  最悪の場合は,Dijkstraのアルゴリズムを$V$回繰り返すことになので,計算量は
  $\mathcal{O}(VE+V^2\log V)$である.

  同様,最悪の場合は,Dijkstraのアルゴリズムを$V$回繰り返すことになので,計算量は
  $\mathcal{O}(VE+V^2\log V)$である.

  最悪の場合は,すべての頂点組のペア依存度を更新するので,
  Brandesのアルゴリズムの計算量と一致する.つまり,
  $\mathcal{O}(VE+V^2\log V)$.

  データ構造の都合で,頂点数$\times$頂点数の2次元配列の要素をひとつの軸に沿って
  足し合わせる.そのため,計算量は$\mathcal{O}(V^2)$である.

  したがって,全体の最悪計算量は,
  \[ \mathcal{O}(VE+V^2\log V+V^2)=\mathcal{O}(VE+V^2\log V) \]
  である.
\end{proof}

実験結果から,提案手法の実行時間はBrandesのアルゴリズムの定数倍に近いことが分かる.
このことは,両アルゴリズムの計算量は一致するが,提案手法の方は計算量を足し合わせているので,その分計算時間が余計にかかっていることを示唆する.よって,解析の結果と矛盾しない.

