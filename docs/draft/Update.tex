\chapter{媒介中心性の変化量の理論解析}
本章では辺の挿入と削除によって各頂点の媒介中心性がどのように変化するかを理論的に解析する.

\section{最短経路の長さと個数}
本節では,最短経路の長さと個数について,いくつかの補題を示す.

\begin{lemma}[Brandes~\cite{Brandes2001}]
$G=(V,E)$の異なる2頂点$s,t \in V$に対して,$s$から$t$への最短経路$G_{st}=(V_{st},E_{st})$が$i \in V$を含む,すなわち$i \in V_{st}$であるための必要十分条件は
\begin{equation}
  d_{st}=d_{si}+d_{it}
  \label{eqn:inclusion0}
\end{equation}
が成り立つことである.
\label{lemma:1}
\end{lemma}
\begin{proof}
はじめに$i=s$の場合を考える.このとき明らかに$i \in V_{st}$であり,かつ$d_{si}+d_{it}=d_{ss}+d_{st}=d_{st}$より\eqref{eqn:inclusion0}もつねに成り立つ.$i=t$の場合も同様である.そこで以下では$i \not\in \{s,t\}$と仮定する.$G_{st}$が頂点$i$を含むならば,$G_{st}$の中に$s \rightarrow \cdots \rightarrow i \rightarrow \cdots \rightarrow t$の順に頂点を通る有向道が存在する.この有向道の長さは$d_{si}+d_{it}$で与えられるので\eqref{eqn:inclusion0}が成り立つ.次に\eqref{eqn:inclusion0}が成り立つと仮定する.頂点$s$を出発して$s$から$i$への最短経路の一つを通って$i$に行き,次に$i$から$t$への最短経路の一つを通って$t$に行く有向道を考える.この有向道は$s$から$t$への最短経路の一つである.なぜなら,その長さは$d_{si}+d_{it}$で与えられ,\eqref{eqn:inclusion0}より$s$から$t$への最短経路長$d_{st}$に等しいからである.したがって$i \in V_{st}$が成り立つ.
\end{proof}

\begin{lemma}
$G=(V,E)$の異なる2頂点$s,t\in V$と辺$\{i,j\} \in E$を考える.$s$から$t$への最短経路$G_{st}=(V_{st},E_{st})$が有向辺$(i,j)$を含む,すなわち$(i,j) \in E_{st}$が成り立つための必要十分条件は
\begin{equation}
d_{st}=d_{si}+d_{jt}+1
\label{eqn:inclusion}
\end{equation}
が成り立つことである.
\label{lemma:2}
\end{lemma}
\begin{proof}
はじめに$i=s$, $j=t$の場合を考える.このとき明らかに$G_{st}$は
有向辺$(i,j)$を含み,かつ$d_{si}+d_{jt}+1=d_{ss}+d_{tt}+1=1=d_{st}$
より\eqref{eqn:inclusion}も成り立つ.次に$i=s$, $j \neq t$の場合を
考える.このとき$(i,j)\in E_{st}$であるための必要十分条件は$G_{st}$が
$j$を含むことである.これは補題~\ref{lemma:1}より
$d_{st}=d_{sj}+d_{jt}$と等価であり,さらに右辺$d_{sj}+d_{jt}$は
\eqref{eqn:inclusion}の右辺と等しい.$i\neq s$, $j=t$の場合も同様である.
そこで以下では$s,t,i,j$がすべて異なると仮定する.
$G_{st}$が有向辺$(i,j)$を含むならば,$G_{st}$の中に
$s \rightarrow \cdots \rightarrow i \rightarrow j \rightarrow \cdots \rightarrow t$
の順に頂点を通る有向道が存在する.
この有向道の長さは$d_{si}+1+d_{jt}$で与えられるので\eqref{eqn:inclusion}が
成り立つ.次に\eqref{eqn:inclusion}が成り立つと仮定する.頂点
$s$を出発して$s$から$i$への最短経路の一つを通って$i$に行き,次に
辺$\{i,j\}$を通って$j$に行き,最後に$j$から$t$への最短経路の一つを
通って$t$に行く有向道が存在する.この有向道は$s$から$t$への最短経路の一つで
ある.なぜなら,その長さは$d_{si}+1+d_{jt}$で与えられ,
\eqref{eqn:inclusion}より$s$から$t$への最短経路長$d_{st}$に
等しいからである.したがって,$(i,j) \in E_{st}$が成り立つ.
\end{proof}

\begin{lemma}[Brandes\cite{Brandes2001}]
$G=(V,E)$の異なる2頂点$s,t \in V$に対して,
$s$から$t$への最短経路の中で$i$を通るものの個数$\sigma_{st}(i)$は
次式で与えられる.
\begin{equation}
\sigma_{st}(i)=
\left\{
\begin{array}{ll}
\sigma_{si} \sigma_{it}, & d_{st}=d_{si}+d_{it}\,\mbox{のとき} \\
0, & \mbox{それ以外のとき}
\end{array}
\right.
\label{eqn:sigma_sti}
\end{equation}
\label{lemma:3}
\end{lemma}

\begin{lemma}
$G=(V,E)$の異なる2頂点$s,t \in V$に対して,
$s$から$t$への最短経路の中で有向辺$(i,j)$を通るものの個数を
$\sigma_{st}(i,j)$とおくと,それは次式で与えられる.
\begin{equation*}
\sigma_{st}(i,j)=
\left\{
\begin{array}{ll}
\sigma_{si} \sigma_{jt}, & d_{st}=d_{si}+d_{jt}+1\,\mbox{のとき} \\
0, & \mbox{それ以外のとき}
\end{array}
\right.
%\label{eqn:sigma_stij}
\end{equation*}
\label{lemma:4}
\end{lemma}

次の補題は,ある二頂点間の最短経路の長さと,その経路に含まれる短い最短経路の長さに関するものである.
\begin{lemma}
  \label{lemma:distance-of-path}
  $G=(V,E)$の異なる二頂点$s,t\in V$について,次が成り立つ.
  \begin{equation*}
    d_{st}=\min\{d_{si}+d_{it}|i\in V\}
  \end{equation*}
\end{lemma}
\begin{proof}
  TBA
\end{proof}

次の補題は,ある二頂点間の最短経路の個数と,その経路に含まれる
短い最短経路の個数との関係を示す.
\begin{lemma}
  \label{lemma:number-of-paths}
  $G=(V,E)$の異なる二頂点$s,t\in V$について,$v$を$d_{st}=d_{sv}+d_{vt}$
  である頂点(ただし$v\neq s,t$)とすると,次が成り立つ.
  \begin{equation}
    \label{eq:number-of-paths}
    \sigma_{st}=\frac{\sum_{v}\sigma_{sv}\sigma_{vt}}{d_{st}-1}
  \end{equation}
\end{lemma}
\begin{proof}
  $s$と$t$の間の一般的な経路を図\ref{fig:proof-number-of-paths}に示す.
  \begin{figure}
    \centering
    \def\svgwidth{.5\columnwidth}
    \input{proof-number-of-paths.pdf_tex}
    \caption{$s$と$t$の一般的な最短経路}
    \label{fig:proof-number-of-paths}
  \end{figure}
  $s$からの距離が一定の頂点を並べて,一つの層とする.$d_{sv}=k$なる頂点
  $v$の集合を,第$k$層と定義し,$L_k$と表す.
  $L_k$に属する頂点の数を$n_k$,$L_k$に属する$l$番目の頂点を$v_{kl}$と表す.
  ここで,第$k$層に属する頂点$v$は,隣接する層(第$k-1$層と第$k+1$層)
  以外の層に属する頂点$w$と隣接しないことに注意する.
  もしそのような頂点が存在すると,最短経路長が変化する.
  式\eqref{eq:number-of-paths}の両辺に$d_{st}-1$を掛けて,
  次の式\eqref{eq:number-of-paths1}を得る.
  \begin{equation}
    \sigma_{st}(d_{st}-1)=\sum_{v}\sigma_{sv}\sigma_{vt}
    \label{eq:number-of-paths1}
  \end{equation}
  式\eqref{eq:number-of-paths1}の右辺を,
  図\ref{fig:proof-number-of-paths}にならって表すと,
  \begin{equation}
    \sum_{v}\sigma_{sv}\sigma_{vt}=
    \sum_{k=1}^m\sum_{l=1}^{n_k}\sigma_{sv_{kl}}\sigma_{v_{kl}t}
    \label{eq:number-of-paths2}
  \end{equation}
  が得られる.ここで,二つの頂点$v$と$w$について,次の隣接を表す記号$a$を導入する.
  \begin{align*}
    a_{vw}=
    \begin{cases}
      1 & vとwが隣接しているとき \\
      0 & vとwが隣接していないとき
    \end{cases}
  \end{align*}
  各々の$\sigma_{sv_{kl}}\sigma_{v_{kl}t}$について議論する.
  $a$の定義を用いて式を変形すると,
  \begin{align}
    &\sigma_{sv_{kl}}\sigma_{v_{kl}t}\nonumber\\
    =&\left(\sum_{v'\in L_{k-1}}\sigma_{sv'}a_{v'v_{kl}}\right)
    \left(\sum_{v'\in L_{k+1}}\sigma_{v_{kl}v'}a_{v't}\right)
    \nonumber\\
    =&\left(\sum_{v''\in L_{k-2}}\sum_{v'\in L_{k-1}}
    \sigma_{sv''}a_{v''v'}a_{v'v_{kl}}\right)
    \left(\sum_{v'\in L_{k+1}}\sum_{v''\in L_{k+2}}
    a_{v_{kl}v'}a_{v'v''}\sigma_{v''t}\right)
    \nonumber\\
    &\vdots\nonumber\\
    =&\left(\sum_{(v_1,\ldots,v_{k-1})\in L_1\times\cdots\times L_{k-1}}
    a_{sv_1}\cdots a_{v_{k-1}v_{kl}}\right)
    \left(\sum_{(v_{k+1},\ldots,v_m)\in L_{k+1}\times\cdots\times L_m}
    a_{v_{kl}v_{k+1}}\cdots a_{v_mv_t}\right)\nonumber\\
    =&\sum_{(v_1,\ldots,v_{k-1},v_{k+1},\ldots,v_m)\in L_1\times\cdots\times L_{k-1}\times L_{k+1}\times\cdots\times L_m}
    a_{sv_1}\cdots a_{v_{k-1}v_{kl}}a_{v_{kl}v_{k+1}}\cdots a_{v_mt}
    \label{eq:number-of-paths3}
  \end{align}
  が得られる.式\eqref{eq:number-of-paths3}を式\eqref{eq:number-of-paths2}に
  代入すると,
  \begin{align}
    &\sum_{k=1}^m\sum_{l=1}^{n_k}\sigma_{sv_{kl}}\sigma_{v_{kl}t}\nonumber\\
    =&\sum_{k=1}^m\sum_{l=1}^{n_k}\sum_{
      (v_1,\ldots,v_{k-1},v_{k+1},\ldots,v_m)\in
      L_1\times\cdots\times L_{k-1}\times L_{k+1}\times\cdots\times L_m
    }a_{sv_1}\cdots a_{v_{k-1}v_{kl}}a_{v_{kl}v_{k+1}}\cdots a_{v_mt}\nonumber\\
    =&\sum_{k=1}^m\sum_{(v_1,\ldots,v_m)\in L_1\times\cdots\times L_m}
    a_{sv_1}\cdots a_{v_mt}\nonumber\\
    =&m\left(\sum_{(v_1,\ldots,v_m)\in L_1\times\cdots\times L_m}
    a_{sv_1}\cdots a_{v_mt}\right)
    \label{eq:number-of-paths4}
  \end{align}
  と変形できる.式\eqref{eq:number-of-paths4}の総和の対象が$1$となるのは,
  $a_{sv_1},\ldots,a_{v_mt}$のすべてが$1$のとき,
  すなわち,$s$と$v_1$,$v_1$と$v_2$,$\ldots$,$v_m$と$t$がすべて隣接している
  とき,すなわち,$s$と$t$の最短経路となっているときである.
  従って,総和の値は$s$と$t$の最短経路の数と一致し,
  式\eqref{eq:number-of-paths4}は$\sigma_{st}(d_{st}-1)$と等しい.
  従って,補題が成り立つ.
\end{proof}

\section{一辺挿入時の媒介中心性の変化量}
\label{sect:update-bc-on-insert}
本節では,$G=(V,E)$に辺$e=\{\alpha,\beta\} \not\in E$を挿入して得られるグラフを
$G'=(V,E')$とする.$G'$における頂点$s$から頂点$t\,(\neq s)$への最短経路を有向グラフ
$G'_{st}=(V'_{st},E'_{st})$で表し,頂点$s$から他のすべての頂点への最短経
路を有向グラフ$G'_s=(V,E'_s)$で表す.また,$G'$における頂点$s$から
頂点$t\,(\neq s)$への最短経路の個数と長さをそれぞれ$\sigma'_{st}$, 
$d'_{st}$で表し,$\sigma'_{st}$個の最短経路のうち頂点$i$を通るもの
の個数を$\sigma'_{st}(i)$で表す.

以下では$G_{st}$, $d_{st}$, $\sigma_{st}$と
$G'_{st}$, $d'_{st}$, $\sigma'_{st}$, $\sigma'_{st}(i)$の関係について考察する.
ただし,一般性を失うことなく
\begin{equation}
d_{s\alpha} \leq d_{s\beta}
\label{eqn:assum}
\end{equation}
と仮定し,便宜上
$V_{ss}=V'_{ss}=\{s\}$, $E_{ss}=E'_{ss}=\emptyset$, $d'_{ss}=0$, 
$\sigma'_{ss}=0$とおく.

\begin{lemma}
$t=\alpha$ならば$G'_{st}=G_{st}$, $d'_{st}=d_{st}$, 
$\sigma'_{st}=\sigma_{st}$である.
\label{lemma:5}
\end{lemma}
\begin{proof}
$G'_{st} \neq G_{st}$となるのは$d_{s\beta}+1 \leq d_{st}(=d_{s\alpha})$が
成り立つときに限られる.これは\eqref{eqn:assum}と矛盾するので
$G'_{st}=G_{st}$である.このとき明らかに$d'_{st}=d_{st}$, 
$\sigma'_{st}=\sigma_{st}$が成り立つ.
\end{proof}

\begin{lemma}
$t=\beta$ならば以下が成り立つ.
\begin{enumerate}
\item $d_{s\alpha}+1<d_{st}$ならば$G'_{st}$は$G_{s\alpha}$と有向辺$(\alpha,t)$からなり,$d'_{st}=d_{s\alpha}+1$, $\sigma'_{st}=\sigma_{s\alpha}$である.
\item $d_{s\alpha}+1=d_{st}$ならば$G'_{st}$は$G_{st}$と$G_{s\alpha}$と有向辺$(\alpha,t)$からなり,$d'_{st}=d_{st}$, $\sigma'_{st}=\sigma_{st}+\sigma_{s\alpha}$である.
\item $d_{s\alpha}=d_{st}$ならば$G'_{st}=G_{st}$, $d'_{st}=d_{st}$, $\sigma'_{st}=\sigma_{st}$である.
\end{enumerate}
\label{lemma:6}
\end{lemma}
\begin{proof}
$t=\beta$ならば$d_{s\alpha}>d_{st}(=d_{s\beta})$が成り立つことはない.
式\eqref{eqn:assum}と矛盾するからである.
はじめに1番目の主張を証明する.$d_{s\alpha}+1<d_{st}$と仮定する.
$G'$において$s$から辺$\{\alpha,t\}$を通って$t$に行く最短経路の長さは$d_{s\alpha}+1$で与えられ,$s$から辺$\{\alpha,t\}$を通らずに$t$に行く最短経路の長さは$G$における$s$から$t$への最短経路の長さ$d_{st}$に等しい.このことと仮定$d_{s\alpha}+1<d_{st}$より,$G'$における$s$から$t$へのすべての最短経路は辺$\{\alpha,t\}$を通る.すなわち$G'_{st}$は$G_{s\alpha}$と有向辺$(\alpha,t)$からなる.また,補題\ref{lemma:1}より$d'_{st}=d_{s\alpha}+d_{\alpha t}=d_{s\alpha}+1$であり,補題\ref{lemma:3}より$\sigma'_{st}=\sigma_{s\alpha}\sigma_{\alpha t}=\sigma_{s\alpha}$である.
%
次に2番目の主張を証明する.$d_{s\alpha}+1=d_{st}$と仮定する.このとき$\alpha \not\in V_{st}$である.なぜなら,仮に$\alpha \in V_{st}$とすると,補題\ref{lemma:1}より$d_{st}=d_{s\alpha}+d_{\alpha t}$が成り立ち,$\{\alpha,t\} \not\in E$より$d_{\alpha t} \geq 2$ が成り立つので,$d_{s\alpha}+1=d_{st}$と矛盾するからである.$G'$において$s$から$\alpha$を
通って$t$に行く最短経路の長さは$d_{s\alpha}+1$で与えられ,$s$から$\alpha$
を通らずに$t$に行く最短経路の長さは$d_{st}$で与えられる.仮定$d_{s\alpha}+1=d_{st}$
よりこれらは等しいので,$G'$における$s$から$t$への最短経路は
$G_{st}$と$G_{s\alpha}$と有向辺$(\alpha,t)$からなる.このとき$d'_{st}=d_{st}$, $\sigma'_{st}=\sigma_{st}+\sigma_{s\alpha}$である.
%
最後に3番目の主張を証明する.$d_{s\alpha}=d_{st}$と仮定する.
このとき$s$から辺$\{\alpha,t\}$を通って
$t$に行く最短経路の長さ$d_{s\alpha}+1$が$d_{st}$より大きいことから,$s$から
$t$への最短経路の中に$\{\alpha,t\}$を通るものはない.したがって,$G'_{st}=G_{st}$
であり,これより明らかに$d'_{st}=d_{st}$, $\sigma'_{st}=\sigma_{st}$である.
\end{proof}

\begin{lemma}
$t \neq \alpha, \beta$かつ$\min\{d_{s\alpha}+d_{\beta t}, d_{s\beta}+d_{\alpha t}\}+1
\leq d_{st}$ならば$d_{s\alpha}<d_{s\beta}$かつ$d_{s\alpha}+d_{\beta t} < d_{s\beta}+d_{\alpha t}$である.
\label{lemma:7}
\end{lemma}
\begin{proof}
背理法で証明する.はじめに$d_{s\alpha}=d_{s\beta}$と仮定する.また,一般性を
失うことなく$d_{\beta t} \leq d_{\alpha t}$とする.このとき
\[
 d_{s\beta}+d_{\beta t}+1=d_{s\alpha}+d_{\beta t}+1 \leq d_{st}
\]
が成り立つが,これは$s$を出発して$\beta$を通って$t$に行く最短経路の長さ
$d_{s\beta}+d_{\beta t}$が$d_{st}$より短いことを意味するので矛盾である.
したがって$d_{s\alpha}<d_{s\beta}$でなければならない.次に
$d_{s\alpha}+d_{\beta t} > d_{s\beta}+d_{\alpha t}$と仮定する.
このとき$d_{s\alpha}<d_{s\beta}$より
\[
 d_{s\alpha}+d_{\alpha t}+1 < d_{s\beta}+d_{\alpha t}+1 \leq d_{st}
\]
が得られる.これは$s$を出発して$\alpha$を通って$t$に行く最短経路の長さ
$d_{s\alpha}+d_{\alpha t}$が$d_{st}$より
小さいことを意味するので矛盾である.最後に
$d_{s\alpha}+d_{\beta t} = d_{s\beta}+d_{\alpha t}$と仮定すると
\begin{align}
d_{s\alpha}+d_{\beta t}+1 &\leq d_{st} \\
d_{s\beta }+d_{\alpha t}+1 &\leq d_{st}
\end{align}
が成り立つ.これらの辺々を加えると
\[
 (d_{s\alpha}+d_{\alpha t})+(d_{s\beta}+d_{\beta t})+2 \leq 2d_{st}
\]
となるので,$d_{s\alpha}+d_{\alpha t}$と$d_{s\beta}+d_{\beta t}$の少なくとも
一つは$d_{st}-1$以下である.ところがこれは$s$から$t$への経路の中で距離が
$d_{st}$より小さいものが存在することを意味するので矛盾である.
\end{proof}
%

補題\ref{lemma:7}より次の補題が得られる.

\begin{lemma}
$t \neq \alpha, \beta$ならば以下が成り立つ.
\begin{enumerate}
\item $\min\{d_{s\alpha}+d_{\beta t}, d_{s\beta}+d_{\alpha t}\}+1<d_{st}$
ならば$G'_{st}$は$G_{s\alpha}$と有向辺$(\alpha,\beta)$と$G_{\beta t}$からなり,$d'_{st}=d_{s\alpha}+1+d_{\beta t}$, $\sigma'_{st}=\sigma_{s\alpha}\sigma_{\beta t}$である.
\item $\min\{d_{s\alpha}+d_{\beta t}, d_{s\beta}+d_{\alpha t}\}+1=d_{st}$
ならば$G'_{st}$は$G_{st}$と$G_{s\alpha}$と有向辺$(\alpha,\beta)$と$G_{\beta t}$からなり,$d'_{st}=d_{st}$, $\sigma'_{st}=\sigma_{st}+\sigma_{s\alpha}\sigma_{\beta t}$である.
\item $\min\{d_{s\alpha}+d_{\beta t}, d_{s\beta}+d_{\alpha t}\}+1>d_{st}$ならば$G'_{st}=G_{st}$, $d'_{st}=d_{st}$, $\sigma'_{st}=\sigma_{st}$である.
\end{enumerate}
\label{lemma:8}
\end{lemma}
\begin{proof}
まず1番目の主張を証明する.
$\min\{d_{s\alpha}+d_{\beta t}, d_{s\beta}+d_{\alpha t}\}+1<d_{st}$
ならば補題\ref{lemma:7}より$d_{s\alpha}+d_{\beta t}<d_{s\beta}+d_{\alpha t}$が成り立つので,$G'$において$s$から辺$\{\alpha,\beta\}$を通って$t$に行くすべての最短経路は辺$\{\alpha,\beta\}$を$\alpha \rightarrow \beta$の向きに通る.
また,それらの長さは$d_{s\alpha}+1+d_{\beta t}$で与えられる.一方,
$s$から辺$\{\alpha,\beta\}$を通らずに$t$に行く最短経路の長さは$G$に
おける$s$から$t$への最短経路の長さ$d_{st}$に等しい.
このことと$d_{s\alpha}+1+d_{\beta t}<d_{st}$より,$G'$における$s$から$t$への
すべての最短経路は辺$\{\alpha,\beta\}$を$\alpha \rightarrow \beta$の向きに通る.したがって,$G'_{st}$は$G_{s\alpha}$と有向辺$(\alpha,\beta)$と$G_{\beta t}$からなる.また,補題~\ref{lemma:2}より$d'_{st}=d_{s\alpha}+d_{\beta t}+1$であり,補題~\ref{lemma:4}より$\sigma'_{st}=\sigma_{s\alpha}\sigma_{\beta t}$である.
%
次に2番目の主張を証明する.
$\min\{d_{s\alpha}+d_{\beta t}, d_{s\beta}+d_{\alpha t}\}+1=d_{st}$ならば
補題~\ref{lemma:7}より$d_{s\alpha}+d_{\beta t}<d_{s\beta}+d_{\alpha t}$が成り立つので,$G'$における$s$から$t$への最短経路は,
辺$\{\alpha,\beta\}$を$\alpha \rightarrow \beta$の向きに通るものと
辺$\{\alpha,\beta\}$を通らないものに分類される.したがって
$G'$における$s$から$t$への最短経路の集合は$G_{st}$と$G_{s\alpha}$と有向辺$(\alpha,\beta)$と$G_{\beta t}$からなる.また,このとき$d'_{st}=d_{st}$, $\sigma'_{st}=\sigma_{st}+\sigma_{s\alpha}\sigma_{\beta t}$である.

最後に3番目の主張を証明する.
$\min\{d_{s\alpha}+d_{\beta t}, d_{s\beta}+d_{\alpha t}\}+1>d_{st}$ならば
$G'$における$s$から$t$への最短経路$G'_{st}$は$G_{st}$に等しいので,
$d'_{st}=d_{st}$, $\sigma'_{st}=\sigma_{st}$である.
\end{proof}

補題~\ref{lemma:8}では$\min\{d_{s\alpha}+d_{\beta t}, d_{s\beta}+d_{\alpha t}\}+1$
と$d_{st}$の大小関係によって場合分けが行われている.次の補題は,それが別の等価な
条件によって判定できることを示している.

\begin{lemma}
$t \neq \alpha, \beta$ならば以下が成り立つ.
\begin{enumerate}
\item $\min\{d_{s\alpha}+d_{\beta t}, d_{s\beta}+d_{\alpha t}\}+1<d_{st}$で
あるための必要十分条件は$d_{s\alpha} < d_{s\beta}$と
$d_{s\alpha}+d_{\beta t}+1<d_{st}$がともに成り立つことである.
\item $\min\{d_{s\alpha}+d_{\beta t}, d_{s\beta}+d_{\alpha t}\}+1=d_{st}$
であるための必要十分条件は$d_{s\alpha} < d_{s\beta}$と 
$d_{s\alpha}+d_{\beta t}+1=d_{st}$がともに成り立つことである.
\item $\min\{d_{s\alpha}+d_{\beta t}, d_{s\beta}+d_{\alpha t}\}+1>d_{st}$
であるための必要十分条件は$d_{s\alpha}=d_{s\beta}$と$d_{s\alpha}+d_{\beta t}+1>d_{st}$の少なくとも一つが成り立つことである.
\end{enumerate}
\label{lemma:8b}
\end{lemma}
\begin{proof}
まず1番目の主張を証明する.
$\min\{d_{s\alpha}+d_{\beta s}, d_{s\beta}+d_{\alpha t}\}+1<d_{st}$ならば補題\ref{lemma:7}より$d_{s\alpha}<d_{s\beta}$と
$d_{s\alpha}+d_{\beta t}<d_{s\beta}+d_{\alpha t}$が成り立ち,さらに後者より
\[
d_{s\alpha}+d_{\beta t}+1=
 \min\{d_{s\alpha}+d_{\beta s}, d_{s\beta}+d_{\alpha t}\}+1<d_{st}
\]
が成り立つ.逆に$d_{s\alpha}+d_{\beta t}+1<d_{st}$ならば
\[
 \min\{d_{s\alpha}+d_{\beta s}, d_{s\beta}+d_{\alpha t}\}+1 \leq 
d_{s\alpha}+d_{\beta s}+1<d_{st}
\]
が成り立つ.次に2番目の主張を証明する.
$\min\{d_{s\alpha}+d_{\beta s}, d_{s\beta}+d_{\alpha t}\}+1=d_{st}$ならば補題\ref{lemma:7}より$d_{s\alpha}<d_{s\beta}$と
$d_{s\alpha}+d_{\beta t}<d_{s\beta}+d_{\alpha t}$が成り立ち,さらに後者より
\[
d_{s\alpha}+d_{\beta t}+1=
 \min\{d_{s\alpha}+d_{\beta s}, d_{s\beta}+d_{\alpha t}\}+1=d_{st}
\]
が成り立つ.逆に$d_{s\alpha}<d_{s\beta}$かつ$d_{s\alpha}+d_{\beta t}+1=d_{st}$ならば
\[
d_{s\beta}+d_{\alpha t}>d_{s\alpha}+d_{\alpha t} \geq d_{st}
=d_{s\alpha}+d_{\beta t}+1
\]
より
\[
 d_{s\alpha}+d_{\beta t} < d_{s\beta}+d_{\alpha t}
\]
が得られ,さらにこれより
\[
\min\{d_{s\alpha}+d_{\beta s}, d_{s\beta}+d_{\alpha t}\}+1=
d_{s\alpha}+d_{\beta s}+1=d_{st}
\]
が成り立つ.最後に3番目の主張を証明する.
$\min\{d_{s\alpha}+d_{\beta s}, d_{s\beta}+d_{\alpha t}\}+1>d_{st}$ならば
\[
 d_{s\alpha}+d_{\beta s}+1 \geq 
\min\{d_{s\alpha}+d_{\beta s}, d_{s\beta}+d_{\alpha t}\}+1>d_{st}
\]
が成り立つ.十分性については対偶を考えればよい.すなわち,
$\min\{d_{s\alpha}+d_{\beta s}, d_{s\beta}+d_{\alpha t}\}+1 \leq d_{st}$
ならば$d_{s\alpha}<d_{s\beta}$と$d_{s\alpha}+d_{\beta t}+1\leq d_{st}$の
両方が成り立つことを示せばよい.これは1番目と2番目の主張で既に示されている.
\end{proof}

補題~\ref{lemma:8}と補題~\ref{lemma:8b}より次の補題が得られる.
\begin{lemma}
$t \neq \alpha, \beta$ならば以下が成り立つ.
\begin{enumerate}
\item $d_{s\alpha}<d_{s\beta}$かつ$d_{s\alpha}+d_{\beta t}+1<d_{st}$
ならば$G'_{st}$は$G_{s\alpha}$と有向辺$(\alpha,\beta)$と$G_{\beta t}$からなり,$d'_{st}=d_{s\alpha}+1+d_{\beta t}$, $\sigma'_{st}=\sigma_{s\alpha}\sigma_{\beta t}$である.
\item $d_{s\alpha}<d_{s\beta}$かつ$d_{s\alpha}+d_{\beta t}+1=d_{st}$
ならば$G'_{st}$は$G_{st}$と$G_{s\alpha}$と有向辺$(\alpha,\beta)$と$G_{\beta t}$からなり,$d'_{st}=d_{st}$, $\sigma'_{st}=\sigma_{st}+\sigma_{s\alpha}\sigma_{\beta t}$である.
\item $d_{s\alpha}=d_{s\beta}$または$d_{s\alpha}+d_{\beta t}+1>d_{st}$ならば$G'_{st}=G_{st}$, $d'_{st}=d_{st}$, $\sigma'_{st}=\sigma_{st}$である.
\end{enumerate}
\label{lemma:9}
\end{lemma}

さらに,補題\ref{lemma:5}, \ref{lemma:6}, \ref{lemma:9}は次のように
一つにまとめられる.

\begin{theorem}
仮定\eqref{eqn:assum}の下で以下が成り立つ.
\begin{enumerate}
\item $d_{s\alpha}<d_{s\beta}$かつ$d_{s\alpha}+d_{\beta t}+1<d_{st}$
ならば$G'_{st}$は$G_{s\alpha}$と有向辺$(\alpha,\beta)$と$G_{\beta t}$からなり,$d'_{st}=d_{s\alpha}+1+d_{\beta t}$, $\sigma'_{st}=\sigma_{s\alpha}\sigma_{\beta t}$である.
\item $d_{s\alpha}<d_{s\beta}$かつ$d_{s\alpha}+d_{\beta t}+1=d_{st}$
ならば$G'_{st}$は$G_{st}$と$G_{s\alpha}$と有向辺$(\alpha,\beta)$と$G_{\beta t}$からなり,$d'_{st}=d_{st}$, $\sigma'_{st}=\sigma_{st}+\sigma_{s\alpha}\sigma_{\beta t}$である.
\item $d_{s\alpha}=d_{s\beta}$または$d_{s\alpha}+d_{\beta t}+1>d_{st}$ならば$G'_{st}=G_{st}$, $d'_{st}=d_{st}$, $\sigma'_{st}=\sigma_{st}$である.
\end{enumerate}
\label{theorem:1}
\end{theorem}

同様にして$\sigma'_{st}(i)$に関する次の結果が得られる.証明は紙数の都合上省略する.
\begin{theorem}
仮定\eqref{eqn:assum}の下で以下が成り立つ.
\begin{enumerate}
\item $d_{s\alpha}<d_{s\beta}$かつ$d_{s\alpha}+d_{\beta t}+1<d_{st}$ならば
$\sigma'_{st}(i)$は次式で与えられる.
\[
 \sigma'_{st}(i)=\left\{
\begin{array}{ll}
\sigma_{si}\sigma_{i\alpha}\sigma_{\beta t}, & i \in V_{s\alpha}\,\mbox{のとき} \\
\sigma_{s\alpha}\sigma_{\beta i}\sigma_{it}, & i \in V_{\beta t}\,\mbox{のとき} \\
0, & i \not\in V_{s\alpha} \cup V_{\beta t}\,\mbox{のとき}
\end{array}
\right.
\]
%
\item $d_{s\alpha}<d_{s\beta}$かつ$d_{s\alpha}+d_{\beta t}+1=d_{st}$ならば
$\sigma'_{st}(i)$は次式で与えられる.
\[
 \sigma'_{st}(i)=\left\{
\begin{array}{ll}
\sigma_{st}(i)+\sigma_{si}\sigma_{i\alpha}\sigma_{\beta t}, & i \in V_{s\alpha}\,\mbox{のとき} \\
\sigma_{st}(i)+\sigma_{s\alpha}\sigma_{\beta i}\sigma_{it}, & i \in V_{\beta t}\,\mbox{のとき} \\
\sigma_{st}(i), & i \not\in V_{s\alpha} \cup V_{\beta t}\,\mbox{のとき}
\end{array}
\right.
\]
%
\item $d_{s\alpha}=d_{s\beta}$または$d_{s\alpha}+d_{\beta t}+1>d_{st}$ならば
$\sigma'_{st}(i)=\sigma_{st}(i)$である.
\end{enumerate}
\label{theorem:2}
\end{theorem}

定理~\ref{theorem:2}において$\sigma'_{st}(i)$を求めるには
$\sigma_{st}(i)$の値や頂点$i$が$V_{s\alpha}$や$V_{\beta t}$に属するか
否かの判定が必要である.前者は\eqref{eqn:sigma_sti}
で与えられるので,$G$のすべての頂点間の最短経路の個数と長さから$\sigma'_{st}(i)$
を求めることができる.後者についても,$i \in V_{s\alpha}$, $i \in V_{\beta t}$で
あるための必要十分条件がそれぞれ$d_{si}+d_{i\alpha}=d_{s\alpha}$, 
$d_{\beta i}+d_{it}=d_{it}$
であるから,$G$のすべての頂点間の最短経路の長さを用いて判定することができる.

以上の議論を用いて,一辺挿入時の媒介中心性を更新するアルゴリズムをアルゴリズム\ref{algo:update-bc-on-insert}に示す.

\begin{algorithm}[H]
  \caption{一辺挿入時の媒介中心性更新アルゴリズム}
  \label{algo:update-bc-on-insert}
  \begin{algorithmic}[1]
    \Require グラフ$G=(V,E)$,挿入辺$\{\alpha,\beta\}$,最短経路長$\{d_{pq}\}_{p,q=1}^N$,最短経路数$\{\sigma_{pq}\}_{p,q=1}^N$,ペア依存度$\delta_s(i)$
    \Ensure 挿入後の最短経路長$\{d'_{pq}\}_{p,q=1}^N$,最短経路数$\{\sigma'_{pq}\}_{p,q=1}^N$,媒介中心性$B'_1,B'_2,\ldots,B'_N$,ペア依存度$\delta'_s(i)$
    \ForAll{$s\in\{1,2,\ldots,N\}$}
    \If{$d_{s\alpha}=d_{s\beta}$}
    \State $d'_{st}\gets d_{st}\qquad\sigma'_{st}\gets\sigma_{st}\qquad\delta'_s(i)\gets\delta_s(i)\ (i=1,2,\ldots,N)$
    \Else
    \State\Comment 最短経路数・最短経路長の更新
    \If{$d_{s\alpha}<d_{s\beta}$かつ$d_{st}>d_{s\alpha}+d_{\beta t}+1$}
    \State $d'_{st}\gets d_{s\alpha}+d_{\beta t}+1\qquad\sigma'_{st}\gets \sigma_{s\alpha}\sigma_{\beta t}$
    \ElsIf{$d_{s\alpha}<d_{s\beta}$かつ$d_{st}=d_{s\alpha}+d_{\beta t}+1$}
    \State $d'_{st}\gets d_{st}\qquad\sigma'_{st}\gets \sigma_{st}+\sigma_{s\alpha}\sigma_{\beta t}$
    \ElsIf{$d_{s\alpha}=d_{s\beta}$または$d_{st}<d_{s\alpha}+d_{\beta t}+1$}
    \State $d'_{st}\gets d_{st}\qquad\sigma'_{st}\gets\sigma_{st}$
    \EndIf
    \State\Comment ペア依存度の更新
    \State $\delta'_s(i)\gets0\ (i=1,2,\ldots,N)$
    \State 頂点$s$からすべての頂点への最短経路$G_s=(V,E_s)$と$d_{si},\ \sigma_{si}\ (i=1,2,\ldots,N)$を求める. 
    \ForAll{$(i,j)\in E_s$}
    \State \begin{equation*} \delta'_s(i)\gets\delta'_s(i)+\frac{\sigma_{si}}{\sigma_{sj}}(1+\delta'_s(j)) \end{equation*}
    \EndFor
    \EndIf
    \EndFor
    \State $B'_i\gets\sum_{s\in\{1,2,\ldots,N\}}\delta'_s(i)$
    \State \textbf{return} $B'_i\ (i=1,2,\ldots,N)$
  \end{algorithmic}
\end{algorithm}

\section{一辺削除時の媒介中心性の変化量}
\label{sect:update-bc-on-delete}
本節では,$G=(V,E)$に辺$e=\{\alpha,\beta\}\in E$を削除して得られるグラフを
$G'=(V,E')$とする.$G'$における頂点$s$から頂点$t\,(\neq s)$への最短経路を有向グラフ
$G'_{st}=(V'_{st},E'_{st})$で表し,頂点$s$から他のすべての頂点への最短経
路を有向グラフ$G'_s=(V,E'_s)$で表す.また,$G'$における頂点$s$から
頂点$t\,(\neq s)$への最短経路の個数と長さをそれぞれ$\sigma'_{st}$, 
$d'_{st}$で表し,$\sigma'_{st}$個の最短経路のうち頂点$i$を通るもの
の個数を$\sigma'_{st}(i)$で表し,$\sigma'_{st}$個の最短経路のうち有効辺$(\alpha,\beta)$を通るものの個数を$\sigma'_{st}(\alpha,\beta)$で表す.

まず,更新時の距離に関するいくつかの補題を示す.
\begin{lemma}
  \label{lemma:existence-of-vertex}
  $G'$において,異なる二頂点$s,t\in V'$が同じ連結成分に含まれるとき,$\sigma_{si}>\sigma_{si}(\alpha,\beta)$かつ$\sigma_{it}>\sigma_{it}(\alpha,\beta)$なる$i\in V_{st}'$が存在する.
\end{lemma}
\begin{proof}
  背理法で証明する.$\sigma_{si}>\sigma_{si}(\alpha,\beta)$かつ$\sigma_{it}>\sigma_{it}(\alpha,\beta)$である$i\in V_{st}'$が存在しない,すなわち,すべての$i\in V_{st}'$について$\sigma_{si}=\sigma_{si}(\alpha,\beta)$または$\sigma_{it}=\sigma_{it}(\alpha,\beta)$ならば,$G$において,$s$と$t$のすべての最短経路に$(\alpha,\beta)$を含む.すると,$G'$において,$s$と$t$が異なる連結成分に含まれることとなり,仮定と矛盾する.
\end{proof}

\begin{lemma}
  \label{lemma:update-distance-on-delete}
  頂点$s$と$t$が同じ連結成分に含まれるとき,次が成り立つ.
  \begin{equation}
    d'_{st}=\min\{d_{si}+d_{it}|i\in V,\sigma_{si}>\sigma_{si}(\alpha,\beta),\sigma_{it}>\sigma_{it}(\alpha,\beta)\}
    \label{eq:update-distance-on-delete}
  \end{equation}
\end{lemma}
\begin{proof}
  補題\ref{lemma:distance-of-path}より,
  \begin{equation}
    d'_{st}=\min\{d'_{si}+d'_{it}|i\in V'\}
    \label{eq:distance-of-path-after-delete}
  \end{equation}
  が成り立つ.ここで,
  \begin{equation*}
    d'_{si}+d'_{it}=
    \begin{cases}
      d_{si}+d_{it} & \sigma_{si}>\sigma_{si}(\alpha,\beta)\textrm{かつ}\sigma_{it}>\sigma_{it}(\alpha,\beta)のとき \\
      d'_{si}+d_{it} & \sigma_{si}=\sigma_{si}(\alpha,\beta)\textrm{かつ}\sigma_{it}>\sigma_{it}(\alpha,\beta)のとき \\
      d_{si}+d'_{it} & \sigma_{si}>\sigma_{si}(\alpha,\beta)\textrm{かつ}\sigma_{it}=\sigma_{it}(\alpha,\beta)のとき
    \end{cases}
  \end{equation*}
  であることと,$d'_{ij}\geq d_{ij}$であることと補題\ref{lemma:existence-of-vertex}から,式\eqref{eq:distance-of-path-after-delete}は,式\eqref{eq:update-distance-on-delete}に変形できる.
\end{proof}

次の二つの定理は,一辺削除後の最短経路長と最短経路数の変化量に関するものである.
\begin{theorem}
  \label{thm:update-distance-on-delete}
  $d_{s\alpha}\leq d_{s\beta}$のとき,$d'_{st}$について,次が成り立つ.
  \begin{equation*}
    d'_{st}=
    \begin{cases}
      d_{st} & \sigma_{st}(\alpha,\beta)<\sigma_{st} のとき \\
      \min\{d_{sv}+d_{vt}\,|\,v\in V,\,\sigma_{sv}>\sigma_{sv}(\alpha,\beta),\,\sigma_{vt}>\sigma_{vt}(\alpha,\beta) & \sigma_{st}(\alpha,\beta)=\sigma_{st} のとき
    \end{cases}
  \end{equation*}
\end{theorem}
\begin{proof}
  $\sigma_{st}(\alpha,\beta)<\sigma_{st}$ならば,長さが$d_{st}$である$s$と$t$の最短経路が存在するので,$d'_{st}=d_{st}$である.  $\sigma_{st}(\alpha,\beta)=\sigma_{st}$ならば,$s$と$t$のすべての最短経路が$(\alpha,\beta)$を含むので,補題\ref{lemma:update-distance-on-delete}を用いて再計算する.
\end{proof}
\begin{theorem}
  \label{thm:update-pathnum-on-delete}
  $d_{s\alpha}\leq d_{s\beta}$のとき,$\sigma'_{st}$について,次が成り立つ.
  \begin{equation*}
    \sigma'_{st}=
    \begin{cases}
      \sigma_{st} & \sigma_{st}(\alpha,\beta)=0 のとき \\
      \sigma_{st}-\sigma_{s\alpha}\sigma_{\beta t} & 0<\sigma_{st}(\alpha,\beta)<\sigma_{st} のとき \\
      \frac{\sum_v(\sigma'_{sv}\sigma'_{vt})}{d'_{st}-1}\:(v\in V,\,v\neq s,t,\,d'_{st}=d'_{sv}+d'_{vt}) & \sigma_{st}(\alpha,\beta)=\sigma_{st} のとき
    \end{cases}
  \end{equation*}
\end{theorem}
\begin{proof}
  $\sigma_{st}(\alpha,\beta)=0$のとき,$(\alpha,\beta)\notin E'_{st}$なので,$\sigma'_{st}=\sigma_{st}$が成り立つ.
  $0<\sigma_{st}(\alpha,\beta)<\sigma_{st}$のとき,削除により$\sigma_{s\alpha}\sigma_{\beta t}$個の最短経路がなくなるため,$\sigma'_{st}=\sigma_{st}-\sigma_{s\alpha}\sigma_{\beta t}$が成り立つ.
  $\sigma_{st}=\sigma_{st}(\alpha,\beta)$のとき,$s$と$t$のすべての最短経路が$(\alpha,\beta)$を含むため,補題\ref{lemma:number-of-paths}により再計算する.
\end{proof}

定理\ref{thm:update-distance-on-delete}と定理\ref{thm:update-pathnum-on-delete}より,一辺削除時の頂点間距離の更新の方法が分かる.
この方法に基づく一辺削除時の媒介中心性を更新するアルゴリズムをアルゴリズム\ref{algo:update-bc-on-delete}に示す.

\begin{algorithm}[H]
  \caption{一辺削除時の媒介中心性更新アルゴリズム}
  \label{algo:update-bc-on-delete}
  \begin{algorithmic}[1]
    \Require グラフ$G=(V,E)$,削除辺$\{\alpha,\beta\}$,最短経路長$\{d_{pq}\}_{p,q=1}^N$,最短経路数$\{\sigma_{pq}\}_{p,q=1}^N$,ペア依存度$\delta_s(i)$
    \Ensure 削除後の最短経路長$\{d'_{pq}\}_{p,q=1}^N$,最短経路数$\{\sigma'_{pq}\}_{p,q=1}^N$,媒介中心性$B'_1,B'_2,\ldots,B'_N$,ペア依存度$\delta'_s(i)$
    \ForAll{$s\in\{1,2,\ldots,N\}$}
    \If{$d_{s\alpha}=d_{s\beta}$}
    \State $d'_{st}\gets d_{st}\qquad\sigma'_{st}\gets\sigma_{st}\qquad\delta'_s(i)\gets\delta_s(i)\ (i=1,2,\ldots,N)$
    \Else
    \State\Comment 最短経路数・最短経路長の更新
    \State $P\gets()$
    \Comment{更新の対象となる頂点組${st}$}
    \ForAll{$t\in\{1,2,\ldots,N\}$}
    \If{$\sigma_{st}(\alpha,\beta)=0$}
    \State $d'_{st}=d_{st}$
    \State $\sigma'_{st}=\sigma_{st}$
    \ElsIf{$0<\sigma_{st}(\alpha,\beta)<\sigma_{st}$}
    \State $d'_{st}=d_{st}$
    \State $\sigma'_{st}=\sigma_{st}-\sigma_{st}(\alpha,\beta)$
    \ElsIf{$\sigma_{st}(\alpha,\beta)=\sigma_{st}>0$}
    \Comment{頂点間距離を再計算}
    \State $d_{\min}\gets \infty$
    \ForAll{$v\in V,\,\sigma_{sv}>\sigma_{sv}(\alpha,\beta),\,\sigma_{vt}>\sigma_{vt}(\alpha,\beta)$}
    \State $d_{\min}\gets\min\left\{d_{\min},d_{sv}+d_{vt}\right\}$
    \EndFor
    \If{$d_{\min}=\infty$}
    \State $d'_{st}\gets\infty$
    \State $\sigma'_{st}\gets0$
    \Else
    \State $d'_{st}\gets d_{\min}$
    \State \parbox[t]{\linewidth}{
      $P\gets(\ldots,(s_i,t_i),(s,t),(s_{i+1},t_{i+1}),\ldots)$ \\
      ただし,$d'_{s_i,t_i}\leq d'_{st}\leq d'_{s_{i+1},t_{i+1}},\ \ldots,(s_i,t_i),\ldots\in P$
    }
    \EndIf
    \EndIf
    \EndFor
    \ForAll{$(s_i,t_i)\in P$}
    \Comment{最短経路数を再計算}
    \State $\sigma'\gets0$
    \ForAll{$v\in V,\:v\neq s_i,t_i,\:d'_{s_i,t_i}=d'_{s_i,v}+d'_{v,t_i}$}
    \State $\sigma'\gets\sigma'+\sigma'_{s_i,v}\sigma'_{v,t_i}$
    \EndFor
    \State $\sigma'_{s_i,t_i}\gets\sigma'/(d'_{s_i,t_i}-1)$
    \EndFor
    \State\Comment ペア依存度の更新
    \State $\delta'_s(i)\gets0\ (i=1,2,\ldots,N)$
    \State 頂点$s$からすべての頂点への最短経路$G_s=(V,E_s)$と$d_{si},\ \sigma_{si}\ (i=1,2,\ldots,N)$を求める. 
    \ForAll{$(i,j)\in E_s$}
    \State \begin{equation*} \delta'_s(i)\gets\delta'_s(i)+\frac{\sigma_{si}}{\sigma_{sj}}(1+\delta'_s(j)) \end{equation*}
    \EndFor
    \EndIf
    \EndFor
    \State $B'_i\gets\sum_{s\in\{1,2,\ldots,N\}}\delta'_s(i)$
    \State \textbf{return} $B'_i\ (i=1,2,\ldots,N)$
  \end{algorithmic}
\end{algorithm}

