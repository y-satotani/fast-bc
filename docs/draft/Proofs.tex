\chapter{最短経路の長さと個数の証明}
最短経路の長さと個数について,いくつかの補題を示す.

\begin{lemma}[Brandes~\cite{Brandes2001}]
$G=(V,E)$の異なる2頂点$s,t \in V$に対して,$s$から$t$への最短経路$G_{st}=(V_{st},E_{st})$が$i \in V$を含む,すなわち$i \in V_{st}$であるための必要十分条件は
\begin{equation}
  d_{st}=d_{si}+d_{it}
  \label{eqn:inclusion0}
\end{equation}
が成り立つことである.
\label{lemma:1}
\end{lemma}
\begin{proof}
はじめに$i=s$の場合を考える.このとき明らかに$i \in V_{st}$であり,かつ$d_{si}+d_{it}=d_{ss}+d_{st}=d_{st}$より\eqref{eqn:inclusion0}もつねに成り立つ.$i=t$の場合も同様である.そこで以下では$i \not\in \{s,t\}$と仮定する.$G_{st}$が頂点$i$を含むならば,$G_{st}$の中に$s \rightarrow \cdots \rightarrow i \rightarrow \cdots \rightarrow t$の順に頂点を通る有向道が存在する.この有向道の長さは$d_{si}+d_{it}$で与えられるので\eqref{eqn:inclusion0}が成り立つ.次に\eqref{eqn:inclusion0}が成り立つと仮定する.頂点$s$を出発して$s$から$i$への最短経路の一つを通って$i$に行き,次に$i$から$t$への最短経路の一つを通って$t$に行く有向道を考える.この有向道は$s$から$t$への最短経路の一つである.なぜなら,その長さは$d_{si}+d_{it}$で与えられ,\eqref{eqn:inclusion0}より$s$から$t$への最短経路長$d_{st}$に等しいからである.したがって$i \in V_{st}$が成り立つ.
\end{proof}

\begin{lemma}
$G=(V,E)$の異なる2頂点$s,t\in V$と辺$\{i,j\} \in E$を考える.$s$から$t$への最短経路$G_{st}=(V_{st},E_{st})$が有向辺$(i,j)$を含む,すなわち$(i,j) \in E_{st}$が成り立つための必要十分条件は
\begin{equation}
d_{st}=d_{si}+d_{jt}+1
\label{eqn:inclusion}
\end{equation}
が成り立つことである.
\label{lemma:2}
\end{lemma}
\begin{proof}
はじめに$i=s$, $j=t$の場合を考える.このとき明らかに$G_{st}$は
有向辺$(i,j)$を含み,かつ$d_{si}+d_{jt}+1=d_{ss}+d_{tt}+1=1=d_{st}$
より\eqref{eqn:inclusion}も成り立つ.次に$i=s$, $j \neq t$の場合を
考える.このとき$(i,j)\in E_{st}$であるための必要十分条件は$G_{st}$が
$j$を含むことである.これは補題~\ref{lemma:1}より
$d_{st}=d_{sj}+d_{jt}$と等価であり,さらに右辺$d_{sj}+d_{jt}$は
\eqref{eqn:inclusion}の右辺と等しい.$i\neq s$, $j=t$の場合も同様である.
そこで以下では$s,t,i,j$がすべて異なると仮定する.
$G_{st}$が有向辺$(i,j)$を含むならば,$G_{st}$の中に
$s \rightarrow \cdots \rightarrow i \rightarrow j \rightarrow \cdots \rightarrow t$
の順に頂点を通る有向道が存在する.
この有向道の長さは$d_{si}+1+d_{jt}$で与えられるので\eqref{eqn:inclusion}が
成り立つ.次に\eqref{eqn:inclusion}が成り立つと仮定する.頂点
$s$を出発して$s$から$i$への最短経路の一つを通って$i$に行き,次に
辺$\{i,j\}$を通って$j$に行き,最後に$j$から$t$への最短経路の一つを
通って$t$に行く有向道が存在する.この有向道は$s$から$t$への最短経路の一つで
ある.なぜなら,その長さは$d_{si}+1+d_{jt}$で与えられ,
\eqref{eqn:inclusion}より$s$から$t$への最短経路長$d_{st}$に
等しいからである.したがって,$(i,j) \in E_{st}$が成り立つ.
\end{proof}

\begin{lemma}[Brandes\cite{Brandes2001}]
$G=(V,E)$の異なる2頂点$s,t \in V$に対して,
$s$から$t$への最短経路の中で$i$を通るものの個数$\sigma_{st}(i)$は
次式で与えられる.
\begin{equation}
\sigma_{st}(i)=
\left\{
\begin{array}{ll}
\sigma_{si} \sigma_{it}, & d_{st}=d_{si}+d_{it}\,\mbox{のとき} \\
0, & \mbox{それ以外のとき}
\end{array}
\right.
\label{eqn:sigma_sti}
\end{equation}
\label{lemma:3}
\end{lemma}

\begin{lemma}
$G=(V,E)$の異なる2頂点$s,t \in V$に対して,
$s$から$t$への最短経路の中で有向辺$(i,j)$を通るものの個数を
$\sigma_{st}(i,j)$とおくと,それは次式で与えられる.
\begin{equation*}
\sigma_{st}(i,j)=
\left\{
\begin{array}{ll}
\sigma_{si} \sigma_{jt}, & d_{st}=d_{si}+d_{jt}+1\,\mbox{のとき} \\
0, & \mbox{それ以外のとき}
\end{array}
\right.
%\label{eqn:sigma_stij}
\end{equation*}
\label{lemma:4}
\end{lemma}

次の補題は,ある二頂点間の最短経路の長さと,その経路に含まれる短い最短経路の長さに関するものである.
\begin{lemma}
  \label{lemma:distance-of-path}
  $G=(V,E)$の異なる二頂点$s,t\in V$について,次が成り立つ.
  \begin{equation*}
    d_{st}=\min\{d_{si}+d_{it}|i\in V\}
  \end{equation*}
\end{lemma}
\begin{proof}
  TBA
\end{proof}

次の補題は,ある二頂点間の最短経路の個数と,その経路に含まれる
短い最短経路の個数との関係を示す.
\begin{lemma}
  \label{lemma:number-of-paths}
  $G=(V,E)$の異なる二頂点$s,t\in V$について,$v$を$d_{st}=d_{sv}+d_{vt}$
  である頂点(ただし$v\neq s,t$)とすると,次が成り立つ.
  \begin{equation}
    \label{eq:number-of-paths}
    \sigma_{st}=\frac{\sum_{v}\sigma_{sv}\sigma_{vt}}{d_{st}-1}
  \end{equation}
\end{lemma}
\begin{proof}
  $s$と$t$の間の一般的な経路を図\ref{fig:proof-number-of-paths}に示す.
  \begin{figure}
    \centering
    \def\svgwidth{.5\columnwidth}
    \input{proof-number-of-paths.pdf_tex}
    \caption{$s$と$t$の一般的な最短経路}
    \label{fig:proof-number-of-paths}
  \end{figure}
  $s$からの距離が一定の頂点を並べて,一つの層とする.$d_{sv}=k$なる頂点
  $v$の集合を,第$k$層と定義し,$L_k$と表す.
  $L_k$に属する頂点の数を$n_k$,$L_k$に属する$l$番目の頂点を$v_{kl}$と表す.
  ここで,第$k$層に属する頂点$v$は,隣接する層(第$k-1$層と第$k+1$層)
  以外の層に属する頂点$w$と隣接しないことに注意する.
  もしそのような頂点が存在すると,最短経路長が変化する.
  式\eqref{eq:number-of-paths}の両辺に$d_{st}-1$を掛けて,
  次の式\eqref{eq:number-of-paths1}を得る.
  \begin{equation}
    \sigma_{st}(d_{st}-1)=\sum_{v}\sigma_{sv}\sigma_{vt}
    \label{eq:number-of-paths1}
  \end{equation}
  式\eqref{eq:number-of-paths1}の右辺を,
  図\ref{fig:proof-number-of-paths}にならって表すと,
  \begin{equation}
    \sum_{v}\sigma_{sv}\sigma_{vt}=
    \sum_{k=1}^m\sum_{l=1}^{n_k}\sigma_{sv_{kl}}\sigma_{v_{kl}t}
    \label{eq:number-of-paths2}
  \end{equation}
  が得られる.ここで,二つの頂点$v$と$w$について,次の隣接を表す記号$a$を導入する.
  \begin{align*}
    a_{vw}=
    \begin{cases}
      1 & vとwが隣接しているとき \\
      0 & vとwが隣接していないとき
    \end{cases}
  \end{align*}
  各々の$\sigma_{sv_{kl}}\sigma_{v_{kl}t}$について議論する.
  $a$の定義を用いて式を変形すると,
  \begin{align}
    &\sigma_{sv_{kl}}\sigma_{v_{kl}t}\nonumber\\
    =&\left(\sum_{v'\in L_{k-1}}\sigma_{sv'}a_{v'v_{kl}}\right)
    \left(\sum_{v'\in L_{k+1}}\sigma_{v_{kl}v'}a_{v't}\right)
    \nonumber\\
    =&\left(\sum_{v''\in L_{k-2}}\sum_{v'\in L_{k-1}}
    \sigma_{sv''}a_{v''v'}a_{v'v_{kl}}\right)
    \left(\sum_{v'\in L_{k+1}}\sum_{v''\in L_{k+2}}
    a_{v_{kl}v'}a_{v'v''}\sigma_{v''t}\right)
    \nonumber\\
    &\vdots\nonumber\\
    =&\left(\sum_{(v_1,\ldots,v_{k-1})\in L_1\times\cdots\times L_{k-1}}
    a_{sv_1}\cdots a_{v_{k-1}v_{kl}}\right)
    \left(\sum_{(v_{k+1},\ldots,v_m)\in L_{k+1}\times\cdots\times L_m}
    a_{v_{kl}v_{k+1}}\cdots a_{v_mv_t}\right)\nonumber\\
    =&\sum_{(v_1,\ldots,v_{k-1},v_{k+1},\ldots,v_m)\in L_1\times\cdots\times L_{k-1}\times L_{k+1}\times\cdots\times L_m}
    a_{sv_1}\cdots a_{v_{k-1}v_{kl}}a_{v_{kl}v_{k+1}}\cdots a_{v_mt}
    \label{eq:number-of-paths3}
  \end{align}
  が得られる.式\eqref{eq:number-of-paths3}を式\eqref{eq:number-of-paths2}に
  代入すると,
  \begin{align}
    &\sum_{k=1}^m\sum_{l=1}^{n_k}\sigma_{sv_{kl}}\sigma_{v_{kl}t}\nonumber\\
    =&\sum_{k=1}^m\sum_{l=1}^{n_k}\sum_{
      (v_1,\ldots,v_{k-1},v_{k+1},\ldots,v_m)\in
      L_1\times\cdots\times L_{k-1}\times L_{k+1}\times\cdots\times L_m
    }a_{sv_1}\cdots a_{v_{k-1}v_{kl}}a_{v_{kl}v_{k+1}}\cdots a_{v_mt}\nonumber\\
    =&\sum_{k=1}^m\sum_{(v_1,\ldots,v_m)\in L_1\times\cdots\times L_m}
    a_{sv_1}\cdots a_{v_mt}\nonumber\\
    =&m\left(\sum_{(v_1,\ldots,v_m)\in L_1\times\cdots\times L_m}
    a_{sv_1}\cdots a_{v_mt}\right)
    \label{eq:number-of-paths4}
  \end{align}
  と変形できる.式\eqref{eq:number-of-paths4}の総和の対象が$1$となるのは,
  $a_{sv_1},\ldots,a_{v_mt}$のすべてが$1$のとき,
  すなわち,$s$と$v_1$,$v_1$と$v_2$,$\ldots$,$v_m$と$t$がすべて隣接している
  とき,すなわち,$s$と$t$の最短経路となっているときである.
  従って,総和の値は$s$と$t$の最短経路の数と一致し,
  式\eqref{eq:number-of-paths4}は$\sigma_{st}(d_{st}-1)$と等しい.
  従って,補題が成り立つ.
\end{proof}

