\chapter{計算量の解析}
\label{chap:complexity-analysis}

本章では,前の章で示したアルゴリズムの計算量を与える.
まず,以下の議論のためにいくつかの記号を導入しておく.
$\tau(s)$および$\tau'(s)$は,それぞれ$T(s)$と$\Delta_{s\bullet}(x)>0$
および$\Delta'_{s\bullet}(x)>0$を満たす頂点の和集合である.つまり,
$\tau(s)=T(s)\cup\{x|\Delta_{s\bullet}(x)>0\}$,
$\tau'(s)=T(s)\cup\{x|\Delta'_{s\bullet}(x)>0\}$とする.
また,頂点集合$V$について$\left\vert V\right\vert$を$V$の要素数,
$\left\Vert V\right\Vert$を$V$の各頂点と接続する辺の数とする.

次の定理は,最短経路の更新の前後に行われる媒介中心性の更新の計算量を示す.
まず,古い最短経路に関する依存度を減算するアルゴリズムの計算量を示す.

\begin{theorem}[Bergamini et.al.\cite{Bergamini2017}]
  \label{thm:decrease-betweenness-weight-complexity}
  アルゴリズム\ref{alg:decrease-betweenness}の\textproc{DecreaseBetweenness}の,
  重み付きグラフと重みなしグラフに対する計算量は,それぞれ次式で与えられる.
  \begin{align}
    &\mathcal{O}(\sum_{s\in S(v)}\left(\left\|\tau(s)\right\|+\left|\tau(s)\right|\log\left|\tau(s)\right|\right))
    \label{eq:decrease-betweenness-weighted-complexity} \\
    &\mathcal{O}(\sum_{s\in S(v)}\left\|\tau(s)\right\|)
    \label{eq:decrease-betweenness-unweighted-complexity}
  \end{align}
\end{theorem}
\begin{proof}
  まず,アルゴリズム\ref{alg:decrease-betweenness}の8行目から27行目の部分に相当する,
  与えられた$s$に対する$\Delta_{s\bullet}(x)$を求めるアルゴリズムの計算量を示す.
  優先度付きキューとして,重み付きグラフでは二分ヒープを,重みなしグラフでは
  基数ヒープを用いると仮定する.
  すると,12行目の取り出しの操作はそれぞれ対数時間と定数時間で実行でき,
  14行目から26行目で取り出された要素の近傍を走査する.
  残りは優先度付きキューから取り出された要素の数が$|\tau(s)|$と一致することを示す.
  10行目で\texttt{PQ}には$|T(s)|$個の要素が入っている.
  また,24行で\texttt{PQ}に追加される要素は,$s$と$T(s)$のひとつの頂点との間の最短経路の上にある.
  これは,少なくとも一つの$t\in T(s)$が,$\sigma_{st}(y)>0$であることを示し,
  $\Delta_{s\bullet}(y)>0$が成り立つ.したがって,\texttt{PQ}に追加される要素は
  $T(s)\cup\{x|\Delta_{s\bullet}(x)>0\}=\tau(s)$である.
  したがって,アルゴリズムの8行目から27行目の部分に相当する部分の計算量は,
  重みグラフに対して$\mathcal{O}(\|\tau(s)\|+|\tau(s)|\log|\tau(s)|)$
  で,重みなしグラフに対して$\mathcal{O}(\|\tau(s)\|)$である.

  $s\in S(v)$についての総和をとって,\textproc{DecreaseBetweenness}全体の計算量を求めると,
  それぞれ式\eqref{eq:decrease-betweenness-weighted-complexity}と
  式\eqref{eq:decrease-betweenness-unweighted-complexity}になる.
\end{proof}

次に,最短経路の更新の後に行う,新たに追加された最短経路に関する各頂点の依存度を求めるアルゴリズムの
計算量を求める.

\begin{theorem}[Bergamini et.al.\cite{Bergamini2017}]
  \label{thm:increase-betweenness-weight-complexity}
  アルゴリズム\ref{alg:decrease-betweenness}の\textproc{IncreaseBetweenness}の
  重み付きグラフと重みなしグラフに対する計算量は,それぞれ次式で与えられる.
  \begin{align}
    &\mathcal{O}(\sum_{s\in S(v)}(\|\tau'(s)\|+|\tau'(s)|\log|\tau'(s)|))
    \label{eq:increase-betweenness-weighted-complexity} \\
    &\mathcal{O}(\sum_{s\in S(v)}\|\tau'(s)\|)
    \label{eq:increase-betweenness-unweighted-complexity}
  \end{align}
\end{theorem}
\begin{proof}
  定理\ref{thm:decrease-betweenness-weight-complexity}と同じ要領で示すことができる.
  まず,アルゴリズム\ref{alg:decrease-betweenness}の8行目から27行目の部分に相当する,
  与えられた$s$に対する$\Delta'_{s\bullet}(x)$を求めるアルゴリズムの計算量を示す.

  12行目の取り出しの操作はそれぞれ対数時間と定数時間で実行でき,
  14行目から26行目で取り出された要素の近傍を走査する.
  残りは優先度付きキューから取り出された要素の数が$|\tau'(s)|$と一致することを示す.
  10行目で\texttt{PQ}には$|T(s)|$個の要素が入っている.
  また,24行で\texttt{PQ}に追加される要素は,$s$と$T(s)$のひとつの頂点との間の最短経路の上にある.
  これは,少なくとも一つの$t\in T(s)$が,$\sigma'_{st}(y)>0$であることを示し,
  $\Delta'_{s\bullet}(y)>0$が成り立つ.したがって,\texttt{PQ}に追加される要素は
  $T(s)\cup\{x|\Delta'_{s\bullet}(x)>0\}=\tau'(s)$である.
  したがって,アルゴリズムの8行目から27行目の部分に相当する部分の計算量は,
  重みグラフに対して$\mathcal{O}(\|\tau'(s)\|+|\tau'(s)|\log|\tau'(s)|)$
  で,重みなしグラフに対して$\mathcal{O}(\|\tau'(s)\|)$である.

  $s\in S(v)$についての総和をとって,\textproc{IncreaseBetweenness}全体の計算量を求めると,
  それぞれ式\eqref{eq:increase-betweenness-weighted-complexity}と
  式\eqref{eq:increase-betweenness-unweighted-complexity}になる.

\end{proof}

次の定理は,最短経路更新アルゴリズムの計算量を示す.
まず,辺挿入時の最短経路変更アルゴリズムの計算量を示す.

\begin{theorem}
  \label{thm:incremental-shortest-path-update}
  アルゴリズム\ref{alg:incremental-algorithm}の\textproc{UpdateShortestPath}の
  重み付きグラフと重みなしグラフに対する計算量はそれぞれ次式で与えられる.
  \begin{align}
    &\mathcal{O}(\sum_{s\in S(v)}(\|T(s)\|\log|T(s)|)+\sum_{t\in T(u)}(\|S(t)\|\log|S(t)|))
    \label{eq:incremental-shortest-path-update-weighted} \\
    &\mathcal{O}(\sum_{s\in S(v)}\|T(s)\|+\sum_{t\in T(u)}\|S(t)\|)
    \label{eq:incremental-shortest-path-update-unweighted}
  \end{align}
\end{theorem}
\begin{proof}
  式\eqref{eq:incremental-shortest-path-update-weighted}のみについて示す.
  式\eqref{eq:incremental-shortest-path-update-unweighted}については,
  式\eqref{eq:incremental-shortest-path-update-weighted}と同じ要領で示すことができる.

  まず,アルゴリズム\ref{alg:incremental-algorithm}の\textproc{UpdatePart}に,
  ある$t$が与えられたときの計算量を導出する.
  16行目から26行目のループで,20行目および21行目で優先度つきキューから取り出された
  要素の近傍を走査している.さらに,23行で優先度付きキューに要素を追加している.
  そして22行目の条件から,このループは$s\in S(t)$に対して繰り返される.
  以上から,アルゴリズム\ref{alg:incremental-algorithm}の\textproc{UpdatePart}に,
  ある$t$が与えられたときの計算量は,
  \begin{equation*}
    \begin{aligned}
      \mathcal{O}(\sum_{s\in S(t)}(|(\mathcal{S}(s)|+|\mathcal{P}(s)|\log|S(t)|))
      &=\mathcal{O}(\|S(t)\|+\|S(t)\|\log|S(t)|) \\
      &=\mathcal{O}(\|S(t)\|\log|S(t)|)
    \end{aligned}
  \end{equation*}

  \textproc{UpdateShortestPath}は,$S(v)$と$T(u)$を求め,\textproc{UpdatePart}を
  $s\in S(v)$と$t\in T(u)$を対象として実行する.
  したがって,\textproc{UpdateShortestPath}全体の計算量は
  \textproc{UpdatePart}に渡すグラフの向きに注意して,
  \begin{equation*}
    \begin{aligned}
      &\mathcal{O}(\|S(v)\|+\|T(u)\|+\sum_{s\in S(v)}(\|T(s)\|\log|T(s)|)+\sum_{t\in T(u)}(\|S(t)\|\log|S(t)|)) \\
      &\:=\mathcal{O}(\sum_{s\in S(v)}(\|T(s)\|\log|T(s)|)+\sum_{t\in T(u)}(\|S(t)\|\log|S(t)|)).
    \end{aligned}
  \end{equation*}
\end{proof}

次に,辺挿削除時の最短経路変更アルゴリズムの計算量を示す.

\begin{theorem}
  \label{thm:decremental-shortest-path-update}
  アルゴリズム\ref{alg:decremental-algorithm}の\textproc{UpdateShortestPath}の
  重み付きグラフと重みなしグラフに対する計算量はそれぞれ次式で与えられる.
  \begin{align}
    &\mathcal{O}(\sum_{s\in S(v)}(\|T(s)\|\log|T(s)|)+\sum_{t\in T(u)}(\|S(t)\|\log|S(t)|))
    \label{eq:decremental-shortest-path-update-weighted} \\
    &\mathcal{O}(\sum_{s\in S(v)}\|T(s)\|+\sum_{t\in T(u)}\|S(t)\|)
    \label{eq:decremental-shortest-path-update-unweighted}
  \end{align}
\end{theorem}
\begin{proof}
  式\eqref{eq:decremental-shortest-path-update-weighted}のみについて示す.
  式\eqref{eq:decremental-shortest-path-update-unweighted}については,
  式\eqref{eq:decremental-shortest-path-update-weighted}と同じ要領で示すことができる.

  まず,アルゴリズム\ref{alg:decremental-algorithm}の\textproc{UpdatePart}に,
  ある$t$が与えられたときの計算量を導出する.

  $s\in S(t)$を16行目の条件に従って二つに分ける.つまり,
  $\exists y|y\in\mathcal{S}(s)\land y\notin S(t)$と
  $\forall y\in\mathcal{S}(s),\,y\in S(t)$に分ける.
  16行目を満たす頂点$s\in S(t)$に対して,実行される処理は以下である.
  16行目および17行目で$s$の後継を走査し,18行目で優先度付きキューに追加している.
  また,25行目で$\sigma_{st}$を更新するために後継を走査して,
  26行目から30行目のループで,$s$の先行を走査し,優先度付きキューに追加している.
  これらから,16行目を満たす頂点$s\in S(t)$に対する\textproc{UpdatePart}
  の計算量は,
  $\mathcal{O}(|\mathcal{S}(s)|+|\mathcal{P}(s)|+|\mathcal{P}(s)|\log|S(t)|)$
  である.

  16行目を満たさない頂点$s\in S(t)$に対して実行される処理は,
  16行目を満たす頂点に対して行われる実行から,18行目の優先度付きキュー
  に追加する処理を除いたものである.
  よって,16行目を満たさない頂点$s\in S(t)$に対して実行される処理の計算量は,
  $\mathcal{O}(|\mathcal{S}(s)|+|\mathcal{P}(s)|+|\mathcal{P}(s)|\log|S(t)|)$
  である.したがって,\textproc{UpdatePart}にある$t$が与えられたときの計算量は,
  \begin{equation*}
    \begin{aligned}
      \mathcal{O}(\sum_{s\in S(t)}(|\mathcal{S}(s)|+|\mathcal{P}(s)|+|\mathcal{P}(s)|\log|S(t)|)
      &=\mathcal{O}(\|S(t)\|+\|S(t)\|\log|S(t)|) \\
      &=\mathcal{O}(\|S(t)\|\log|S(t)|)
    \end{aligned}
  \end{equation*}

  挿入時と同様に,\textproc{UpdateShortestPath}は,$S(v)$と$T(u)$を求め,
  \textproc{UpdatePart}を$s\in S(v)$と$t\in T(u)$を対象として実行する.
  したがって,\textproc{UpdateShortestPath}全体の計算量は
  \textproc{UpdatePart}に渡すグラフの向きに注意して,
  \begin{equation*}
    \begin{aligned}
      &\mathcal{O}(\|S(v)\|+\|T(u)\|+\sum_{s\in S(v)}(\|T(s)\|\log|T(s)|)+\sum_{t\in T(u)}(\|S(t)\|\log|S(t)|)) \\
      &\:=\mathcal{O}(\sum_{s\in S(v)}(\|T(s)\|\log|T(s)|)+\sum_{t\in T(u)}(\|S(t)\|\log|S(t)|)).
    \end{aligned}
  \end{equation*}
\end{proof}

重み付きグラフ$G=(V,E)$に対するアルゴリズムの最悪計算量を評価する.
定理\ref{thm:decrease-betweenness-weight-complexity}および
定理\ref{thm:increase-betweenness-weight-complexity}より,
最短経路更新の前後で媒介中心性を更新するアルゴリズムの最悪計算量は,$S(v)=\tau(s)=V$のときで,
$\mathcal{O}(|V||E|+|V|^2\log|V|)$である.これは,Brandesのアルゴリズムの計算量と同じである.
また,定理\ref{thm:incremental-shortest-path-update}および
定理\ref{thm:decremental-shortest-path-update}より,
最短経路を更新するアルゴリズムの最悪計算量は,$S(v)=T(u)=V$のときで,
$\mathcal{O}(|V||E|\log|V|)$である.これは,Dijkstraのアルゴリズムを
全頂点に対して行うときの計算量と同じである.
