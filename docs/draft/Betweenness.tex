\chapter{媒介中心性の定義と計算法}
\label{chap:definition-of-betweenness-centrality}
本章では,媒介中心性の定義を示し,その計算法であるBrandesのアルゴリズムについて説明する.

\section{媒介中心性}
本稿を通してネットワークは単純連結無向グラフ$G=(V,E)$で表されるとする.ただし$V=\{1,2,\ldots,N\}$は頂点集合,$E=\{e_1,e_2,\ldots,\allowbreak e_M\}$は辺集合である.$G$が単純無向グラフであるから$E$の各要素は異なる2頂点の非順序対である.

$G$の頂点$s$から頂点$t\,(\neq s)$への最短経路を$G_{st}=(V_{st},E_{st})$と表す.$V_{st} \subseteq V$は$s$から$t$への最短経路上にあるすべての頂点の集合($s$, $t$も含む)を表し,$E_{st}$は$s$から$t$への最短経路上にあるすべての有向辺の集合を表している.$E_{st}$の各要素は異なる2頂点の順序対である.また,頂点$s$からすべての頂点への最短経路を$G_{s}=(V,E_s)$と表す.$E_s$は$s$から他のすべての頂点への最短経路上にあるすべての有向辺の集合である.明らかに$E_{st}$は$E_s$の部分集合である.

グラフ$G=(V,E)$の頂点$s$から頂点$t$への最短経路の長さと個数をそれぞれ$d_{st}$, $\sigma_{st}$で表す.$G$は無向グラフなので,異なる2頂点の組$\{s,t\}$のすべてに対して$d_{st}=d_{ts}$および$\sigma_{st}=\sigma_{ts}$が成り立つ.以下の議論では,便宜上,すべての$s \in V$に対し$d_{ss}=0$, $\sigma_{ss}=1$とする.

以上の記号に加えて,グラフ$G=(V,E)$の頂点$s$から頂点$t$への最短経路の中で頂点$i$を通るものの個数を$\sigma_{st}(i)$で表すと,頂点$i$の媒介中心性$B_i$は
\begin{equation}
B_i=\sum_{s\neq i}\sum_{t\neq {i,s}}\frac{\sigma_{st}(i)}{\sigma_{st}}
\label{eq:bc}
\end{equation}
で定義される\cite{Freeman1977}.
すなわち,頂点$i$の媒介中心性は$s$から$t$への最短経路の個数とその中で頂点$i$を通るものの個数の比を$s,t$のすべての組について足し合わせたものである.したがって,媒介中心性の大きな頂点は多くの2頂点を結ぶ最短経路上にあり,この意味で重要度が高いと考えられる.

\section{媒介中心性を計算するBrandesのアルゴリズム}
すべての頂点の媒介中心性を求める単純な方法は以下の通りである.まず,$s=1,2,\ldots,N$に対して,幅優先探索を用いて$s$から他のすべての頂点への最短経路を求める.その過程で,$s$から他のすべての頂点への最短経路長と最短経路数も同時に求める.次に,互いに異なる3頂点の組$\{s,t,i\}$のすべてに対して,$\sigma_{st}(i)$の値を後述の(\ref{eqn:sigma_sti})により計算する.
最後に媒介中心性$B_i$ $(i=1,2,\ldots,N)$を(\ref{eq:bc})により計算する.しかしながら,この方法の計算量は$\mathcal{O}(N^3)$であり,大規模なネットワークでは膨大な計算時間が必要になる.

媒介中心性の効率的計算法として最も広く用いられているのはBrandes~\cite{Brandes2001}によって提案されたアルゴリズムである.それを以下に示す.

\begin{algorithm}[H]
  \caption{Brandesのアルゴリズム}
\label{algo:Brandes}
\rm \ \\
入力:単純無向連結グラフ$G=(V,E)$\\
出力:$B_1,B_2,\ldots,B_N$
{\renewcommand{\labelenumi}{\arabic{enumi}.}
\begin{enumerate}
\item $s\leftarrow 1$,$B_i \leftarrow 0$\ ($i=1,2,\ldots,N$)とする.
%
\item 頂点$s$からすべての頂点への最短経路$G_s=(V,E_s)$を幅優先探索で求め,同時に$d_{si}$, $\sigma_{si}$\ ($i=1,2,\ldots,N$)も求める. 
%
\item $\delta_s(i) \leftarrow 0$ ($i=1,2,\ldots,N$)とする. 
%
\item 次式によって$\delta_s(i)$の値を更新する.
\begin{multline*}
\delta_s(i)\leftarrow \sum_{j:(i,j)\in E_s} \frac{\sigma_{si}}{\sigma_{sj}}(1+\delta_s(j)), 
\\ i=1,2,\ldots,s-1,s+1,s+2,\ldots,N.
\end{multline*}
%
\item 次式によって$B_i$の値を更新する.
\begin{equation*}
B_i \leftarrow B_i+\delta_s(i), \quad i=1,2,\ldots,N.
\end{equation*}
%
\item $s=N$ならば$B_1,B_2,\ldots,B_N$を返して終了する.そうでなければ$s\leftarrow s+1$としてステップ3に戻る. 
\end{enumerate}
}
\end{algorithm}

このアルゴリズムの計算量を考える.ステップ2は$\mathcal{O}(M)$の時間で実行できる.
ステップ4についても,$G$のすべての頂点を$s$からの最短経路の逆向きに一度だけ辿ることにより$\{\delta_s(i)\}_{i=1}^N$の値を更新できるため,$\mathcal{O}(M)$の時間で実行できる~\cite{Brandes2001}.
したがって,アルゴリズム~\ref{algo:Brandes}の全体の計算量は$\mathcal{O}(NM)$であり,それは$\mathcal{O}(N^3)$よりも小さい.特にグラフが疎であるとき,すなわち$M \ll N^2$が成り立つとき前者は後者に比べて非常に小さな値となる.
