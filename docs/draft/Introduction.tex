\chapter{序論}

本章ではグラフの頂点の重要度を表す中心性と,更新アルゴリズムと,特に本研究と関連がある媒介中心性更新アルゴリズムについて述べる.

\section{中心性・媒介中心性}

中心性の提唱から各種中心性の特徴

媒介中心性を効率的に計算する方法の研究

\section{オンラインアルゴリズム}

一方,

\section{関連研究}

\section{メモ}

媒介中心性\cite{Freeman1977}はグラフの頂点の重要度を示す指標のひとつである.高い媒介中心性をもつ頂点は,多くの最短経路が通っていることを意味する.媒介中心性を求めるアルゴリズムとしてBrandesのアルゴリズム\cite{Brandes2001}が知られているが,辺の追加や削除といった操作がされると計算を最初から行う必要がある.

そこで,辺の操作に対して,各頂点の媒介中心性を高速に更新するアルゴリズムが開発されてきた.それらの中には,専用のデータ構造を用いるアルゴリズム\cite{Lee2012,Hayashi2015}や,最短経路長と最短経路数の更新を伴うアルゴリズム\cite{Pontecorvi2015,Bergamini2017}がある.また,PontecorviらはDemetrescuらの方法\cite{Demetrescu2003}を応用した方法を開発し\cite{Pontecorvi2015},BergaminiらはRamalingamらの方法\cite{Ramalingam1996}を応用して辺挿入に対する方法を開発した\cite{Bergamini2017}.

一方,Ramalingamらの方法に基づく辺削除時の媒介中心性更新アルゴリズムは知られていない.そこで,本稿ではRamalingamらの最短経路長更新アルゴリズムに基づく媒介中心性更新法を提案する.また,その方法の有効性を実験によって検証する.

