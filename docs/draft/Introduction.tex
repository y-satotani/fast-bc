\chapter{序論}
\label{chap:introduction}

本章では,本研究に関連した分野であるネットワーク科学の概要と応用例について説明する.その中で,ネットワークノードの重要度を測る中心性と,グラフのオンラインアルゴリズムについてはより詳しく説明する.

\section{ネットワーク科学}
\label{sect:network-science}

ネットワーク科学とは,二つの主体間の関係を表すデータの分析,モデル化に関する研究分野である.
ここでの主体とは,対象とする関係によって異なる.例えば,友人関係ならば,主体は個人を表し,食物連鎖の関係ならば,主体は動物の種を表す.
二つの主体間の関係を表すデータであれば主体の種類は問わない.
そのため,ネットワーク科学はあらゆる主体を対象にできる普遍性を有する.
実際,データ構造の面で近いグラフ理論の他,化学\cite{Sylvester1878}や社会科学\cite{Moreno1978}など複数の研究分野と相互に影響している.

一方,高性能な計算機がが普及したことによって,現実にある膨大なネットワークを分析することが可能になった.
例えば,SNSのTwitterの友人関係や,QAサイトのStackExchangeの質問・解答\cite{Movshovitz-Attias2013,Paranjape2017},Social CodingのGitHub\cite{Lima2014}
が挙げられる.

その影響から,現実のネットワークの特徴を上手く表現するようなモデルが盛んに研究されてきた.
古くからあるモデルは1959年にErd{\"{o}}sとR{\'{e}}nyiのランダムグラフで,パーコレーション理論で言い換えると,完全グラフからランダムに辺を取り出す操作を繰り返すことで得られる.
一方,近年では,社会ネットワークに見られる六次の隔たり\cite{Travers1969}の特徴を表したWattsとStrogatz\cite{Watts1998}のスモールワールドネットワークや,
次数分布がべき分布であるスケールフリー性を再現した,Barab{\'{a}}siとAlbertによるスケールフリーネットワークが開発されたことにより,
ネットワーク科学のモデル化に関する部分が進展した.
また,既存のネットワークに時間の概念を加えたTemporal Network\cite{Holme2012}
多層ネットワークMultilayer Network\cite{Kivela2014}も提案されている.

以後,複雑ネットワークのいくつかの応用を示す.
\begin{enumerate}
\item 感染症の伝搬
\item 交通網の分析
\item 社会ネットワークの分析
\end{enumerate}

また,ネットワーク科学の成り立ちやその他の応用は,\cite{Barabasi2016}が詳しい.

\subsection{感染症の伝搬}

2009年にH1N1型インフルエンザウイルスによる21世紀初のパンデミックが発生した.
メキシコで発生したウイルスが日本に到達し,猛威を奮ったことは記憶に新しい.
その一方で,ネットワーク科学を応用した予測\cite{Hufnagel2004}が確立されてから最初の
パンデミックで,その規模を正確に予測した\cite{Balcan2009}.

ネットワーク科学を応用した予測の肝は,輸送ネットワークを分析してウイルスの拡散を
予測,抑制することである.
20世紀以前の感染症拡散は"仕切り"に基づくモデルで,同じ社会物理的な空間にいるならば,
誰もが誰かから感染するものだった.つまり,駅構内や病院などの一定の広さの空間のみを対象としていた.
Hufnagelらのモデルは,従来の局所的な感染に加え,感染者が飛行機に乗って来ることの影響も含んでいる.
このような感染症予測は,インフルエンザあるいはエボラ出血熱のパンデミックの予測に使われる,ネットワーク科学の最も活発な応用のひとつである.

また,生物を媒介するウイルスの他に,コンピュータウイルスや社会ウイルス(ミーム)の拡散を予測することも可能である.
例えば,2010年に携帯電話を媒介するウイルスが拡散する条件を予測した\cite{Wang2009}.
この予測モデルが提案されてから最初に発生したパンデミックは2010年の秋のことで,予測に近い30万台を超える携帯電話が毎日感染した.

\subsection{交通機関の分析}

日々利用する交通網もネットワークとして分析することができる.
特に,交通網の脆弱な箇所を見つけることは,社会インフラの維持の観点から
重要な課題である.

Taylorらは,高速道路をネットワークとみなすことで交通網の脆弱な箇所を特定した\cite{Taylor2007}.
Taylorらによるネットワークの指標には,信頼性と脆弱性がある.
信頼性とはコンピュータネットワークの信頼性と同じく,ある時間の水準で
車が到達できるかを表す.道路の通行のしやすさおよび移動できる確率に関連する.
一方,脆弱性とは道路が不全になった時に引き起こされる社会的,経済的な影響の大きさを表す.つまり,故障に対する結末に関係する.
著者らは,これらの概念に対する簡単な指標を定義して,実際のオーストラリアの主要道路を分析した.

Tizghadamらは,交通網のネットワークの頑健性を最適化するために,
道路の重みを調整するアルゴリズムを開発した\cite{Tizghadam2010}.
進化計算に基づくアルゴリズムである.
ネットワークの頑健性は後に説明する媒介中心性の派生に基づいている.
その指標ランダムウォークの考え方に基づき,より頻繁に通過される頂点・辺の重要度が高いとみなされる.

\subsection{ソーシャルネットワークの分析}

現代において,ソーシャルネットワークといえば,いわゆるSNSの友人関係を表すことが
多いが,もともとは学校や職場といった特定の場所に集まる人のネットワークを指す.
その中で,集団での心理学\cite{Moreno1978}や個人間の情報のやり取り\cite{Christie1952}について議論された.

最近では,ソーシャルネットワーキングサービスの台頭により,従来とは比較にならない程の大きさのネットワークが取得可能となった.
それにより,より巨視的な議論が可能となった.

SNAのハンドブックたち\cite{Pappi2006,Butts2008}

ネットワーク全体をいくつかの集団に分けるコミュニティ検出\cite{Fortunato2010,Leskovec2010}や,
集団の中で重要な人物を特定することをマーケティングに応用する例\cite{Kempe2015},
ゲーム利用者が協力することの理解\cite{Szell2010,Rand2011}が注目される.

複雑ネットワークの応用をいくつか紹介したが,いずれも重要なノードを発見することに意義がありそうである.
次節では,重要なノードの発見に役立つ中心性について述べる.

\section{中心性}
\label{sect:centrality}

ネットワーク内のノードの重要度をはかることは古くからの社会ネットワークの課題である\cite{Christie1952,Ibarra1993}.その重要度は中心性(centrality)と呼ばれ,用途に沿っていくつかの中心性が開発されてきた.その中で最も簡単なものは各ノードの次数(隣接しているノードの数)をそのまま使う次数中心性である.

しかし,次数が高いからと言って,真に"重要"でない場合がある.
以下,現在まで提案された主な中心性について説明する.
グラフ理論の記号および意味に関しては,第\ref{chap:preliminary}章を参照されたい.

\subsection{近接中心性}
\label{subsect:closeness}

近接中心性(Closeness centrality)\cite{Bavelas1948,Beauchamp1965}とは,
ネットワーク内の他のノードとの距離の総和の逆数として定義される.つまり,
\begin{equation*}
  C_v=\frac{1}{\sum_wd(w,x)}.
\end{equation*}

近接中心性が高いノードは他のノードに近い位置にあり,ここから情報を伝達すると
情報を速く伝達できると考えられる.
また,この定義は,通信は最短経路を通ることを仮定しているので,ある通信モデルでは
現実的でないことがある.これを考慮して,ランダムウォークに基づく近接中心性も
提案されている.

\subsection{媒介中心性}
\label{subsect:betweenness}

媒介中心性(Betweenness centrality)\cite{Anthonisse1971,Freeman1977}とは,
頂点ペア$(s,t)$の最短経路数に占める,自身を通る最短経路数の割合の総和として
定義される.つまり,
\begin{equation*}
  B_v=\sum_{s\neq v}\sum_{t\neq {v,s}}\frac{\sigma_{st}(v)}{\sigma_{st}}.
\end{equation*}

媒介中心性の大きなノードは多くの最短経路上にあり,通信に欠かせないノードであると言える.
近接中心性と同じく,この定義は,通信は最短経路を通ることを仮定しているので,ある通信モデルでは
現実的でないことがある.これを考慮して,ランダムウォークに基づく媒介中心性も
提案されている.

\subsection{固有ベクトル中心性}
\label{subsect:eigenvector}

固有ベクトル中心性(Eigenvector centrality)\cite{Bonacich1991}とは,
隣接する頂点の固有ベクトル中心性の総和として表現される.
つまり,$\lambda$を定数として,
\begin{equation*}
  x_v=\frac{1}{\lambda}\sum_{w\in\mathcal{N}_G(v)}x_w.
\end{equation*}
また,$\mathbf{A}$をグラフの隣接行列とすると,固有ベクトル方程式として表すことができる.
\begin{equation*}
  \mathbf{A}\mathbf{x}=\lambda\mathbf{x}.
\end{equation*}
一般的に,$\mathbf{A}$の固有値は複数存在するが,$\lambda$は$\mathbf{A}$の最大の固有値とする.

固有ベクトル中心性は,近傍の固有ベクトル中心性が高いほど自身の中心性も高い
性質があるので,Webページや学術論文誌のスコアリングに応用される.

\section{関連研究}
本節では,本研究と特に関連する媒介中心性の高速計算法の研究と,
グラフのオンラインアルゴリズムの研究と,それらを合わせた研究について述べる.
本節の最後に,本研究の研究目的を説明する.

\subsection{媒介中心性の計算法}
ここまで,複雑ネットワーク科学およびネットワークの中心性について解説した.
その中でも,媒介中心性はネットワークの頑健性と関係する指標で,これを高速に
計算することが求められる.

重み付き無向グラフの全頂点の媒介中心性をFreemanの定義に沿って計算すると,
その計算量は頂点数を$N$,辺数を$M$とすると$\mathcal{O}(N^2\log M+N^3)$である.
2001年,Brandesは媒介中心性を$\mathcal{O}(NM+N^2\log N$で計算するアルゴリズム
\cite{Brandes2001}を開発した.

\subsection{グラフオンラインアルゴリズム}
しかしながら,辺の追加や削除といった操作がされると計算を最初から行う必要がある.

そのため,辺の操作に対して,最短経路などのグラフの指標を保持しつつ,
操作の差分に対してその指標を更新するオンラインアルゴリズムが注目されてきた.

中でも,本研究と関連するものは,最短経路を更新するアルゴリズムである.

\begin{enumerate}
\item Ramalingamらの方法\cite{Ramalingam1996}
\item Demetrescuらの方法\cite{Demetrescu2003}
\end{enumerate}

\subsection{媒介中心性を求めるオンラインアルゴリズム}

そこで,辺の操作に対して,各頂点の媒介中心性を高速に更新するアルゴリズムが開発
されてきた.
それらの中には,専用のデータ構造を用いるアルゴリズム\cite{Lee2012,Hayashi2015}
や,最短経路長と最短経路数の更新を伴うアルゴリズム
\cite{Pontecorvi2015,Bergamini2017}がある.
また,PontecorviらはDemetrescuらの方法を応用した方法を
開発し\cite{Pontecorvi2015},
BergaminiらはRamalingamらの方法を応用して辺挿入に対する
方法を開発した\cite{Bergamini2017}.

一方,Ramalingamらの方法に基づく辺削除時の媒介中心性更新アルゴリズムは知られていない.そこで,本稿ではRamalingamらの最短経路長更新アルゴリズムに基づく媒介中心性更新法を提案する.また,その方法の有効性を実験によって検証する.

