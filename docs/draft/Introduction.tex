\chapter{序論}
\label{chap:introduction}

\begin{enumerate}
  \item ネットワーク科学が注目を集めて久しいのです
\item 用途に応じた中心性がいろいろあるのです
\item その中でも,媒介中心性は頑健性の観点から重要さが分かるのです
\item 関連研究がたくさんあるのです
\item その中でも,
  \begin{itemize}
  \item Single Source Shortest Path (SSSP) DAGを保存する
  \item RamalingamとRepsのアルゴリズムに基づく
  \item 辺削除時の媒介中心性を更新するアルゴリズム
  \end{itemize}
  \par が無いのです.なので開発して理論的・実験的に評価するですよ.
\item 論文の構成はこんな感じにするのです
  \begin{enumerate}
  \item 第\ref{chap:network-science}章でネットワーク科学の背景を説明するのです
  \item 第\ref{chap:related-work}章で関連研究を挙げて本研究の目的を示すのです
  \item 第\ref{chap:preliminary}章でグラフの数学的表現など,その後の章の準備をするのです
  \item 第\ref{chap:algorithm}章で本研究で提案するアルゴリズムを説明するのです
  \item 第\ref{chap:complexity-analysis}章で計算量を解析して第\ref{chap:algorithm}章で提案したアルゴリズムの性能を理論的に評価するですよ
  \item 第\ref{chap:experiment}章で性能を実験的に評価するのです
  \item 最後に第\ref{chap:conclusion}章で結論を述べるですよ,シャー
  \end{enumerate}
\end{enumerate}

