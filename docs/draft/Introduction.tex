\chapter{序論}
\label{chap:introduction}

グラフ理論から派生したネットワーク科学が注目を集めて久しい.
ネットワーク科学の強みは,物とそれらの間の関係が定義できれば,あらゆる分野に応用できることである.
例えば,社会科学における人のつながりや,化学における原子や分子の結合を対象にできる.

初期のネットワーク科学は学校や職場,あるいは低分子といった小規模な集まりを対象としていたが,
計算機の発展と普及により,対象とできる規模は大きくなった.例えば,感染症の世界的流行のシミュレーション
や国レベルの道路ネットワークの分析,ソーシャルネットワーキングサイトの分析が挙げられる.

分析が大規模になった一方,現実のネットワークの特徴を上手く表現するようなモデルが開発されてきた.
その中でも著名なものは,WattsとStrogatzのスモールワールドネットワーク\cite{Watts1998}と,
Barab{\'{a}}siとAlbertのスケールフリーネットワーク\cite{Barabasi1999}である.

分析でもモデルでも,重要なノードないしノードの重要性に関して議論されることは自然である.
例えば,いわゆるインフルエンサーや,スケールフリーネットワークにおけるハブといった重要なノードは
ネットワーク全体に影響を与えうる.
ノードの重要性に関して定量的に議論するため,中心性の概念が導入され,
現在まで多くの中心性が考案されてきた.その中でも媒介中心性\cite{Freeman1977}は,
最短経路がそのノードを通る数を表し,ネットワーク全体の頑健性を表す指標としても用いられる.

媒介中心性を求めるアルゴリズムとしてBrandesのアルゴリズム\cite{Brandes2001}が開発されて以来,
多くのアルゴリズムが開発された.
例えば,前処理によりネットワークを簡単にする方法\cite{Puzis2012,Bentert2018}や,
分割統治法\cite{Erdos2015}の考え方を用いた方法がある.
また,処理を並列に行い高速化を図った方法\cite{Bader2006,Tan2009,Edmonds2010,Bernaschi2016}や,
媒介中心性の近似値を効率的に求める方法\cite{Brandes2007,Bader2007,Pfeffer2012,Yoshida2014}も
開発されている.

一方,現実のネットワークは時間とともにリンクが出現したり消滅したりする時変ネットワーク\cite{Holme2012}
であることから,そのようなネットワークの媒介中心性を効率的に計算する方法も開発されている.
Minimum union cycleと呼ばれる閉路の集合と媒介中心性を保持し,変更に対してそれらを更新する方法
\cite{Lee2012,Singh2015}や,
Hypergraph sketch\cite{Yoshida2014}とTwo-ball indexとSpecial-purpose reachability index
を保持,更新することで媒介中心性の近似値を求める方法\cite{Hayashi2015}や,
時不変ネットワークに対する近似法を用いた方法\cite{Bergamini2015a,Bergamini2015b}が開発されている.

その中で,最短経路やそれに関する値を保存する方法も提案されている.
Kasらは,RamalingamとRepsの最短経路更新法\cite{Ramalingam1996}を応用して,
辺挿入時の媒介中心性を更新する方法を開発した\cite{Kas2013}.
また,Nasreらは,Kargerらの\cite{Karger1993}の方法を時変ネットワークに適用して
辺挿入時の媒介中心性を更新する方法を開発し\cite{Nasre2014a},
DemetrescuとItalianoの最短経路更新法\cite{Demetrescu2003}を
基にして辺削除時の媒介中心性を更新する方法を開発した\cite{Nasre2014b}.
また,PontecorviとRamachandranは,DemetrescuとItalianoの方法を応用して
頂点追加時の媒介中心性を更新する方法を開発した\cite{Pontecorvi2015}.
さらに,Bergaminiらは,RamalingamとRepsのアルゴリズムを基に,
辺挿入時の媒介中心性更新アルゴリズムを開発した\cite{Bergamini2017}.

しかし,RamalingamとRepsの方法に基づく辺削除時の媒介中心性更新アルゴリズムは知られていない.
そこで,本研究ではRamalingamとRepsの最短経路更新アルゴリズムに基づく媒介中心性更新法を提案する.
また,その方法の有効性を実験によって検証する.

本稿の構成は次のとおりである.
まず,第\ref{chap:network-science}章でネットワーク科学や中心性といった背景を説明し,
第\ref{chap:related-work}章で関連研究を挙げて本研究の目的を示す.
続いて第\ref{chap:preliminary}章でグラフの数学的表現などの説明をして,
第\ref{chap:algorithm}章でRamalingamとRepsの方法に基づく,
辺削除時の媒介中心性更新アルゴリズムを提案する.
さらに提案したアルゴリズムの性能を第\ref{chap:complexity-analysis}章で解析的に,
第\ref{chap:experiment}章で実験的に評価する.
最後に第\ref{chap:conclusion}章で結論を述べる.
