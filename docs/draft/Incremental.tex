\chapter{一辺挿入時の媒介中心性更新法}
\label{chap:update-bc-on-insert}
本章では,$G=(V,E)$に辺$e=\{v,w\} \not\in E$を挿入して得られるグラフを
$G'=(V,E')$とする.$G'$における頂点$s$から頂点$t\,(\neq s)$への最短経路を有向グラフ$G'_{st}=(V'_{st},E'_{st})$で表し,すべての頂点から頂点$s$への最短経路を有向グラフ$G'_s=(V,E'_s)$で表す.また,$G'$における頂点$s$から頂点$t\,(\neq s)$への最短経路の個数と長さをそれぞれ$\sigma'_{st}$, $d'_{st}$で表す.

\section{最短経路長の更新}
\label{sect:update-dist-on-insert}

挿入後の頂点$z$への最短経路長$d'_{xz}$を求めるアルゴリズムはRamalingamとRepsによって与えられている.以下,そのアルゴリズムについて説明する.

\begin{lemma-without-proof}
  頂点$v$が$d'_{xz}<d_{xz}$を満たすための必要十分条件は以下である.
  \begin{equation*}
    d_{vz}>l_{vw}+d_{wz}
  \end{equation*}
\end{lemma-without-proof}

\begin{lemma-without-proof}
  頂点$x\in G,x\neq v$が$d'_{xz}<d_{xz}$を満たすための必要十分条件は以下である.
  \begin{equation*}
    \exists y\in\mathcal{N}_G(x),yが変化する\land d_{xz}>l_{xy}+d_{yz}
  \end{equation*}
\end{lemma-without-proof}

これらの頂点は,最短経路長の観点から削除の影響を受ける.

\begin{figure}[tb]
  \centering
  \def\svgwidth{.35\linewidth}
  \input{incremental-dist-idea-1.pdf_tex}
  \caption{$z$への最短経路長が変化する頂点を求める例.数字は$z$への最短経路長を表す}
  \label{fig:incremental-dist-idea-1}
\end{figure}

\begin{figure}[tb]
  \centering
  \def\svgwidth{.35\linewidth}
  \input{incremental-dist-idea-2.pdf_tex}
  \caption{$z$への最短経路長が変化する頂点を求める例.数字は$z$への最短経路長を表す}
  \label{fig:incremental-dist-idea-2}
\end{figure}

頂点$x\in G$が変化することが分かった時,次の式を用いて挿入後の距離$d'_{xz}$をもとめる.

\begin{corollary-without-proof}
  \begin{equation*}
    d'_{xz}=l_{xy}+d'_{yz} (y\mathrm{は条件の}y.)
  \end{equation*}
\end{corollary-without-proof}

さらに,$xy\ldots z\in E_z,y\in\mathcal{N}_G(x)$は$E'_z$に含まれないので,$y$をペア依存度$\delta_z(y)$が更新するリストに追加する.

\section{最短経路数の更新}
\label{sect:update-sigma-on-delete}

挿入後の$z$への最短経路数$\sigma'_{xz}$を求めるアルゴリズムを与える.以下,そのアルゴリズムを説明する.

頂点$v$が$\sigma'_{xz}\neq\sigma_{xz}$を満たすための必要十分条件は以下である.
\begin{equation*}
  d_{vz}\geq l_{vw}+d_{wz}
\end{equation*}

頂点$x\in G,x\neq v$が$\sigma'_{xz}\neq\sigma_{xz}$を満たすための必要十分条件は以下である.
\begin{equation*}
  \exists y\in\mathcal{N}_G(x),yが変化する\land d_{xz}\geq l_{xy}+d_{yz}
\end{equation*}

これらの頂点は,最短経路長の観点から削除の影響を受ける.

頂点$x\in G$が,変化することが分かった時,次の式を用いて挿入後の最短経路数$\sigma'_{xz}$をもとめる.式より,$d'_{xz}$の昇順で計算する.

\begin{equation*}
  \sigma'_{xz}=\sum_{(x,y)\in E'}\sigma'_{yz}
\end{equation*}

さらに,$\sigma_{xz}$が変化したので,$x$をペア依存度$\delta_z(y)$が更新するリストに追加する.

\section{ペア依存度の更新}
\label{sect:update-delta-on-delete}

挿入後のペア依存度$\delta'_z(x)$をBrandesのアルゴリズムによって求める.具体的には,\ref{sect:update-dist-on-insert}と\ref{sect:update-sigma-on-insert}で求めた影響を受ける頂点から,$d'_{xz}$の降順となるように次式を計算しながら走査する.
\[ \delta'_{z}(x)\gets\sum_{(y,x)\in E_z}\frac{\sigma'_{xz}}{\sigma'_{yz}}(1+\delta'_z(y)) \]

\section{媒介中心性の更新}
媒介中心性を更新する前に,すべての頂点に対するペア依存度を更新する必要がある.すべての頂点に対するペア依存度を更新した後,各頂点$x$について,$B'_x=\sum_{z\neq x}\delta'_{z}(x)$として辺削除後の媒介中心性を求める.

