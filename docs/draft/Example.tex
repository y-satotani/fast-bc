\chapter{例}

\bgroup
\setlength{\fboxrule}{0pt}
\setlength{\fboxsep}{10pt}

\section{挿入例}

\begin{tabular}{cp{8cm}}
  \raisebox{-.8\totalheight}{\fbox{
    \def\svgwidth{.35\linewidth}
    \input{incremental-path-0.pdf_tex}
  }}
  &
  もともとのグラフ
  \\
  \raisebox{-.8\totalheight}{\fbox{
    \def\svgwidth{.35\linewidth}
    \input{incremental-path-1.pdf_tex}
  }}
  &
  $(E,H)$を挿入,$l_{EH}=2$
  \\
  \raisebox{-.8\totalheight}{\fbox{
    \def\svgwidth{.35\linewidth}
    \input{incremental-path-2.pdf_tex}
  }}
  &
  $d_{EH}=5$なので,$d_{EH}>l_{EH}+d'_{HH}$.
  よって,$d'_{EH}\gets l_{EH}+d'_{HH}=2$.
  さらに,$(E,F)\in E_H$なので,$(E,F)$をキューに追加する.
  \\
  \raisebox{-.8\totalheight}{\fbox{
    \def\svgwidth{.35\linewidth}
    \input{incremental-path-3.pdf_tex}
  }}
  &
  $d_{FH}=4$なので,$d_{FH}>l_{FE}+d'_{EH}$.よって,
  $d'_{FH}\gets l_{FE}+d'_{EH}=3$.

  さらに,$(F,E)\in \bar{E_H}$つまり,$(E,F)in E_H$なので,
  $\delta_H(E)$が変化するフラグを設定する.

  さらに,$(F,G)\in\bar{E_H}$かつ$d'_{GH}<l_{FG}+d'_{FH}$なので,
  $(F,G)\notin \bar{E'_H}$.よって,
  $\delta_H(G)$が変化するフラグを設定する.

  さらに,$(C,F)\in\bar{E_H}$かつ,$d_{CH}\gt l_{CF}+d'_{FH}$なので,
  $(C,F)$をキューに追加する.
\end{tabular}
\newpage
\begin{tabular}{cp{8cm}}
  \\
  \raisebox{-.8\totalheight}{\fbox{
    \def\svgwidth{.35\linewidth}
    \input{incremental-path-4.pdf_tex}
  }}
  &
  $d_{CH}=6$.
  $d_{CH}\gt l_{CF}+d'_{FH}$んおで,$d_{CH}\gets l_{CF}+d'_{FH}=5$.

  また,$(C,B)\in\bar{E_H}$かつ,$d_{BH}<l_{BC}+d'_{CH}$なので,
  $(C,B)\notin\bar{E'_H}$.よって,
  $\delta_H(B)$を,ペア依存度が変化するフラグを設定する.

  さらに,$(D,C)\in\bar{E_H}$かつ$d_{DH}\gt l_{DC}+d'_{CH}$なので,
  $(D,C)$をキューに追加する.
  \\
  \raisebox{-.8\totalheight}{\fbox{
    \def\svgwidth{.35\linewidth}
    \input{incremental-path-5.pdf_tex}
  }}
  &
  $d_{DH}=9$.
  $d_{DH}>l_{DC}+d'_{CH}$なので,$d'_{DH}\gets l_{DC}+d'_{CH}=8$.

  また,$(A,D)\in\bar{E_H}$なので,$(A,D)$をキューに追加する.
  \\
  \raisebox{-.8\totalheight}{\fbox{
    \def\svgwidth{.35\linewidth}
    \input{incremental-path-6.pdf_tex}
  }}
  &
  \\
  \raisebox{-.8\totalheight}{\fbox{
    \def\svgwidth{.35\linewidth}
    \input{incremental-path-7.pdf_tex}
  }}
  &
\end{tabular}

\newpage

\section{削除例}

\begin{tabular}{cp{8cm}}
  \raisebox{-.8\totalheight}{\fbox{
    \def\svgwidth{.35\linewidth}
    \input{decremental-path-0.pdf_tex}
  }}
  &
  もともとのグラフ
  \\
  \raisebox{-.8\totalheight}{\fbox{
    \def\svgwidth{.35\linewidth}
    \input{decremental-path-1.pdf_tex}
  }}
  &
  $(E,H)$を削除.
  \\
  \raisebox{-.8\totalheight}{\fbox{
    \def\svgwidth{.35\linewidth}
    \input{decremental-path-2.pdf_tex}
  }}
  &
  $d_{EH}=l_{EH}+d_{HH}$なので,$E$を距離更新リストに追加する.
  \\
  \raisebox{-.8\totalheight}{\fbox{
    \def\svgwidth{.35\linewidth}
    \input{decremental-path-3.pdf_tex}
  }}
  &
  $d_{FH}=l_{FE}+d_{EH}$なので,$F$を距離更新リストに追加する.

  同様に,$C,D,A$も距離更新リストに追加する.
\end{tabular}
\newpage
\begin{tabular}{cp{8cm}}
  \\
  \raisebox{-.8\totalheight}{\fbox{
    \def\svgwidth{.35\linewidth}
    \input{decremental-path-4.pdf_tex}
  }}
  &
  $C,F$について,その近傍に距離更新リストに無い頂点が存在するので,
  それぞれ,
  $(C,l_{CB}+d_{BH}),(F,l_{FG}+d_{GH})$
  を更新キューに追加する.

  更新キュー:$((F,4),(C,6))$.
  \\
  \raisebox{-.8\totalheight}{\fbox{
    \def\svgwidth{.35\linewidth}
    \input{decremental-path-5.pdf_tex}
  }}
  &
  キューから,$(F,4)$を取り出す.

  $d'_{FH}\gets 4$と更新.
  また,$\sigma'_{FH}\gets\sigma'_{GH}=1$と更新.

  $F$の近傍のうち,更新リストに含まれているものについて,
  $(E,l_{EF}+d'_{FH})$を距離更新キューに追加する.

  そして,$F$を距離更新リストから削除する.

  更新キュー:$((E,5),(C,6))$.
  \\
  \raisebox{-.8\totalheight}{\fbox{
    \def\svgwidth{.35\linewidth}
    \input{decremental-path-6.pdf_tex}
  }}
  &
  キューから$(E,5)$を取り出す.

  $d'_{EH}\gets 5$と更新する.
  また,$\sigma'_{EH}\gets\sigma'_{FH}=1$と更新する.

  $E$を,$\delta$更新リストに追加する.

  そして,$E$を距離更新リストから削除する.

  距離更新キュー:$((C,6))$
  \\
  \raisebox{-.8\totalheight}{\fbox{
    \def\svgwidth{.35\linewidth}
    \input{decremental-path-7.pdf_tex}
  }}
  &
  キューから$(C,6)$を取り出す.

  $d'_{CH}\gets 6$と更新する.
  また,$\sigma'_{CH}\gets\sigma'_{BH}+\sigma'_{FH}=2$と更新する.

  $C$の近傍のうち,更新リストに含まれているものについて,
  $(D,l_{DC}+d'_{CH})$を距離更新キューに追加する.

  $C$を,$\delta$更新リストに追加する.

  そして,$C$を距離更新リストから削除する.

  距離更新キュー:$((D,9))$
\end{tabular}
\newpage
\begin{tabular}{cp{8cm}}
  \raisebox{-.8\totalheight}{\fbox{
    \def\svgwidth{.35\linewidth}
    \input{decremental-path-8.pdf_tex}
  }}
  &
  これらを,距離更新キューが空になるまで繰り返す.
  \\
  \raisebox{-.8\totalheight}{\fbox{
    \def\svgwidth{.35\linewidth}
    \input{decremental-path-9.pdf_tex}
  }}
  &
\end{tabular}

\egroup
