
\section{Proofs}

\subsection{Proof of Lemma \ref{lm:affected-vertices}}
\begin{proof}
  If $d_{xv}+l_{vw}+d_{wt}=d_{vt}$, at least one shortest path contains $(v,w)$ and one of the following conditions satisfies;
  \begin{equation}
    \begin{cases}
      d'_{xt}>d_{xt} & \text{(all paths contain $(v,w)$)},\\
      d'_{xt}=d_{xt},\,\sigma'_{xt}<\sigma_{xt} & \text{(otherwise)}.
    \end{cases}
    \label{eq:affected-vertices-1}
  \end{equation}
  In both cases, $d_{xt}\neq d'_{xt}\lor\sigma_{xt}\neq\sigma'_{xt}$ satisfies.
  
  Conversely, since $(v,w)$ was deleted, one of conditions on equation \eqref{eq:affected-vertices-1} satisfies if $d_{xt}\neq d'_{xt}\lor\sigma_{xt}\neq\sigma'_{xt}$ and at least one path contains $(v,w)$.
  Therefore, $d_{xv}+l_{vw}+d_{wt}=d_{vt}$ satisfies.
\end{proof}

\subsection{Proof of Lemma \ref{lm:update-delta-1}}
\begin{proof}
  \begin{align}
    &\delta_s(v)\nonumber\\
    =&\sum_{w\in\mathcal{S}_{G_s}(v)}\cfrac{\sigma_{sv}}{\sigma_{sw}}(1+\delta_s(w)) \nonumber\\
    =&\sum_{w\in\mathcal{S}_{G_s}(v)}\cfrac{\sigma_{sv}}{\sigma_{sw}}\left(1+
    \sum_{w\in\mathcal{S}_{G_s}(w)}\cfrac{\sigma_{sw}}{\sigma_{sw}}(1+\delta_s(w))
    \right) \nonumber\\
    &\vdots \nonumber\\
    =&\sum_{w\in\mathcal{S}_{G_s}(v)}\cfrac{\sigma_{sv}}{\sigma_{sw}}\left(1+\cdots
    \sum_{w\in\mathcal{S}_{G_s}(w)}\cfrac{\sigma_{sw}}{\sigma_{sw}}(1+\delta_s(w))
    \cdots\right)
    \label{eq:update-delta-1}
  \end{align}
  On equation \eqref{eq:update-delta-1}, since $G_s$ is finite and acyclic,
  $\delta_{s\bullet}(v)$ cannot be expanded infinitely.
  with the following propositions: $\delta_{s\bullet}(w)=0$ for $\mathcal{S}_{G_s}(w)=\varnothing$,
  $E_{G_s}(v)=E_{G'_s}(v)$ is satisfied if $V_{G_s}(v)=V_{G'_s}(v)$, and
  $\mathcal{S}_{G_s}(v)=\mathcal{S}_{G'_s}(v)$,
  if $E_{G_s}(v)=E_{G'_s}(v)$ and $\sigma_{sw}=\sigma'_{sw}$ for all $w\in V_{G_s}$,
  \begin{equation*}
    \begin{aligned}
      &\delta_s(v) \\
      =&\sum_{w\in\mathcal{S}_{G_s}(v)}\cfrac{\sigma_{sv}}{\sigma_{sw}}\left(1+\cdots
      \sum_{w\in\mathcal{S}_{G_s}(w)}\cfrac{\sigma_{sw}}{\sigma_{sw}}(1+\delta_s(w))
      \cdots\right) \\
      =&\sum_{w\in\mathcal{S}_{G_s}(v)}\cfrac{\sigma_{sv}}{\sigma_{sw}}\left(1+\cdots
      \sum_{w\in\mathcal{S}_{G_s}(w)}\cfrac{\sigma_{sw}}{\sigma_{sw}}
      \cdots\right) \\
      =&\sum_{w\in\mathcal{S}_{G'_s}(v)}\cfrac{\sigma'_{sv}}{\sigma'_{sw}}\left(1+\cdots
      \sum_{w\in\mathcal{S}_{G'_s}(w)}\cfrac{\sigma'_{sw}}{\sigma'_{sw}}
      \cdots\right) \\
      =&\delta'_s(v).
    \end{aligned}
  \end{equation*}
\end{proof}

\subsection{Proof of Lemma \ref{lm:update-delta-2}}
\begin{proof}
  Firstly, we show \ref{item:update-delta-2-1}$\iff$\ref{item:update-delta-2-2}.
  Since $\bigcup_{(w,x)\in E_s(v)}\{w,x\}=V_s(v)$, $E_{G_s}(v)=E_{G'_s}(v)$ then $V_{G_s}(v)=V_{G'_s}(v)$.
  And, if $E_{G_s}(v)=E_{G'_s}(v)$ then $\mathcal{S}_{G_s}(w)=\mathcal{S}_{G'_s}(w)$ for all $w\in V_{G_s}(v)$. Therefore, \ref{item:update-delta-2-1}$\implies$\ref{item:update-delta-2-2}.

  Conversely, if $V_{G_s}(v)=V_{G'_s}(v)$ and $\mathcal{S}_{G_s}(w)=\mathcal{S}_{G'_s}(w)$ for all $w\in V_{G_s}(v)$, then, $E_{G_s}(v)=E_{G'_s}(v)$.
  Therefore, \ref{item:update-delta-2-2}$\implies$\ref{item:update-delta-2-1}.

  Next, we prove \ref{item:update-delta-2-2}$\iff$\ref{item:update-delta-2-3}.
  \begin{equation*}
    \begin{aligned}
      &\forall w\in V_s(v),\,\mathcal{S}_{G_s}(w)=\mathcal{S}_{G'_s}(w)\\
      \iff&\forall w\in V_s(v),\,\mathcal{S}_{G_s}(w)\subset\mathcal{S}_{G'_s}(w)
      \land\mathcal{S}_{G_s}(w)\supset\mathcal{S}_{G'_s}(w)\\
      \iff&\forall w\in V_s(v),
      \,\left(\forall x\in\mathcal{S}_{G_s}(w),\,x\in\mathcal{S}_{G'_s}(w)\right)
      \land\left(\forall x\in\mathcal{S}_{G'_s}(w),\,x\in\mathcal{S}_{G_s}(w)\right)\\
      \iff&\forall w\in V_s(v),
      \left(\forall x\in\mathcal{N}_{G_s}(w),x\in\mathcal{S}_{G_s}(w),x\in\mathcal{S}_{G'_s}(w)\right.
      \left.\lor x\notin\mathcal{S}_{G_s}(w),x\notin\mathcal{S}_{G'_s}(w)\right)\\
      &\land\left(\forall x\in\mathcal{N}_{G_s}(w),\,x\in\mathcal{S}_{G'_s}(w),x\in\mathcal{S}_{G_s}(w)\right.
      \left.\lor x\notin\mathcal{S}_{G'_s}(w),x\notin\mathcal{S}_{G_s}(w)\right)\\
      \iff&\forall w\in V_s(v),
      \left(\forall x\in\mathcal{N}_{G_s}(w),\,
      x\in\mathcal{S}_{G_s}(w),x\in\mathcal{S}_{G'_s}(w)\right.
      \left.\lor x\notin\mathcal{S}_{G_s}(w),x\notin\mathcal{S}_{G'_s}(w)\right)\\
      \iff&\forall w\in V_s(v),\,\left(\forall x\in\mathcal{N}_{G_s}(w),\right.
      \left.x\in\mathcal{S}_{G_s}(w)\land x\in\mathcal{S}_{G'_s}(w)\right.
      \left.\lor x\notin\mathcal{S}_{G_s}(w)\land x\notin\mathcal{S}_{G'_s}(w)\right)
    \end{aligned}
  \end{equation*}
\end{proof}

\section*{Proof of Theorem \ref{tm:time-complexity}}
In this section, we examine the efficiency of our proposed algorithm from time complexity analysis.
Firstly, we introduce some concepts on affected amount of graph components for later discussion.

\begin{definition}
  A set of vertices with updated its dependency on $t$ is denoted by $V_\delta(t)$. In other words, all elements $x$ in $V_\delta(t)$ had updated their $\delta_{t\bullet}(x)$..
  $E_\delta(t)$ denotes an edge set consists of edges, incident to vertex in $V_\delta(t)$.
  The mathematical definitions are followings:
  \begin{equation*}
    \begin{aligned}
      V_\delta(t)&=\{x|x\in V,\exists w\in\mathcal{N}_G(x),
      x\in\mathcal{S}_{G_t}(w)\veebar x\in\mathcal{S}_{G'_t}(w)\}\\
      E_\delta(t)&=\{(x,y)|(x,y)\in E,\,x\in V_\delta(t)\lor y\in V_\delta(t)\}.
    \end{aligned}
  \end{equation*}

  Similarly, $V_{\delta'}(t)$ denotes a vertex set consists of vertices which are updated its distance or the number of paths.
  $E_{\delta'}(t)$ denotes an edge set consists of edges, incident to vertex in $V_{\delta'}(t)$.
  The mathematical definitions are followings:
  \begin{equation*}
    \begin{aligned}
      V_{\delta'}(t)&=\{x|\:d_{xt}\neq d'_{xt}\lor\sigma_{xt}\neq\sigma'_{xt}\} \\
      E_{\delta'}(t)&=\{\{x,y\}|x\in V_{\delta'}(t)\lor y\in V_{\delta'}(t)\}
    \end{aligned}
  \end{equation*}
\end{definition}

\begin{proof}
  Firstly, we find the time complexity of \textproc{DecrementalPart} on Algorithm \ref{algo:decremental-algorithm}.
  \textproc{DecrementalPart} consists of following procedures: find affected vertices by delete, update $d_{st}$ and $\sigma_{st}$ for all $s\in V$, and update $\delta_{t\bullet}(v)$ for all $v\in V$.
  
  \begin{enumerate}[label=(\alph*)]
  \item Finding affected vertices
    \par From Algorithm \ref{algo:decremental-algorithm}, the algorithm iterates through $y\in\mathcal{N}_{G'}(x)$ for all $x\in V_{\delta'}(t)$ and push/pop to/from $\text{WorkSet}$.
    Therefore, if we fix $t$, the time complexity to find affected vertices is:
    \begin{equation}
      \begin{aligned}
        &\mathcal{O}(\sum_{x\in V_{\delta'}(t)}(\lvert\mathcal{N}_G(x)\rvert
        +\log\lvert V_{\delta'}(t)\rvert)) \nonumber\\
        &\:=\mathcal{O}(\lvert E_{\delta'}(t)\rvert
        +\lvert V_{\delta'}(t)\rvert\log\lvert V_{\delta'}(t)\rvert) \nonumber
      \end{aligned}
    \end{equation}

  \item Updating $d_{xt}$ and $\sigma_{xt}$
    \par From Algorithm \ref{algo:decremental-algorithm}, the algorithm iterates through $y\in\mathcal{N}_{G'}(x)$ for all $x\in V_{\delta'}(t)$ and push/pop to/from the priority queue.
    Therefore, if we fix $t$, the time complexity to update $d_{xt}$ and $\sigma_{xt}$ is:
    \begin{equation}
      \begin{aligned}
        &\mathcal{O}(\sum_{x\in V_{\delta'}(t)}(\lvert\mathcal{N}_G(x)\rvert
        +\log\lvert V_{\delta'}(t)\rvert)) \nonumber\\
        &\:=\mathcal{O}(\lvert E_{\delta'}(t)\rvert
        +\lvert V_{\delta'}(t)\rvert\log\lvert V_{\delta'}(t)\rvert) \nonumber
      \end{aligned}
    \end{equation}
    
  \item Updating dependency $\delta_{t\bullet}(v)$
    \par From Algorithm \ref{algo:decremental-algorithm}, the algorithm iterate through $y\in\mathcal{N}_{G'}(x)$ for all $x\in V_{\delta}(t)$ and push/pop to/from the priority queue.
    Therefore, if we fix $t$, the time complexity to update $\delta_{t\bullet}$ is:
    \begin{equation}
      \begin{aligned}
        &\mathcal{O}(\sum_{x\in V_\delta(t)}(\lvert\mathcal{N}_G(x)\rvert
        +\log\lvert V_\delta(t)\rvert)) \nonumber\\
        &\:=\mathcal{O}(\lvert E_\delta(t)\rvert
        +\lvert V_\delta(t)\rvert\log\lvert V_\delta(t)\rvert) \nonumber
      \end{aligned}
    \end{equation}
  \end{enumerate}

  Since $V_\delta(t)\supset V_{\delta'}(t)$ (see Lemma \ref{lm:update-delta-1} and Lemma \ref{lm:update-delta-2},
  time complexity of \textproc{DecrementalPart} is
  \begin{equation*}
    \begin{aligned}
      &\mathcal{O}(
      \lvert E_{\delta'}(t)\rvert+\lvert V_{\delta'}(t)\rvert\log\lvert V_{\delta'}(t)\rvert
      +\lvert E_{\delta'}(t)\rvert+\lvert V_{\delta'}(t)\rvert\log\lvert V_{\delta'}(t)\rvert
      +\lvert E_\delta(t)\rvert+\lvert V_\delta(t)\rvert\log\lvert V_\delta(t)\rvert
      ) \nonumber\\
      &\:=\mathcal{O}(\lvert E_\delta(t)\rvert+\lvert V_\delta(t)\rvert\log\lvert V_\delta(t)\rvert)
    \end{aligned}
  \end{equation*}

  \textproc{DecrementalPart} is called from \textproc{Decremental} for all $t\in V$.
  For vertex $t$ satisfies $t\notin V_\delta$ (i.e. no update occur),
  the time complexity of \textproc{DecrementalPart} is $\mathcal{O}(1)$ since the procedure ends on first if statement.
  Therefore, the time complexity of \textproc{Decremental} is
  \begin{equation*}
    \begin{aligned}
      &\mathcal{O}(
      \sum_{t\in V_\delta}(\lvert E_\delta(t)\rvert+\lvert V_\delta(t)\rvert\log\lvert V_\delta(t)\rvert)
      +\sum_{t\notin V_\delta}1) \nonumber\\
      &=\:\mathcal{O}(\sum_{t\in V_\delta}(\lvert E_\delta\rvert
      +\lvert V_\delta\rvert\log\lvert V_\delta\rvert))\nonumber\\
      &\:=\mathcal{O}(\lvert V_\delta\rvert\lvert E_\delta\rvert
      +\lvert V_\delta\rvert^2\log\lvert V_\delta\rvert) \nonumber.\\
    \end{aligned}
  \end{equation*}
\end{proof}

