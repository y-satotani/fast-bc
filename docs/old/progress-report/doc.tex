\documentclass{jarticle}
\usepackage{amsmath,amsthm,amssymb}
\usepackage[top=2cm,left=1cm,right=1cm,bottom=2cm]{geometry}
\usepackage[dvipdfmx]{graphicx}
\usepackage{algorithm,algpseudocode}
\usepackage{subcaption}

\newtheorem{theorem}{定理}
\newtheorem{corollary}[theorem]{系}
\newtheorem{lemma}[theorem]{補題}
\newtheorem{conjecture}[theorem]{予想}

\title{一辺削除時の媒介中心性の更新方法}
\author{里谷 佳紀}
\date{\today}

\begin{document}
\maketitle

\section{概要}
グラフから辺が削除された時の媒介中心性を更新する二種類の方法を開発する.
ひとつは,頂点間距離と最短経路数を更新し,媒介中心性をその定義から計算する
方法である.
もうひとつは,ある頂点から他の頂点への最短経路を表すグラフを更新し,
Brandesのアルゴリズムに基づく方法で媒介中心性を計算する方法である.

どちらの方法も,\verb|igraph|が実装するBrandesのアルゴリズムより
悪い結果となった.

\section{距離行列の更新アルゴリズム}
グラフ$G=(V,E)$において,頂点$v,w\in V$の距離と最短経路の数をそれぞれ$d(v,w)$と
$\sigma(v,w)$と表す.今回,紙面の都合により定義などは省略する.

グラフ$G$から辺$\{\alpha,\beta\}$を削除した後の頂点間距離$d'(s,t)$と
最短経路数$\sigma'(s,t)$を求める方法をアルゴリズム
\ref{algo:update-distance-on-delete}に示す.

\begin{algorithm}[H]
  \caption{辺$\{\alpha,\beta\}$が削除されたときの$d'(s,t)$と$\sigma'(s,t)$の
    更新}\label{algo:update-distance-on-delete}
  \begin{algorithmic}[1]
    \Require $G=(V,E),\,d(s,t),\,\sigma(s,t)$
    \Ensure $d'(s,t),\,\sigma'(s,t)$
    \State $P\gets()$
    \Comment{更新の対象となる頂点組$(s,t)$}
    \ForAll{$(s,t)\in V\times V,\,s<t$}
    \If{$\sigma_{\alpha\beta}(s,t)=0$}
    \State $d'(s,t)=d(s,t),\:\sigma'(s,t)=\sigma(s,t)$
    \ElsIf{$0<\sigma_{\alpha\beta}(s,t)<\sigma(s,t)$}
    \State $d'(s,t)=d(s,t),\:\sigma'(s,t)=\sigma(s,t)-\sigma_{\alpha\beta}(s,t)$
    \ElsIf{$\sigma_{\alpha\beta}(s,t)=\sigma(s,t)>0$}
    \Comment{頂点間距離を再計算}
    \State $d_{\min}\gets \min\{d(s,v)+d(v,t)\:|\:v\in V,\,\sigma(s,v)>\sigma_{\alpha\beta}(s,v),\,\sigma(v,t)>\sigma_{\alpha\beta}(v,t)\}$
    \If{$d_{\min}=\infty$}
    \State $d'(s,t)\gets\infty,\:\sigma'(s,t)\gets0$
    \Else
    \State $d'(s,t)\gets d_{\min}$
    \State \parbox[t]{\linewidth}{
      $P\gets(\ldots,(s_i,t_i),(s,t),(s_{i+1},t_{i+1}),\ldots)$
      ただし,$d'(s_i,t_i)\leq d'(s,t)\leq d'(s_{i+1},t_{i+1}),\:$
      $\ldots,(s_i,t_i),\ldots\in P$
    }
    \EndIf
    \EndIf
    \EndFor
    \ForAll{$(s_i,t_i)\in P$}
    \Comment{最短経路数を再計算}
    \State $\sigma'(s_i,t_i)\gets\frac{\sum_v\sigma'(s_i,v)\sigma'(v,t_i)}{d'(s_i,t_i)-1}$ただし,$v\in\{v\in V,\,v\neq s_i,t_i,\,d'(s_i,t_i)=d'(s_i,v)+d'(v,t_i)\}$
    \EndFor
  \end{algorithmic}
\end{algorithm}

\section{有向辺集合の更新アルゴリズム}
グラフ$G=(V,E)$において,頂点$s$から他の頂点への最短経路を表す有向グラフを
$G_s=(V,E_s)$と表す.定義から,次が成り立つ.

\begin{corollary}
  グラフ$G=(V,E)$と頂点$s,v,w\in V$において,$(v,w)$が$E_s$に属することの
  必要十分条件は,$\sigma(s,v)=0$かつ$d(v,w)=1$かつ$d(s,v)+1=d(s,w)$を
  満たすことである.
\end{corollary}

グラフ$G$から辺$\{\alpha,\beta\}$が削除されたときの$E_s$の更新について考える.
次の定理は,辺の削除によって$E_s$が変化するための条件を示すものである.

\begin{theorem}
  グラフ$G=(V,E)$から辺$\{\alpha,\beta\}$を削除したとき,
  有向辺集合$E_s$が変化することの必要十分条件は,$d(s,\alpha)<d(s,\beta)$を
  満たすことである.
\end{theorem}

以上の事柄を踏まえ,グラフ$G$から辺$\{\alpha,\beta\}$が削除されたときの
$E_s$の更新の手順をアルゴリズム\ref{algo:update-edgeset-on-delete}に示す.
この手順は,アルゴリズム\ref{algo:update-distance-on-delete}によって
削除後の頂点間距離$d'(s,t)$と最短経路数$\sigma'(s,t)$が計算済みであることを
前提とする.

\begin{algorithm}[H]
  \caption{辺$\{\alpha,\beta\}$が削除されたときの有向辺集合$E_s$の更新}
  \label{algo:update-edgeset-on-delete}
  \begin{algorithmic}[1]
    \Require $G=(V,E),\,d(s,t),\,\sigma(s,t),\,d'(s,t),\,\sigma'(s,t),\,E_s$
    \Ensure $E'_s$
    \State $P$をアルゴリズム\ref{algo:update-distance-on-delete}のものとする.
    \State $P\gets\{\{s_i,t_i\}\:|\:(s_i,t_i)\in P\}$
    \Comment{$P$自身とその要素を集合にする}
    \ForAll{$s\in V$}
    \State $E'_s\gets E_s$
    \If{$d(s,\alpha)=d(s,\beta)$}
    \State \textbf{continue}
    \EndIf
    \State $E'_s\gets E'_s\setminus\{(v,w)\:|\:(v,w)\in E'_s,\,\sigma'(s,v)=0\lor d'(s,v)+1\neq d'(s,w)\}$
    \State $E'_s\gets E'_s\cup\{(v,w)\:|\:\{s,w\}\in P,\,v\in N(w),\,\sigma'(s,v)>0\land d'(s,v)+1=d'(s,w)\}$
    \EndFor
  \end{algorithmic}
\end{algorithm}

\section{媒介中心性の更新アルゴリズム}
更新された頂点間距離$d'(s,t)$と最短経路数$\sigma'(s,t)$を用いて
媒介中心性を更新する手順をアルゴリズム\ref{algo:update-bc-distance}に示す.

\begin{algorithm}[H]
  \caption{辺$\{\alpha,\beta\}$が削除されたときの頂点$i$の媒介中心性$B_i$の更新}
  \label{algo:update-bc-distance}
  \begin{algorithmic}[1]
    \Require $G=(V,E),\,d(s,t),\,\sigma(s,t)$
    \Ensure $B'_i$
    \State $d'(s,t)$と$\sigma'(s,t)$をアルゴリズム\ref{algo:update-distance-on-delete}によって求める
    \State $B_i\gets\sum_{s\in V}\sum_{t\in V}\frac{\sigma(s,i)\sigma(i,t)}{\sigma(s,t)}$ただし,$d(s,i)+d(i,t)=d(s,t)$
  \end{algorithmic}
\end{algorithm}

更新された有向辺の集合$E'_s$から媒介中心性を更新するアルゴリズムを
アルゴリズム\ref{algo:update-bc-edgeset}に示す.

\begin{algorithm}[H]
  \caption{辺$\{\alpha,\beta\}$が削除されたときの頂点$i$の媒介中心性$B_i$の更新}
  \label{algo:update-bc-edgeset}
  \begin{algorithmic}[1]
    \Require $G=(V,E),\,d(s,t),\,\sigma(s,t),\,E_s$
    \Ensure $B'_i$
    \State $d'(s,t)$と$\sigma'(s,t)$をアルゴリズム\ref{algo:update-distance-on-delete}によって求める
    \State $E'(s,t)$をアルゴリズム\ref{algo:update-edgeset-on-delete}によって求める.
    \State $B_i\gets 0$
    \ForAll{$s\in V$}
    \State $\delta_i\gets 0$
    \State $E_s$を$d'(i,j)$(ただし$(i,j)\in E_s$)の降順に並べ替える
    \ForAll{$(i,j)\in E_s$}
    \State $\delta_i\gets\delta_i+\frac{\sigma'(s,i)}{\sigma'(s,j)}(1+\delta_j)$
    \EndFor
    \State $B_i\gets B_i+\delta_i$
    \EndFor
    \State $B_i\gets\frac{B_i}{2}$
  \end{algorithmic}
\end{algorithm}

\section{実験}
定めたアルゴリズムの有効性を検証するため,次の実験を行う.
次の方法について,一辺削除ごとの更新に要する時間を測定する.
アルゴリズム\ref{algo:update-bc-distance}については,予めグラフ$G$から
$d(s,t)$と$\sigma(s,t)$を計算しておき,
アルゴリズム\ref{algo:update-bc-distance}の実行時間を測定する.
アルゴリズム\ref{algo:update-bc-edgeset}については,予めグラフ$G$から
$d(s,t)$と$\sigma(s,t)$と$E_s$を計算しておき,
アルゴリズム\ref{algo:update-bc-edgeset}の実行時間を測定する.
比較対象として,\verb|igraph|ライブラリが提供する
\verb|igraph_betweenness|関数を用いる.

実験にはErd\"{o}s-R\'{e}nyiグラフを用いる.頂点数は100,300,500で,
グラフの密度を操作するパラメータ$p$は0.1,0.4,0.7とする.
それぞれの方法と頂点数,およびパラメータについて,10回試行した.

\section{結果}
結果を図\ref{fig:exp-random}に示す.アルゴリズム\ref{algo:update-bc-distance}
は,密度が増加しても計算時間が増加することはない.
アルゴリズム\ref{algo:update-bc-edgeset}の計算時間が予想より長いので,
媒介中心性の性質を応用した更なる高速化が必要である.

\begin{figure}
  \centering
  \includegraphics{random.pdf}
  \captionsetup{justification=centering}
  \caption{ランダムグラフでの各アルゴリズムの実行時間 \\
    {\footnotesize 図中のbrandesは比較対象を,
      dmはアルゴリズム\ref{algo:update-bc-distance}を
      esはアルゴリズム\ref{algo:update-bc-edgeset}を表す.} \\
    {\footnotesize 時間の単位は秒で,対数スケールである.}
  }
  \label{fig:exp-random}
\end{figure}

\end{document}
